\section{Semana 9 (05/09 -- 05/15)}

\note[Álgebra]{Lunes\\2022-05-09}

\begin{probMG}[IMO Shortlist 2001 A4]
	Find all functions $f:\RR\to\RR$, satisfying
	\[f(xy)(f(x)-f(y))=(x-y)f(x)f(y)\]
	for all $x,y$.
\end{probMG}

\begin{proof}
	Con $(x,y)=(1,0)$ y $(x,1)$ tenemos que $f(0)=0$ y que $f(x)=0$ o $f(x)=xf(1)$. Luego, $f(x)=xc$ para todo $x\in G$, y $f(x)=0$ para todo $x\not\in G$ donde $(G,\cdot)$ es un grupo multiplicativo y $c$ es una constante.
\end{proof}

\begin{probMB}[IMO Shortlist 2001 A5]
	Find all positive integers $a_1,a_2,\dots,a_n$ such that
	\[\frac{99}{100}=\frac{a_0}{a_1}+\frac{a_1}{a_2}+\dots+\frac{a_{n-1}}{a_n},\]
	where $a_0=1$ and $(a_{k+1}-1)a_{k-1}\ge a_k^2(a_k-1)$ for $k=1,2,\dots,n-1$.
\end{probMB}

\note[Combinatoria]{Martes\\2022-05-10}

\begin{probEG}[IMO Shortlist 1998]
	Determine the smallest integer $n$, $n\ge 4$, for which one can choose four different numbers $a,b,c,d$ from any $n$ distinct integers such that $a+b-c-d$ is divisible by $20$.
\end{probEG}

\begin{probER}
	Form a $2000\times 2002$ screen with unit screens. Initially, there are more than $1999\times 2001$ unit screens which are \emph{on}. In any $2\times 2$ screen, as soon as there are $3$ unit screens which are \emph{off}, the $4^\text{th}$ screen turns off automatically. Prove that the whole screen can never be totally off.
\end{probER}

\begin{proof}
	Pista: considere a la cantidad de cuadrados de $2\times 2$.
\end{proof}

\begin{probMR}
	Given an initial sequence $a_1,a_2,\dots,a_n$ of real numbers, we perform a series of steps. At each step, we replace the current sequence $x_1,x_2,\dots,x_n$ with $\abs{x_1-a},\abs{x_2-a},\dots,\abs{x_n-a}$ for some $a$. For each step, the value of $a$ can be different.
	\begin{enumerate}[(a)]
		\ii Prove that it is always possible to obtain the null sequence consisting of all $0$'s.
		\ii Determine with proof the minimum number of steps required, regardless of initial sequence, to obtain the null sequence.
	\end{enumerate}
\end{probMR}

\note[Teoría de Números]{Miércoles\\2022-05-11}

\begin{probEG}
	Sean $1\le a_1,a_2,\dots,a_{10^6}\le 9$ algunos dígitos. Demuestre que el conjunto
	\[A=\left\{\ol{a_1a_2\cdots a_k}:1\le k\le 10^6\right\}\]
	contiene a lo más $100$ cuadrados perfectos.
\end{probEG}

\begin{proof}
	Por el absurdo, supongamos que existen más de $100$ índices $k$ tales que $\ol{a_1a_2\cdots a_k}$ es un cuadrado perfecto. Entonces, hay $50$ de la misma paridad. Como $2^{50}>2^{20}=1024^2>1000^2=10^6$, existen índices $j<i<2j$ tales que $2\mid i-j$ y
	\begin{align*}
		\ol{a_1a_2\cdots a_i}
		&=\ol{a_1a_2\cdots a_j}\cdot 10^{i-j}+\ol{a_{j+1}a_{j+2}\cdots a_i}\\
		&=\left(n\cdot 10^\frac{i-j}{2}\right)^2+\ol{a_{j+1}a_{j+2}\cdots a_i}\\
		&\ge\left(n\cdot 10^\frac{i-j}{2}+1\right)^2
	\end{align*}
	de donde
	\[10^{i-j}>\ol{a_{j+1}a_{j+2}\cdots a_i}>2n\cdot 10^\frac{i-j}{2}>10^\frac{i-1}{2}\]
	y $i\ge 2j$, lo cual es un absurdo.
\end{proof}

\note{Viernes\\2022-05-13}

\begin{probMG}[IberoAmerican 2012/6]
	Demostrar que, para todo entero positivo $n$, existen $n$ enteros positivos consecutivos tales que ninguno es divisible por la suma de sus respectivos dígitos.
	\begin{hint}
		Show that, for every positive integer $n$, there exist $n$ consecutive positive integers such that none is divisible by the sum of its digits.
	\end{hint}
	\aops{2814652}
	\omaforos{1201}
\end{probMG}

\begin{proof}
	Procederemos por inducción sobre $n$. Si $n=1$, es claro que $s(11)\nmid 11$ donde $s(x)$ denota la suma de dígitos de $x$. Ahora, supongamos que existe un $m\in\ZZ^+_0$ tal que $s(m+i)\nmid m+i$ para todo $1\le i\le n$. Sea $m'=M\cdot 10^k+m$ donde $M$ es un número formado por $N!$ bloques consecutivos de $N!$, siendo $k,N\in\ZZ^+$ suficientemente grandes. Si $s(m'+i)\mid m'+i$ para algún $1\le i\le n$, entonces
	\[s(m+i)\mid N!\cdot s(N!)+s(m+i)=s(m'+i)\mid m'+i=M\cdot 10^k+m+i\]
	de donde $s(m+i)\mid m+i$, absurdo. Si $s(m'+n+1)\mid m'+n+1$ para $k$ y $k+1$, entonces $s(M)+s(m+n+1)$ divide simultáneamente a $M\cdot 10^k+m+n+1$ y a $M\cdot 10^{k+1}+m+n+1$. Luego,
	\[N!\cdot s(N!)+s(m+n+1)\mid 9(m+n+1)\]
	lo cual no es posible para un valor grande de $N$. Por lo tanto, existe un $m'\in\ZZ^+_0$ tal que $s(m'+i)\nmid m'+i$ para todo $1\le i\le n+1$.
\end{proof}

\begin{probEG}
	Hallar la menor cantidad de dígitos de un número $N$ que contiene a todas las permutaciones de $1234$.
\end{probEG}

\begin{proof}
	Vamos a dividir las $24$ permutaciones en $6$ ciclos. Digamos que dos bloques de $4$ son \emph{adyacentes} si esos bloques son de la forma $abcd$ y $bcde$ donde los dígitos $a,b,c,d,e$ son consecutivos. Si dos bloques son de distintos ciclos, esos bloques no pueden ser adyacentes. Luego, si $t$ es la cantidad de bloques que no son permutaciones de $1234$, entonces $n-3\ge 24+t$. Como existen a lo sumo $t+1$ ciclos, entonces $t+1\ge 6$ de donde $n\ge 27+t\ge 32$, pero es fácil ver que no se da la igualdad. Un ejemplo con $33$ es \[N=123412314231243121342132413214321.\]
\end{proof}

\begin{problem}
	Un número es \emph{ascendente} si sus dígitos están en orden creciente, de izquierda a derecha. Por ejemplo, $123$ y $11224$ son ascendentes. Demuestre que para todo entero positivo $n$, es posible encontrar un cuadrado perfecto de exactamente $n$ dígitos que sea ascendente.
\end{problem}

\begin{problem}
	Demostrar que para cualquier entero positivo $N$, al menos uno de los números $N$ y $N+1$ se puede representar de la forma $k+s(k)$ para algún entero positivo $k$.
\end{problem}

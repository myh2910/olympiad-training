\section{Semana 13 (06/06 -- 06/12)}

\note[Álgebra]{Martes\\2022-06-07}

\begin{probMB}[ISL 2009/A5]
	Sea $f:\RR\to\RR$ una función. Demuestre que existen $x,y\in\RR$ tales que
	\[f(x-f(y))>yf(x)+x.\]
	\forum[aops]{1932913}
	\begin{hint}
		Let $f$ be any function that maps the set of real numbers into the set of real numbers. Prove that there exist real numbers $x$ and $y$ such that
		\[f(x-f(y))>yf(x)+x.\]
	\end{hint}
\end{probMB}

\begin{proof}
	Supongamos por el absurdo que $f(x-f(y))\le yf(x)+x$ para todo $x,y\in\RR$. Si $y=0$, tenemos $f(x)\le x+f(0)$ de donde $f(x)\le 0$ para todo $x\le -f(0)$. Si $(x,y)=(f(x),x)$ y $x\ge 0$, entonces
	\begin{equation}\label{eq:func_ineq}
		f(0)\le xf(f(x))+f(x)
	\end{equation}
	Ahora, si $f(x)\le -f(0)$ tenemos $f(f(x))\le 0$, de donde $f(0)\le f(x)$. Es decir, $f(x)\ge-\abs{f(0)}$ para todo $x\ge 0$. Ahora, digamos que $x\in\RR$ es un número cualquiera. Luego, existe un $y\to-\infty$ tal que $f(y)<x$. Pero si $f(x)>0$, entonces
	\[-\abs{f(0)}\le f(x-f(y))\le yf(x)+x\]
	lo cual es un absurdo. Por ende, $f(x)\le 0$ para todo $x\in\RR$. De la ecuación \eqref{eq:func_ineq} tenemos que $f(x)\ge f(0)$ para todo $x\ge 0$. Ahora, si existe algún $x\ge 0$ tal que $f(x)<0$, sea $y\in\RR^+$ suficientemente grande. Luego,
	\[f(0)\le f(x-f(y))\le yf(x)+x\]
	lo cual es un absurdo. Por ende, $f(x)=0$ para todo $x\ge 0$. Ahora, si existe algún $x<0$ tal que $f(x)<0$, sea $y\in\RR^+$ suficientemente grande. Luego,
	\[-f(x)(y-1)\le x\]
	lo cual es un absurdo. Por lo tanto, $f\equiv 0$ pero claramente esto es un absurdo.
\end{proof}

\begin{probMB}[ISL 2009/A6]
	Suponga que la secuencia $s_1,s_2,s_3,\dots$ es una secuencia estrictamente creciente de enteros positivos tal que las subsecuencias
	\[s_{s_1},s_{s_2},s_{s_3},\dots\quad\text{y}\quad s_{s_1+1},s_{s_2+1},s_{s_3+1},\dots\]
	son ambas progresiones aritméticas. Demuestre que $s_1,s_2,s_3,\dots$ es también una progresión aritmética.
	\forum[aops]{1561573}
	\begin{hint}
		Suppose that $s_1,s_2,s_3,\dots$ is a strictly increasing sequence of positive integers such that the sub-sequences
		\[s_{s_1},s_{s_2},s_{s_3},\dots\quad\text{and}\quad s_{s_1+1},s_{s_2+1},s_{s_3+1},\dots\]
		are both arithmetic progressions. Prove that the sequence $s_1,s_2,s_3,\dots$ is itself an arithmetic progression.
	\end{hint}
\end{probMB}

\begin{proof}
	Sea $s_{s_i}=a_1+(i-1)d_1$ y $s_{s_i+1}=a_2+(i-1)d_2$ para todo $i\in\ZZ^+$. Luego,
	\[s_{s_i}=a_1+(i-1)d_1<s_{s_i+1}=a_2+(i-1)d_2\le s_{s_{i+1}}=a_1+id_1\]
	de donde $d_1\le d_2$, $a_1<a_2$ y $d_2\le d_1$ de donde $d_1=d_2=d$. Entonces,
	\[0<s_{i+1}-s_i\le s_{s_{i+1}}-s_{s_i}=d\]
	y en efecto sean $N$ y $M$ el valor mínimo y máximo de $s_{i+1}-s_i$. Si $M=s_{k+1}-s_k$, entonces
	\[N\cdot M=N\cdot(s_{k+1}-s_k)\le s_{s_{k+1}}-s_{s_k}=d\]
	y análogamente $N\cdot M\ge d$ de donde $d=N\cdot M$ y como se cumple la igualdad, $N=s_{s_k+1}-s_{s_k}=a_2-a_1$ y análogamente $M=a_2-a_1$ de donde $s_{i+1}-s_i$ es constante.
\end{proof}

\begin{probMG}[ISL 2009/A7]
	Determine todas las funciones $f:\RR\to\RR$ tales que
	\[f(xf(x+y))=f(yf(x))+x^2\]
	para todo $x,y\in\RR$.
	\forum[aops]{1932915}
	\begin{hint}
		Find all functions $f$ from the set of real numbers into the set of real numbers which satisfy for all $x,y$ the identity
		\[f(xf(x+y))=f(yf(x))+x^2.\]
	\end{hint}
\end{probMG}

\begin{proof}
	Si $x=0$, tenemos $f(0)=f(yf(0))$ de donde si $f(0)\ne 0$ entonces $f$ es constante, lo cual es un absurdo. Por ende, $f(0)=0$. Si $y=0$, tenemos
	\[f(xf(x))=f(0)+x^2=x^2.\]
	Si $f(x)=0$, $0=f(0)=x^2$ de donde $x=0$. Si $y=-x$, tenemos $f(-xf(x))=-x^2$ de donde $f$ es suryectiva. Ahora, si $f(a)=f(b)$ entonces sea $(x,y)=(a,b-a)$, luego
	\[a^2=f(af(b))=f((b-a)f(a))+a^2\]
	de donde $f((b-a)f(a))=0$ y $a=b$. Es decir, $f$ es inyectiva. Luego,
	\[f(xf(x))=x^2=f(-xf(-x))\]
	de donde $f(-x)=-f(x)$. Si $(x,y)=(x,2x)$ y $(x,y)=(2x,-x)$, entonces
	\[f(2xf(x))=f(-xf(2x))+4x^2=4x^2-(2x^2)=f(xf(2x))\]
	de donde $f(2x)=2f(x)$. Sea $k\in\RR$ tal que $f(k)=1$. Luego,
	\[1=f(k)=f(kf(k))=k^2\]
	de donde $k=\pm 1$, es decir, $f(1)=\pm 1$. Si $(x,y)=(x,1)$ y $(x,y)=(x+1,-x)$ entonces $f(xf(x+1))=f(f(x))+x^2$ y $f(1)f(x+1)=-f(xf(x+1))+(x+1)^2$ de donde $f(f(x))=2x+1-f(1)f(x+1)$. Si $(x,y)=(1,x-1)$ entonces $f(f(x))=f(1)f(x-1)+1$ de donde $2x=f(1)(f(x+1)+f(x-1))$. Si $x=2x+1$, entonces
	\[2(2x+1)=f(1)(f(2x+2)+f(2x))=2f(1)(f(x+1)+f(x))\]
	de donde $f(1)f(x+1)=2x+1-f(1)f(x)$. Es decir,
	\[f(f(x))=2x+1-f(1)f(x+1)=f(1)f(x)\]
	de donde $f(x)=\pm x$ para todo $x\in\RR$.
\end{proof}

\begin{probEB}[ISL 2010/A2]
	Sean $a,b,c,d\in\RR$ tales que $a+b+c+d=6$ y $a^2+b^2+c^2+d^2=12$. Demuestre que
	\[36\le 4(a^3+b^3+c^3+d^3)-(a^4+b^4+c^4+d^4)\le 48.\]
	\forum[aops]{2362276}
	\begin{hint}
		Let the real numbers $a,b,c,d$ satisfy the relations $a+b+c+d=6$ and $a^2+b^2+c^2+d^2=12$. Prove that
		\[36\le 4(a^3+b^3+c^3+d^3)-(a^4+b^4+c^4+d^4)\le 48.\]
	\end{hint}
\end{probEB}

\begin{proof}
	Note que
	\[\cycsum(a-1)^2=\cycsum a^2-2\cycsum a+4=4\]
	de donde
	\[4\le\cycsum(a-1)^4\le 16.\]
	Como
	\[(x-1)^4=x^4-4x^3+6x^2-4x+1\]
	entonces
	\[4\cycsum a^3-\cycsum a^4=6\cycsum a^2-4\cycsum a+4-\cycsum(a-1)^4=52-\cycsum(a-1)^4\]
	de donde
	\[36\le 4\cycsum a^3-\cycsum a^4\le 48.\]
\end{proof}

\begin{probMG}[ISL 2010/A3]
	Sean $x_1,x_2,\dots,x_{100}\in\RR_0^+$ tales que $x_i+x_{i+1}+x_{i+2}\le 1$ para todo $i=1,2,\dots,100$ (donde $x_{101}=x_1$, $x_{102}=x_2$). Determine el máximo valor de $S=\sum_{i=1}^{100}x_ix_{i+2}$.
	\forum[aops]{2362280}
	\begin{hint}
		Let $x_1,\dots,x_{100}$ be nonnegative real numbers such that $x_i+x_{i+1}+x_{i+2}\le 1$ for all $i=1,\dots,100$ (we put $x_{101}=x_1,x_{102}=x_2$). Find the maximal possible value of the sum $S=\sum^{100}_{i=1}x_ix_{i+2}$.
	\end{hint}
\end{probMG}

\begin{proof}
	Note que
	\[x_i^2+x_ix_{i+1}+x_ix_{i+2}\le x_i\]
	de donde
	\[S\le\half\sum(x_i+x_{i+1})(1-x_i-x_{i+1})\le\frac12\cdot\frac14\cdot 100=\frac{25}{2}\]
	y un ejemplo es cuando
	\[x_i=\begin{cases}
		\half&\text{si }2\mid i,\\
		0&\text{si }2\nmid i.
	\end{cases}\]
\end{proof}

\begin{probEG}[ISL 2010/A4]
	Una secuencia $x_1,x_2,\dots$ está definida por $x_1=1$ y $x_{2k}=-x_k,\,x_{2k-1}=(-1)^{k+1}x_k$ para todo $k\ge 1$. Demuestre que $x_1+x_2+\dots+x_n\ge 0$ para todo $n\ge 1$.
	\forum[aops]{2362283}
	\begin{hint}
		A sequence $x_1,x_2,\dots$ is defined by $x_1=1$ and $x_{2k}=-x_k,\,x_{2k-1}=(-1)^{k+1}x_k$ for all $k\ge 1$. Prove that $x_1+x_2+\dots+x_n\ge 0$ for all $n\ge 1$.
	\end{hint}
\end{probEG}

\begin{proof}
	Pista: inducción de $x\mapsto 4x,4x+1,4x+2,4x+3$.
\end{proof}

\begin{probEG}[ISL 2010/A5]
	Determine todas las funciones $f:\QQ^+\to\QQ^+$ tales que
	\[f(f(x)^2y)=x^3f(xy)\]
	para todo $x,y\in\QQ^+$.
	\forum[aops]{2362286}
	\begin{hint}
		Determine all functions $f:\QQ^+\to\QQ^+$ which satisfy the following equation for all $x,y\in\QQ^+$:
		\[f(f(x)^2y)=x^3f(xy).\]
	\end{hint}
\end{probEG}

\begin{proof}
	Si $y=1$ entonces $f(f(x)^2)=x^3f(x)$ por lo que $f$ es inyectiva. Luego, si $y=f(y)^2$ entonces
	\[f(f(x)^2f(y)^2)=x^3f(xf(y)^2)=x^3y^3f(xy)=f(f(xy)^2)\]
	de donde $f$ es multiplicativa. Si $g(x)=\frac{f(f(x))}{x}$, tenemos $f(x)=xg(x)^2$ de donde $g(g(x))^4=g(x)^5$ y esto significa que si $g(x)\ne 1$ y $N$ es el mínimo común múltiplo de todos los exponentes primos de $g(x)$, entonces $4N\mid 5N$ lo cual es un absurdo. Es decir, $g(x)=1$ de donde $f(x)=x$ para todo $x\in\QQ^+$.
\end{proof}

\begin{probHR}[ISL 2010/A6]
	Sean $f,g:\ZZ^+\to\ZZ^+$ tales que $f(g(n))=f(n)+1$ y $g(f(n))=g(n)+1$ para todo $n\in\ZZ^+$. Demuestre que $f(n)=g(n)$ para todo $n\in\ZZ^+$.
	\forum[aops]{2362289}
	\begin{hint}
		Suppose that $f$ and $g$ are two functions defined on the set of positive integers and taking positive integer values. Suppose also that the equations $f(g(n))=f(n)+1$ and $g(f(n))=g(n)+1$ hold for all positive integers. Prove that $f(n)=g(n)$ for all positive integer $n$.
	\end{hint}
\end{probHR}

\note{Miércoles\\2022-06-08}

\begin{probHR}[ISL 2010/A7]
	Sean $a_1,a_2,\dots,a_r$ números reales positivos. Para $n>r$, inductivamente definimos
	\[a_n=\max_{1\le k\le n-1}(a_k+a_{n-k}).\]
	Demuestre que existen enteros positivos $\ell\le r$ y $N$ tales que $a_n=a_{n-\ell}+a_\ell$ para todo $n\ge N$.
	\forum[aops]{1936918}
	\begin{hint}
		\begin{otherlanguage*}{english}
			Let $a_1,\dots,a_r$ be positive real numbers. For $n>r$, we inductively define
			\[a_n=\max_{1\le k\le n-1}(a_k+a_{n-k}).\]
			Prove there exist positive integers $\ell\le r$ and $N$ such that $a_n=a_{n-\ell}+a_\ell$ for all $n\ge N$.
		\end{otherlanguage*}
	\end{hint}
\end{probHR}

\begin{probMR}[ISL 2010/A8]
	Sean $a,b,c,d,e,f\in\RR^+$ tales que $a<b<c<d<e<f$. Consideremos $a+c+e=S$ y $b+d+f=T$. Demuestre que
	\[2ST>\sqrt{3(S+T)\left(S(bd+bf+df)+T(ac+ae+ce)\right)}.\]
	\forum[aops]{2362291}
	\begin{hint}
		Given six positive numbers $a,b,c,d,e,f$ such that $a<b<c<d<e<f$. Let $a+c+e=S$ and $b+d+f=T$. Prove that
		\[2ST>\sqrt{3(S+T)\left(S(bd+bf+df)+T(ac+ae+ce)\right)}.\]
	\end{hint}
\end{probMR}

\begin{probEG}[ISL 2011/A2]
	Determine todas las secuencias $(x_1,$ $x_2,\dots,x_{2011})$ de enteros positivos tales que para todo entero positivo $n$ existe un entero positivo $a$ que satisface:
	\[x_1^n+2x_2^n+\dots+2011x_{2011}^n=a^{n+1}+1.\]
	\forum[aops]{2737640}
	\begin{hint}
		Determine all sequences $(x_1,x_2,\dots,x_{2011})$ of positive integers such that for every positive integer $n$ there is an integer $a$ with
		\[x_1^n+2x_2^n+\dots+2011x_{2011}^n=a^{n+1}+1.\]
	\end{hint}
\end{probEG}

\begin{probMG}[ISL 2011/A3]
	Determine todas las parejas de funciones $(f,g)$, con $f,g:\RR\to\RR$, tales que
	\[g(f(x+y))=f(x)+(2x+y)g(y)\]
	para todo $x,y\in\RR$.
	\forum[aops]{2737643}
	\begin{hint}
		Determine all pairs $(f,g)$ of functions from the set of real numbers to itself that satisfy
		\[g(f(x+y))=f(x)+(2x+y)g(y)\]
		for all real numbers $x$ and $y$.
	\end{hint}
\end{probMG}

\begin{problem}[ISL 2011/A4]
	Determine todas las parejas de funciones $(f,g)$, con $f,g:\ZZ^+\to\ZZ^+$, tales que
	\[f^{g(n)+1}(n)+g^{f(n)}(n)=f(n+1)-g(n+1)+1\]
	para todo $n\in\ZZ^+$.\\[4pt]
	\emph{Nota:} $f^k(n)=\underbrace{f(f(\dots f}_{k\text{ veces}}(n)\dots))$.
	\forum[aops]{2737644}
	\begin{hint}
		Determine all pairs $(f,g)$ of functions from the set of positive integers to itself that satisfy
		\[f^{g(n)+1}(n)+g^{f(n)}(n)=f(n+1)-g(n+1)+1\]
		for every positive integer $n$. Here, $f^k(n)$ means $\underbrace{f(f(\dots f}_k(n)\dots))$.
	\end{hint}
\end{problem}

\begin{proof}
	Sea $a_1\in\ZZ^+$ tal que $f(a_1)$ es mínimo. Note que si $a_1>1$, con $n=a_1-1$ tenemos $f(f^{g(n)}(n))<f(a_1)$ lo cual es un absurdo, por lo que $a_1=1$. Ahora, si $a_2>1$ es tal que $f(a_2)$ es el segundo mínimo, entonces con $n=a_2-1$ tenemos $f(f^{g(n)}(n))<f(a_2)$ de donde
	\[f(f^{g(n)-1}(n))=f^{g(n)}(n)=1.\]
	Es decir, $f^{g(n)-1}(n)=1$ y así sucesivamente hasta que $n=1$, por lo que $a_2=2$. Podemos probar de manera similar que $a_3=3,a_4=4,\dots$ y que $f(a_2)=2,f(a_3)=3,\dots$ de donde $f(n)=n$ para todo $n\in\ZZ^+$. Luego, $g^{f(n)}(n)+g(n+1)=2$ de donde $g(n)=1$ para todo $n\in\ZZ^+$.
\end{proof}

\note[Teoría de Números]{Jueves\\2022-06-09}

\begin{probEG}[ISL 2010/N2]
	Determine el mayor entero $n$ para el cual existe un conjunto $\{s_1,s_2,\dots,s_n\}$ que consiste de $n$ enteros positivos que satisfacen
	\[\left(1-\frac{1}{s_1}\right)\left(1-\frac{1}{s_2}\right)\cdots\left(1-\frac{1}{s_n}\right)=\frac{51}{2010}.\]
	\forum[aops]{2361998}
	\begin{hint}
		Find the least positive integer $n$ for which there exists a set $\{s_1,s_2,\dots,s_n\}$ consisting of $n$ distinct positive integers such that
		\[\left(1-\frac{1}{s_1}\right)\left(1-\frac{1}{s_2}\right)\cdots\left(1-\frac{1}{s_n}\right)=\frac{51}{2010}.\]
	\end{hint}
\end{probEG}

\begin{proof}
	Note que
	\[\frac{51}{2010}\ge\frac12\cdot\frac23\cdots\frac{n}{n+1}=\frac{1}{n+1}\]
	de donde $n\ge 39$, y un ejemplo es $(s_1,\dots,s_{39})=(2,3,\dots,33,35,36,\dots,40,67)$.
\end{proof}

\begin{probEB}[ISL 2010/N3]
	Determine el menor entero positivo $n$ para el cual existen polinomios $f_1,f_2,\dots,f_n$ con coeficientes racionales tales que
	\[x^2+7=f_1(x)^2+f_2(x)^2+\dots+f_n(x)^2.\]
	\forum[aops]{2362006}
	\begin{hint}
		Find the smallest number $n$ such that there exist polynomials $f_1,f_2,\dots,f_n$ with rational coefficients satisfying
		\[x^2+7=f_1(x)^2+f_2(x)^2+\dots+f_n(x)^2.\]
	\end{hint}
\end{probEB}

\begin{proof}
	Es claro que $\deg{f_i}\le 1$ así que sea $f_i(x)=a_ix+b_i$ donde $a_i,b_i\in\QQ$ para todo $1\le i\le 5$. Luego, $\sum a_i^2=1$, $\sum a_ib_i=0$ y $\sum b_i^2=7$. Vamos a multiplicar los $a_i$'s y $b_i$'s por $D\in\ZZ^+$ para que sean todos enteros (digamos que $D$ es el mínimo). Luego,
	\[\sum(a_i+b_i)^2=\sum(a_i-b_i)^2=8D^2.\]
	Si $n\le 4$, como $8\mid\sum(a_i\pm b_i)^2$ entonces $2\mid a_i-b_i$, de donde
	\[D^2=\sum a_i^2\equiv\sum b_i^2=7D^2\pmod 4.\]
	Es decir, $4\mid 6D^2$ de donde $2\mid D$. Luego,
	\[8\mid 2D^2=\sum\left(\frac{a_i\pm b_i}{2}\right)^2\]
	de donde $4\mid a_i\pm b_i$ y
	\[4\mid(a_i+b_i)+(a_i-b_i)=2a_i\]
	de donde $a_i$'s y $b_i$'s son pares. Es decir, podemos dividir los $a_i$'s, los $b_i$'s y el $D$ por $2$, lo cual es un absurdo a la minimalidad de $D$. Por ende, $n\ge 5$ y un ejemplo es $(f_1,\dots,f_5)=(x,2,1,1,1)$.
\end{proof}

\begin{probEG}[ISL 2010/G3\protect\footnote{Peru IMO TST 2011/2}]
	Sea $A_1A_2\cdots A_n$ un polígono convexo. El punto $P$ es escogido en el interior del polígono de tal manera que sus proyecciones $P_1,P_2,\dots,P_n$ sobre las rectas $A_1A_2,A_2A_3,\dots,A_nA_1$, respectivamente, pertenecen a los lados del polígono (y no a sus prolongaciones). Pruebe que si $X_1,\dots,X_n$ son puntos arbitrarios que pertenecen a los lados $A_1A_2,\dots,A_nA_1$, respectivamente, se cumple la desigualdad:
	\[\max\left\{\frac{X_1X_2}{P_1P_2},\dots,\frac{X_nX_1}{P_nP_1}\right\}\ge 1.\]
	\forum[aops]{2361975}
	\begin{hint}
		\begin{otherlanguage*}{english}
			Let $A_1A_2\cdots A_n$ be a convex polygon. Point $P$ inside this polygon is chosen so that its projections $P_1,\dots,P_n$ onto lines $A_1A_2,\dots,A_nA_1$ respectively lie on the sides of the polygon. Prove that for arbitrary points $X_1,\dots,X_n$ on sides $A_1A_2,\dots,A_nA_1$ respectively,
			\[\max\left\{\frac{X_1X_2}{P_1P_2},\dots,\frac{X_nX_1}{P_nP_1}\right\}\ge 1.\]
		\end{otherlanguage*}
	\end{hint}
\end{probEG}

\begin{proof}
	Supongamos por el absurdo que $X_iX_{i+1}<P_iP_{i+1}$ para todo índice $1\le i\le n$.
	Note que si $X_i$ está en el segmento $A_iP_i$, entonces $X_{i+1}$ está en el segmento $A_{i+1}P_{i+1}$ y así sucesivamente. Por ende, supongamos sin pérdida de generalidad que $X_i$ está en el segmento $A_iP_i$ para todo $1\le i\le n$. Es claro que existe un $i$ tal que $\angle PX_iP_i\le\angle PX_{i+1}P_{i+1}$ (digamos $i=1$). Ahora, sea $Y$ un punto en el segmento $A_1P_1$ tal que $\angle PYP_1=\angle PX_2P_2$. Luego, $\triangle PYP_1\sim\triangle PX_2P_2$ de donde $\triangle PYX_2\sim\triangle PP_1P_2$. Por ende,
	\[X_1X_2\ge YX_2=\frac{PY}{PP_1}\cdot P_1P_2\ge P_1P_2\]
	lo cual es un absurdo.
\end{proof}

\begin{probMG}[ISL 2010/N4\protect\footnote{Peru IMO TST 2011/3}]
	Sean $a,b$ números enteros, y sea $P(x)=ax^3+bx$. Dado un entero positivo $n$, decimos que el par ordenado $(a,b)$ es \emph{n-bueno} si $n\mid P(m)-P(k)$ implica que $n\mid m-k$ para todos los enteros $m,k$. Decimos también que el par $(a,b)$ es \emph{muy bueno} si $(a,b)$ es $n$-bueno para infinitos enteros positivos $n$.
	\begin{enumerate}[(a)]
		\ii Encuentre un par $(a,b)$ que sea $51$-bueno, pero que no sea muy bueno.
		\ii Demuestre que todos los pares $2010$-buenos también son muy buenos.
	\end{enumerate}
	\forum[aops]{2362008}
	\begin{hint}
		Let $a,b$ be integers, and let $P(x)=ax^3+bx$. For any positive integer $n$ we say that the pair $(a,b)$ is \emph{n-good} if $n\mid P(m)-P(k)$ implies $n\mid m-k$ for all integers $m,k$. We say that $(a,b)$ is \emph{very good} if $(a,b)$ is $n$-good for infinitely many positive integers $n$.
		\begin{enumerate}[(a)]
			\ii Find a pair $(a,b)$ which is $51$-good, but not very good.
			\ii Show that all $2010$-good pairs are very good.
		\end{enumerate}
	\end{hint}
\end{probMG}

\begin{proof}
	Para la parte (a), considere el polinomio $P(x)=x^3-51^2x$ con $(a,b)=(1,-51^2)$. Luego, si $51\mid P(m)-P(k)$ entonces
	\[m\equiv m^3\equiv k^3\equiv k\pmod 3\]
	y
	\[m\equiv m^{33}\equiv k^{33}\equiv k\pmod 17\]
	lo cual implica que $51\mid m-k$. Por ende, $(a,b)$ es $51$-bueno. Si $n>51$ y $(m,k)=(51,0)$, note que $n\mid 0=P(m)-P(k)$ pero $n\nmid 51=m-k$, así que $(a,b)$ no es muy bueno.

	Para la parte (b), note que $P(m)-P(k)=(m-k)(a(m^2+mk+k^2)+b)$. Supongamos que existe un par de enteros $(m,k)$ tales que $67\mid a(m^2+mk+k^2)+b$. Sea $(m_1,k_1)$ un par de enteros tales que $m_1\equiv k_1\equiv 0\pmod 30$, $m_1\equiv m\pmod{67}$ y $k_1\equiv k\pmod{67}$, considerando que $2010=30\times 67$. Luego, $2010\mid P(m_1)-P(k_1)$ de donde $2010\mid m_1-k_1$, es decir, $m\equiv k\pmod{67}$. Ahora, sea $(m_2,k_2)$ un par de enteros tales que $m_2\equiv k_2\equiv 0\pmod 30$, $m_2\equiv m\pmod{67}$ y $k_2\equiv -2m\pmod{67}$. Luego, $2010\mid P(m_2)-P(k_2)$ de donde $2010\mid m_2-k_2$, es decir, $67\mid m$ y $67\mid b$. Ahora, sea $(m_3,k_3)$ un par de enteros tales que $m_3\equiv k_3\equiv 0\pmod 30$, $m_3\equiv 1\pmod{67}$ y $k_3\equiv 2^{22}\pmod{67}$. Luego, $2010\mid P(m_3)-P(k_3)$ de donde $2010\mid m_3-k_3$, es decir, $2^{22}\equiv 1\pmod{67}$ lo cual es un absurdo. Por ende, $67\nmid a(m^2+mk+k^2)+b$ para todo par de enteros $(m,k)$, lo cual implica que $(a,b)$ es $n$-bueno para todo $n=67^e$ siendo $e\in\ZZ^+$, es decir, $(a,b)$ es muy bueno.
\end{proof}

\begin{probEG}[ISL 2010/N5\protect\footnote{IMO 2010/3}]
	Determine todas las funciones $f:\NN\to\NN$ tales que $(f(m)+n)(m+f(n))$ es un cuadrado perfecto para todos los enteros positivos $m$ y $n$.
	\forum[aops]{1935854}
	\begin{hint}
		Find all functions $f:\NN\to\NN$ such that the number $(f(m)+n)(m+f(n))$ is a square for all $m,n\in\NN$.
	\end{hint}
\end{probEG}

\begin{proof}
	Se sabe que para todo primo $p$ se cumple lo siguiente: para todo $t\in\ZZ$ existen $k_1,k_2\in\ZZ^+$ suficientemente grandes tales que $t=k_1-k_2$, donde $2\mid\nu_p(k_1),\nu_p(k_2)$. Ahora, sea $p$ un primo y supongamos que $p\mid f(m)-f(n)$ para algunos $m,n\in\ZZ^+$. Si $f(m)-f(n)=pt=pk_1-pk_2$, sea $N=pk_1-f(m)=pk_2-f(n)$ un entero positivo. Como $(f(m)+N)(m+f(N))=pk_1(m+f(N))$ es un cuadrado perfecto y $\nu_p(pk_1)$ es impar, entonces $p\mid m+f(N)$ y análogamente $p\mid n+f(N)$ de donde $p\mid m-n$. Si $f(a)=f(b)$ tenemos que $p\mid f(a)-f(b)$ para todo primo $p$, es decir, $p\mid a-b$ para todo primo $p$ de donde $a=b$. Si existe un primo $p$ que divide a $f(a+1)-f(a)$, entonces $p\mid (a+1)-a=1$ lo cual es un absurdo, así que $f(a+1)-f(a)=\pm 1$. Esto implica que $f(x)=x+c$ donde $c\in\ZZ^+_0$ es fijo.
\end{proof}

\begin{probEG}[ISL 2002/N3]
	Sean $p_1,p_2,\dots,p_n$ números primos distintos y mayores que $3$. Demuestre que el número $2^{p_1p_2\cdots p_n}+1$ tiene al menos $4^n$ divisores positivos.
	\forum[aops]{118690}
	\begin{hint}
		Let $p_1,p_2,\dots,p_n$ be distinct primes greater than $3$. Show that $2^{p_1p_2\cdots p_n}+1$ has at least $4^n$ divisors.
	\end{hint}
\end{probEG}

\begin{proof}[Solución 1]
	Procederemos por inducción sobre $n$. Si $n=1$, los números $1<3<\frac{N}{3}<N$ son los divisores del número $N=2^{p_1}+1$. Ahora, sea $n>1$ y supongamos que $a+1$ tiene al menos $4^{n-1}$ divisores, donde $a=2^{p_1p_2\cdots p_{n-1}}$ y $p_1<p_2<\cdots<p_n$. Note que
	\[\mcd(2^{p_1p_2\cdots p_{n-1}}+1,2^{p_n}+1)=2^{\mcd(p_1p_2\cdots p_{n-1},p_n)}+1=3\]
	así que $\frac{2^{p_n}+1}{3}$ divide a $N=\frac{a^{p_n}+1}{a+1}$. En efecto, sea $N=\frac{2^{p_n}+1}{3}\cdot M$. Note que
	\[\nu_3\left(\frac{2^{p_n}+1}{3}\right)=\nu_3(3)=1\]
	de donde si $p$ es un factor primo de $\frac{2^{p_n}+1}{3}$, entonces $p>3$ y $\ord_p(2)=2p_n\mid p-1$. Luego, $p>p_n$ de donde $p$ es coprimo con $p_1p_2\cdots p_n$. Por ende,
	\[\nu_p\left(\frac{2^{p_n}+1}{3}\right)=\nu_p(2^{p_n}+1)=\nu_p(2^{p_1p_2\cdots p_n}+1)\]
	de donde $p\nmid M$. Note que
	\[N=a^{p_n-1}-a^{p_n-2}+\dots+1\equiv p_n\pmod{a+1}\]
	de donde $\mcd(N,a+1)\mid p_n$. Como $\frac{2^{p_n}+1}{3}>p_n$ y $M>p_n$ son coprimos, existen factores primos $q_1\ne q_2$ de $\frac{2^{p_n}+1}{3}$ y $M$, coprimos con $a+1$. Por ende, $2^{p_1p_2\cdots p_n}+1$ tiene al menos $4^{n-1}\cdot 2\cdot 2=4^n$ divisores.
\end{proof}

\begin{proof}[Solución 2]
	Por Zsigmondy, para cada divisor $d$ de $p_1p_2\cdots p_n$ existe un factor primo primitivo de $2^d+1$ (excepto cuando $d=3$, lo cual no puede ocurrir), así que tenemos al menos $2^n$ factores primos de $2^{p_1p_2\cdots p_n}+1$ y por ende al menos $2^{2^n}\ge 4^n$ divisores.
\end{proof}

\note{Viernes\\2022-06-10}

\begin{probEG}[ISL 2002/N4]
	Determine si existe un entero positivo $m$ para el cual la ecuación
	\[\frac1a+\frac1b+\frac1c+\frac{1}{abc}=\frac{m}{a+b+c}\]
	tiene infinitas soluciones en los enteros positivos $a,b,c$.
	\forum[aops]{118691}
	\begin{hint}
		Is there a positive integer $m$ such that the equation
		\[\frac1a+\frac1b+\frac1c+\frac{1}{abc}=\frac{m}{a+b+c}\]
		has infinitely many solutions in positive integers $a,b,c$?
	\end{hint}
\end{probEG}

\begin{proof}
	Probaremos que $m=12$ cumple, y digamos que la terna $(a,b,c)$ de enteros positivos con $a<b<c$ satisface la ecuación dada y que además $a\mid bc+1$, $b\mid c+a$ y $c\mid ab+1$. Ahora, sea $(a',b',c')=\left(b,c,\frac{bc+1}{a}\right)$ una terna de enteros positivos. Note que la terna $(a',b',c')$ también satisface las condiciones anteriores y la ecuación dada, y que $a'+b'+c'>a+b+c$. Como la terna $(1,2,3)$ cumple, es claro que existen infinitas soluciones para la ecuación dada.
\end{proof}

\begin{probEG}[ISL 2002/N6]
	Determine todas las parejas de enteros positivos $m,n\ge 3$ para los cuales existen infinitos enteros positivos $a$ que satisface
	\[\frac{a^m+a-1}{a^n+a^2-1}\in\ZZ.\]
	\forum[aops]{118695}
	\begin{hint}
		Find all pairs of positive integers $m,n\ge 3$ for which there exist infinitely many positive integers $a$ such that
		\[\frac{a^m+a-1}{a^n+a^2-1}\]
		is itself an integer.
	\end{hint}
\end{probEG}

\begin{proof}
	Sean $Q,R\in\ZZ[x]$ con $\deg R<n$ tales que
	\[x^m+x-1=Q(x)\cdot(x^n+x^2-1)+R(x).\]
	Luego, $a^n+a^2-1\mid R(a)$ para infinitos $a\in\ZZ^+$ de donde $R(x)=0$, es decir,
	\[x^m+x-1=Q(x)\cdot(x^n+x^2-1).\]
	Ahora, sea $\eps\in(0,1)$ tal que $\eps^n+\eps^2=1$. Luego,
	\[a^n+a>a^n+a^2=a^m+a=1>a^{2n}+a\]
	de donde $m<2n$. Si $a=2$, entonces $2^n+3\mid 2^m+1$ de donde $2^n+3\mid 3\cdot 2^{m-n}-1$. Como $0<m-n<n$, se cumple la igualdad, entonces $(m,n)=(5,3)$. Note que
	\[x^5+x-1=(x^3+x^2-1)(x^2-x+1)\]
	así que la única pareja que cumple es $(5,3)$.
\end{proof}

\begin{probMG}[ISL 2003/N5]
	Un entero $n$ es llamado \emph{bueno} si $\abs{n}$ no es un cuadrado perfecto. Determine todos los números enteros $m$ con la siguiente propiedad: $m$ se puede representar de infinitas formas, como la suma de tres números buenos distintos cuyo producto es el cuadrado de un número entero impar.
	\forum[aops]{18558}
	\begin{hint}
		An integer $n$ is said to be \emph{good} if $\abs{n}$ is not the square of an integer. Determine all integers $m$ with the following property: $m$ can be represented, in infinitely many ways, as a sum of three distinct good integers whose product is the square of an odd integer.
	\end{hint}
\end{probMG}

\begin{proof}
	Es claro que esos tres números son de la forma $a^2yz,b^2zx,c^2xy$ donde $a,b,c\in\ZZ^+$ y $x,y,z\in\ZZ$ son impares, es decir,
	\[n=a^2yz+b^2zx+c^2xy\equiv\frac{(x+y+z)^2-x^2-y^2-z^2}{2}\equiv 3\pmod 4.\]
	Ahora, sea $p\equiv 5\pmod 8$ un número primo suficientemente grande y sea
	\[(x,y,z)=\left(-p,p+2n,\frac{1+pn(p+2n)}{2}\right)\]
	una terna de enteros impares coprimos. Como $x\equiv y\equiv 3\pmod 8$, entonces $n$ es la suma de los números buenos distintos $yz,zx,n^2xy$ cuyo producto es un cuadrado perfecto de un número impar. Finalmente, todo entero $n\equiv 3\pmod 4$ cumple.
\end{proof}

\begin{probEG}[ISL 2003/N6\protect\footnote{IMO 2003/6}]
	Sea $p$ un número primo. Demuestre que existe un número primo $q$ tal que para todo número entero $n$, el número $n^p-p$ no es múltiplo de $q$.
	\forum[aops]{266}
	\begin{hint}
		Let $p$ be a prime number. Prove that there exists a prime number $q$ such that for every integer $n$, the number $n^p-p$ is not divisible by $q$.
	\end{hint}
\end{probEG}

\begin{proof}
	Sea $q\not\equiv 1\pmod{p^2}$ un factor primo de
	\[\frac{p^p-1}{p-1}=p^{p-1}+p^{p-2}+\dots+1.\]
	Luego, $\ord_qp\mid\mcd(p,q-1)$ de donde $\ord_qp=p$ y $q=pk+1$ tal que $p\nmid k$. Ahora, si existe un $n\in\ZZ$ tal que $q\mid n^p-p$ entonces claramente $q\nmid n$ y
	\[p^k\equiv n^{pk}=n^{q-1}\equiv 1\pmod q\]
	de donde $p\mid k$ lo cual es un absurdo.
\end{proof}

\begin{probMG}[ISL 2005/N7]
	Sea $P(x)=a_nx^n+a_{n-1}x^{n-1}+\dots+a_0$, donde $a_0,\dots,a_n\in\ZZ$, $a_n>0$ y $n\ge 2$. Demuestre que existe un $m\in\ZZ^+$ para el cual $P(m!)$ es un número compuesto.
	\forum[aops]{789443}
	\begin{hint}
		Let $P(x)=a_nx^n+a_{n-1}x^{n-1}+\dots+a_0$, where $a_0,\dots,a_n$ are integers, $a_n>0$, $n\ge 2$. Prove that there exists a positive integer $m$ such that $P(m!)$ is a composite number.
	\end{hint}
\end{probMG}

\begin{proof}
	Si $\abs{a_0}\ne 1$, existe un primo $q$ tal que $q\mid a_0$. Es decir, $q\mid P(N!)>N!$ para algún $N\in\ZZ^+$ suficientemente grande, así que $P(N!)$ es compuesto. Si $a_0=\pm 1$, sea $m=p_i!+1$ para algún $i\in\ZZ^+$ suficientemente grande, donde $p_i$ es el $i$-ésimo primo. Ahora, sea $P'(x)=a_0x^n+a_1x^{n-1}+\dots+a_n$, luego
	\[P\left(\frac{1}{m!}\right)=\frac{P'(m!)}{(m!)^n}=k\cdot\frac{a_n}{(m!)^n}\]
	donde
	\[k=a_0\cdot\frac{(m!)^n}{a_n}+\dots+a_{n-1}\cdot\frac{m!}{a_n}+1\equiv 1\;\;\left(\bmod\ \frac{m!}{a_n}\right).\]
	Es decir, existe un factor primo $q\ge p_i!+p_{i+1}$ de $P'(m!)$ tal que $N!=(q-p_i!-2)!\ge q$, luego
	\begin{align*}
		N!&=(q-p_i!-2)!\\
		&=\frac{(q-1)!}{(q-p_i!-1)\cdots(q-2)(q-1)}\\
		&\equiv\frac{-1}{(-1)^m\cdot m!}\\
		&=\frac{1}{m!}\pmod q
	\end{align*}
	de donde
	\[P(N!)\equiv P\left(\frac{1}{m!}\right)\equiv 0\pmod q.\]
	Luego, $q\mid P(N!)>N!$ así que $P(N!)$ es compuesto.
\end{proof}

\begin{probEG}[ISL 2009/N4]
	Determine todos los $n\in\ZZ^+$ para los cuales existe una secuencia $a_1,a_2,\dots,a_n$ que satisface
	\[a_{k+1}=\frac{a_k^2+1}{a_{k-1}+1}-1\]
	para todo $2\le k\le n-1$.
	\forum[aops]{1932944}
	\begin{hint}
		Find all positive integers $n$ such that there exists a sequence of positive integers $a_1,a_2,\dots,a_n$ satisfying
		\[a_{k+1}=\frac{a_k^2+1}{a_{k-1}+1}-1\]
		for every $k$ with $2\le k\le n-1$.
	\end{hint}
\end{probEG}

\begin{proof}
	Para $n\le 4$, considere la secuencia $4,33,217,1384$. Ahora, suponga que $n\ge 5$. Note que si $a_{k-1}$ es impar entonces $a_k$ es impar y $a_{k+1}$ es par para todo $2\le k\le n-1$, así que si $a_2$ es impar, entonces $a_3$ es impar y $a_4$ es par, pero $a_4$ es impar ya que $a_3$ es impar, lo cual es un absurdo. Es decir, si $(a,b)=(a_1,a_2)$ tenemos que $2\mid b$ y
	\[(a_3,a_4)=\left(\frac{b^2-a}{a+1},\frac{(b^2-a)^2-b(a+1)^2}{(a+1)^2(b+1)}\right)\]
	de donde $a+1\mid b^2+1$ y $b+1\mid a^2+1$. Si $d=\mcd(a+1,b+1)>1$ entonces $d=2$, pero $2\nmid b+1$ lo cual es un absurdo. Luego,
	\[k=\frac{a^2+b^2}{(a+1)(b+1)}\in\ZZ^+.\]
	Ahora, fijemos $k$ y supongamos que $(a,b)$ es la pareja que cumple la ecuación de arriba y que además la suma $a+b$ es la mínima. Note que si $(a,b)$ cumple entonces $a+1\mid b^2+1$ de donde $a\le b^2$. Si $a=b^2$ entonces de $b+1\mid a^2+1$ tenemos que $b+1\mid 2$, de donde $a=b=1$, luego $k=\half\not\in\ZZ^+$ lo cual es un absurdo. Es decir, $a<b^2$ y la pareja $\left(\frac{b^2-a}{a+1},b\right)$ también cumple, así que
	\[\frac{b^2}{a}>\frac{b^2-a}{a+1}\ge a\]
	de donde $b>a$ y análogamente $a>b$ lo cual es un absurdo.
\end{proof}

\begin{probEB}[ISL 2008/N1]
	Sea $n$ un entero positivo y sea $p$ un número primo. Demuestre que si los números enteros $a$, $b$ y $c$ (no necesariamente positivos) satisfacen las ecuaciones
	\[a^n+pb=b^n+pc=c^n+pa\]
	entonces $a=b=c$.
	\forum[aops]{1555927}
	\begin{hint}
		Let $n$ be a positive integer and let $p$ be a prime number. Prove that if $a,b,c$ are integers (not necessarily positive) satisfying the equations
		\[a^n+pb=b^n+pc=c^n+pa\]
		then $a=b=c$.
	\end{hint}
\end{probEB}

\begin{proof}
	Note que si $a=b$ entonces $a=b=c$. Ahora, supongamos que $a,b,c$ son distintos, luego
	\[a-b\mid b^n-a^n=p(b-c)\]
	y análogamente $b-c\mid p(c-a)$ y $c-a\mid p(a-b)$. Si $q\ne p$ es un primo, Entonces $\nu_q(a-b)\le\nu_q(b-c)$ y análogamente con $\nu_q(b-c)$ y $\nu_q(c-a)$, así que $\nu_q(a-b)=\nu_q(b-c)=\nu_q(c-a)$. Es decir,
	\[(\abs{a-b},\abs{b-c},\abs{c-a})=(M\cdot p^\alpha,M\cdot p^\beta,M\cdot p^\gamma)\]
	para algún entero positivo impar $M$ y $\alpha,\beta,\gamma\in\ZZ^+_0$ donde $\alpha=\max(\alpha,\beta,\gamma)$ sin pérdida de generalidad. Luego, $\pm p^\alpha\pm p^\beta\pm p^\gamma=0$ de donde $p=2$. Entonces, $a,b,c$ tienen la misma paridad, de donde
	\[\frac{a^n-b^n}{a-b}\equiv\frac{b^n-c^n}{b-c}\equiv\frac{c^n-a^n}{c-a}\equiv na^{n-1}\pmod 2.\]
	Si $na^{n-1}$ es impar, tenemos
	\[\nu_2(a-b)<\nu_2(2(a-b))=\nu_2(c^n-a^n)=\nu_2(c-a)\]
	lo cual es un absurdo, así que $na^{n-1}$ es par, de donde
	$a-b\mid\frac{b^n-a^n}{2}=b-c$ y análogamente $b-c\mid c-a$ y $c-a\mid a-b$. Luego, $\abs{a-b}=\abs{b-c}=\abs{c-a}$ y $a=b=c$, lo cual es un absurdo.
\end{proof}

\begin{probEG}[ISL 2008/N2]
	Sean $a_1,a_2,\dots,a_n$ números enteros positivos distintos, con $n\ge 3$. Demuestre que existen índices distintos $i$ y $j$ de tal manera que la suma $a_i+a_j$ no divide a ninguno de los números $3a_1,3a_2,\dots,3a_n$.
	\forum[aops]{1555929}
	\begin{hint}
		Let $a_1,a_2,\dots,a_n$ be distinct positive integers, $n\ge 3$. Prove that there exist distinct indices $i$ and $j$ such that $a_i+a_j$ does not divide any of the numbers $3a_1,3a_2,\dots,3a_n$.
	\end{hint}
\end{probEG}

\begin{probEG}[ISL 2009/N2]
	Un entero positivo $N$ es llamado \emph{balanceado} si $N=1$ o si $N$ puede representarse como el producto de una cantidad par de números primos, no necesariamente distintos. Dados los enteros positivos $a$ y $b$, considere el polinomio $P(x)=(x+a)(x+b)$.
	\begin{enumerate}[(a)]
		\ii Demuestre que existen dos enteros positivos distintos $a$ y $b$ para los cuales cada uno de los números $P(1),P(2),\dots,P(50)$ es balanceado.
		\ii Demuestre que si $P(n)$ es balanceado para todo entero positivo $n$, entonces $a=b$.
	\end{enumerate}
	\forum[aops]{1932941}
	\begin{hint}
		A positive integer $N$ is called \emph{balanced} if $N=1$ or if $N$ can be written as a product of an even number of not necessarily distinct primes. Given positive integers $a$ and $b$, consider the polynomial $P$ defined by $P(x)=(x+a)(x+b)$.
		\begin{enumerate}[(a)]
			\ii Prove that there exist distinct positive integers $a$ and $b$ such that all the number $P(1),P(2),\dots,P(50)$ are balanced.
			\ii Prove that if $P(n)$ is balanced for all positive integers $n$, then $a=b$.
		\end{enumerate}
	\end{hint}
\end{probEG}

\begin{proof}
	Si $f(n)=\sum_{p\text{ es primo}}\nu_p(n)$, entonces $N$ es balanceado si y solo si $f(N)$ es par. Es decir, $P(n)$ es balanceado si y solo si $f(x+a)\equiv f(x+b)\pmod 2$. Ahora, por el principio de las casillas existen enteros positivos $a\ne b$ tales que
	\[(f(a+1),f(a+2),\dots,f(a+50))\equiv(f(b+1),f(b+2),\dots,f(b+50))\pmod 2\]
	de donde $P(1),P(2),\dots,P(50)$ es balanceado. Ahora, si $P(n)$ es balanceado para todo $n\in\ZZ^+$, entonces $f(n+a)\equiv f(n+b)\pmod 2$ para todo $n\in\ZZ^+$. Si $a>b$, entonces por Dirichlet existe un primo $p=(a-b)k+1>b$ para algún $k\in\ZZ^+$, de donde
	\[1=f(p)\equiv f(p^2)\equiv 0\pmod 2\]
	lo cual es un absurdo, así que $a=b$.
\end{proof}

\begin{probMG}[ISL 2005/N2\protect\footnote{IMO 2005/2}]
	Sea $a_1,a_2,\dots$ una secuencia de números enteros con infinitos términos positivos e infinitos términos negativos. Suponga que para cada entero positivo $n$, los números $a_1,a_2,\dots,a_n$ tienen $n$ restos distintos al ser divididos entre $n$. Demuestre que cada número entero aparece exactamente una vez en la secuencia.
	\forum[aops]{281572}
	\begin{hint}
		Let $a_1,a_2,\dots$ be a sequence of integers with infinitely many positive terms and infinitely many negative terms. Suppose that for every positive integer $n$, the numbers $a_1,a_2,\dots,a_n$ leave $n$ different remainders on division by $n$. Prove that every integer occurs exactly once in the sequence.
	\end{hint}
\end{probMG}

\begin{probEG}[ISL 2008/N6\protect\footnote{IMO 2008/3}]
	Demuestre que existen infinitos $n\in\ZZ^+$ para los cuales $n^2+1$ tiene un factor primo mayor que $2n+\sqrt{2n}$.
	\forum[aops]{1190546}
	\begin{hint}
		Prove that there are infinitely many positive integers $n$ such that $n^2+1$ has a prime divisor greater than $2n+\sqrt{2n}$.
	\end{hint}
\end{probEG}

\begin{proof}
	Sea $p\equiv 1\pmod 4$ un primo suficientemente grande. Luego, existe un entero $x=\frac{p-1}{2}-m$ donde $1<m<\frac{p-1}{2}$ tal que $p\mid x^2+1$, de donde $p\mid(2m+1)^2+4$. Luego, $p<(2m+1)^2+2m+1$ de donde $2x=p-2m-1<(2m+1)^2$ y $\sqrt{2x}<2m+1=p-2x$ de donde $p>2x+\sqrt{2x}$. Como $x\ge\sqrt{p-1}$, $n^2+1$ tiene un factor primo mayor que $2n+\sqrt{2n}$ para infinitos $n\in\ZZ^+$.
\end{proof}

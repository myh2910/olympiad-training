\section{Semana 11 (05/23 -- 05/29)}

\note[Álgebra]{Lunes\\2022-05-23}

\begin{probEG}
	Sean $a,b,c\in\RR^+$. Demuestre que
	\[\frac{a}{\sqrt{a^2+8bc}}+\frac{b}{\sqrt{b^2+8ca}}+\frac{c}{\sqrt{c^2+8ab}}\ge 1.\]
\end{probEG}

\begin{proof}
	Como la desigualdad dada es homogenea, supongamos que $a+b+c=1$. Sabemos que la función $f(x)=x^{-1/2}$ es convexa, entonces por Jensen
	\begin{align*}
		\cycsum\frac{a}{\sqrt{a^2+8bc}}
		&=\cycsum a\cdot f(a^2+8bc)\\
		&\ge f\left(\cycsum a(a^2+8bc)\right)\\
		&=f(a^3+b^3+c^3+24abc)\\
		&\ge f(a^3+b^3+c^3+3(a+b)(b+c)(c+a))\\
		&=f(1)\\
		&=1.
	\end{align*}
\end{proof}

\begin{probEG}
	Sea $n\ge 3$ un número entero y sean $t_1,t_2,\dots,t_n$ números reales positivos tales que
	\[n^2+1>(t_1+t_2+\dots+t_n)\left(\frac{1}{t_1}+\frac{1}{t_2}+\dots+\frac{1}{t_n}\right).\]
	Demuestre que para todo $i,j,k$, con $1\le i<j<k\le n$, los números $t_i,t_j,t_k$ son las longitudes de los lados de un triángulo.
\end{probEG}

\begin{probEG}
	La secuencia $(a_n)_{n\ge 0}$ está definida por
	\[a_n=\abs{a_{n+1}-a_{n+2}},\quad\forall\,n\ge 0,\]
	donde $a_0,a_1\in\RR^+$, con $a_0\ne a_1$. ¿Puede que la secuencia $(a_n)$ sea acotada?
\end{probEG}

\begin{proof}
	La respuesta es no. Considere tres casos: $a_0>a_1$, $a_0=a_1$ y $a_0<a_1$.
\end{proof}

\begin{probMR}
	Determine todos los polinomios $P(x)$ con coeficientes reales tales que
	\[P(a-b)+P(b-c)+P(c-a)=2P(a+b+c)\]
	para todo $a,b,c\in\RR$, con $ab+bc+ca=0$.
\end{probMR}

\note[Punto\\Dumpty \cite{ref:dumpty}]{Martes\\2022-05-24}

\begin{probEG}[USAMO 2008/2]
	Let $ABC$ be an acute, scalene triangle, and let $M$, $N$, and $P$ be the midpoints of $\ol{BC}$, $\ol{CA}$, and $\ol{AB}$, respectively. Let the perpendicular bisectors of $\ol{AB}$ and $\ol{AC}$ intersect ray $AM$ in points $D$ and $E$ respectively, and let lines $BD$ and $CE$ intersect in point $F$, inside of triangle $ABC$. Prove that points $A$, $N$, $F$, and $P$ all lie on one circle.
	\forum[aops]{1116181}
\end{probEG}

\note{Miércoles\\2022-05-25}

\begin{probMG}[Indian TST Practice Test 2019/2]
	Let $ABC$ be a triangle with $\angle A=\angle C=30\dg$. Points $D,E,F$ are chosen on the sides $AB,BC,CA$ respectively so that $\angle BFD=\angle BFE=60\dg$. Let $p$ and $p_1$ be the perimeters of the triangles $ABC$ and $DEF$, respectively. Prove that $p\le 2p_1$.
	\forum[aops]{12753024}
\end{probMG}

\note[IMO TST Día 1]{Jueves\\2022-05-26}

\begin{probEG}[Peru IMO TST 2022/1]
	Encuentre todos los enteros positivos $n$ tales que para cualquier entero positivo $k$ con $1\le k\le\sqrt{n}$ se cumple que el número $\floor{n}{k}-k$ es impar.
\end{probEG}

\begin{proof}
	Respuesta: $n=2,6,14,30$.
\end{proof}

\begin{probEG}[Peru IMO TST 2022/2]
	Determine todas las funciones $f:\RR\to\RR$ tales que
	\[f(xy+f(x))+f(y)=xf(y)+f(x+y)\]
	para todo par de números reales $x,y$.
\end{probEG}

\begin{proof}
	Respuesta: $f(x)\equiv 2-x,x,0$.
\end{proof}

\begin{probEG}[Peru IMO TST 2022/3]
	Considere un $n$-ágono (un polígono de $n$ lados) con $n\ge 3$. Se asignan números enteros no negativos distintos dos a dos a cada lado y diagonal del $n$-ágono de modo que los números asignados a los lados de cualquier triángulo con vértices en los vértices del $n$-ágono forman una progresión aritmética. Determine el máximo valor de $n$ para el cual esto es posible.
\end{probEG}

\begin{proof}
	Respuesta: $n=4$.
\end{proof}

\begin{probMB}[Peru IMO TST 2022/4]
	Sea $\Omega$ el circuncírculo del triángulo $ABC$, con $\angle BAC>90\dg$ y $AB>AC$. Las rectas tangentes a $\Omega$ en los puntos $B$ y $C$ se intersectan en $D$. La recta tangente a $\Omega$ en el punto $A$ interseca a la recta $BC$ en $E$. La recta paralela a $AE$ que pasa por $D$ interseca a la recta $BC$ en $F$. La circunferencia $\Gamma$ con diámetro $EF$ interseca a la recta $AB$ en los puntos $P$ y $Q$, y a la recta $AC$ en los puntos $X$ e $Y$. Pruebe que uno de los ángulos $\angle AEB$, $\angle PEQ$, $\angle XEY$ es igual a la suma de los otros dos.
\end{probMB}

\begin{proof}
	Sea $Y'=A'B\cap AC$ donde $A'$ es la reflexión de $A$ con respecto a $BC$. Sea $Q'$ la reflexión de $Y'$ con respecto a $BC$. Luego,
	\[\angle EBY'=\angle ABC=\angle EAC=\angle EAY'\]
	de donde $EABY'$ es cíclico. Si $R=AB\cap DF$, entonces $\angle RFB=\angle AEB=\angle CY'B$ de donde
	\[\triangle RFB\cong\triangle AEB\cong\triangle CY'B.\]
	Luego, $B$ es el centro de roto-homotecia que manda $(A,B,C,R)\mapsto(E,B,Y',F)$. Es decir, $\angle EY'F=\angle ACR$. Si $S$ es la intersección de la mediatriz del segmento $BC$ y la recta $AB$, entonces
	\[\angle DRS=180\dg-\angle BRF=180\dg-\angle BCY'=\angle ACB=\angle SBD=\angle DCS\]
	de donde $DRCS$ es cíclico y
	\begin{align*}
		\angle EY'F
		&=\angle ACR\\
		&=\angle CAB-\angle CRB\\
		&=(180\dg-\angle BCD)-\angle CRS\\
		&=180\dg-(90\dg-\angle CDS)-\angle CDS\\
		&=90\dg.
	\end{align*}
	Por lo tanto, $Y'$ y $Q'$ pertenecen a la circunferencia de diámetro $EF$. Sin pérdida de generalidad, $Y'=Y$ y $Q'=Q$. Sabemos que el ángulo que forman las rectas $PX$ y $QY$ es la suma o la diferencia de las medidas de los arcos $PQ$ y $XY$, es decir, la suma o la diferencia de $\angle PEQ$ y $\angle XEY$. Note que $\angle APX=\angle QYA=90\dg-\angle ACB$ y $\angle AQY=90\dg-\angle CBA$, de donde
	\[\angle AQY-\angle APX=\angle ACB-\angle CBA=\angle ACB-\angle CAE=\angle AEB\]
	es la suma o la diferencia de $\angle PEQ$ y $\angle XEY$.
\end{proof}

\note{Sábado\\2022-05-28}

\begin{probMG}
	Demuestre que para cualesquiera enteros positivos $m$ y $n$ existen infinitas parejas de enteros positivos $(a,b)$, coprimos, tales que
	\[a+b\mid a\cdot m^a+b\cdot n^b.\]
\end{probMG}

\begin{proof}
	Pista: considerar $a+b=p$ donde $p$ es un número primo apropiado.
\end{proof}

\note[Simulacro Nivel 2]{}

\begin{probEG}
	Sean $AK$ y $BL$ las alturas de un triángulo $ABC$ trazadas desde los vértices $A$ y $B$, respectivamente. El punto $P$ tomado del segmento $AK$ es tal que $LK=LP$. La paralela a $BC$ que pasa por $P$ intersecta en el punto $Q$ a la paralela a $PL$ que pasa por $B$. Demuestre que $\angle AQB=\angle ACB$.
\end{probEG}

\begin{proof}
	Sea $M$ el punto medio de $QB$. Como $QP\parallel BC\perp AK$, entonces $ML$ es la mediatriz de $PK$. Como $MB\parallel PL$, tenemos
	\[\angle BML=\angle PLM=\angle MLK=\angle A=\angle BAL\]
	de donde $BMAL$ es cíclico y $\angle AMB=\angle ALB=90\dg$. Por ende,
	\[\angle AQB=\angle ABM=\angle ALM=\angle ACB.\]
\end{proof}

\begin{probEB}
	Sea $n>1$ un número entero. Se marcan $k$ casillas en un tablero de $n\times n$. Determine el mayor valor posible de $k$ para el cual es posible permutar filas y columnas de tal manera que las $k$ casillas marcadas estén en la diagonal principal o por encima de ella.
\end{probEB}

\begin{proof}
	Respuesta: $n+1$ por inducción. El siguiente es un contraejemplo de $n+2$.
	\[\begin{tblr}{row{3}={28pt},column{3}={28pt}}
		\clubsuit&         &      &         &\clubsuit\\
		         &\clubsuit&      &         &         \\
		         &         &\ddots&         &         \\
		         &         &      &\clubsuit&         \\
		\clubsuit&         &      &         &\clubsuit
	\end{tblr}\]
\end{proof}

\begin{probMR}
	Sean $a_1$, $a_2$, $a_3$, $b_1$, $b_2$ y $b_3$ enteros positivos distintos dos a dos tales que
	\[(n+1)a_1^n+na_2^n+(n-1)a_3^n\mid(n+1)b_1^n+nb_2^n+(n-1)b_3^n\]
	para todo entero positivo $n$. Demuestre que existe un número entero positivo $k$ tal que $b_i=k\cdot a_i$ para $i=1,2,3$.
\end{probMR}

\note[IMO TST Día 2]{Domingo\\2022-05-29}

\begin{probEG}[Peru IMO TST 2022/5]
	Sea $N$ un entero positivo. Determine todos los enteros positivos $n$ que satisfacen la siguiente condición: para cualquier lista $d_1,d_2,\dots,d_k$ de divisotres de $n$ (no necesariamente distintos) tal que
	\[\frac{1}{d_1}+\frac{1}{d_2}+\dots+\frac{1}{d_k}>N,\]
	algunas de las fracciones $\frac{1}{d_1},\frac{1}{d_2},\dots,\frac{1}{d_k}$ suman exactamente $N$.
\end{probEG}

\begin{proof}
	Respuesta: $n=p^e$ para todo número primo $p$ y $e\in\ZZ^+_0$.
\end{proof}

\begin{probHR}[Peru IMO TST 2022/6]
	Sea $n\ge 2$ un entero y sean $a_1,a_2,\dots,a_n$ números reales positivos cuya suma es $1$. Pruebe que
	\[\sum_{k=1}^n\frac{a_k}{1-a_k}(a_1+a_2+\dots+a_{k-1})^2<\frac13.\]
\end{probHR}

\begin{probMB}[Peru IMO TST 2022/7]
	Sea $m>1$ un entero. Considere un tablero de tamaño $3m\times 3m$. Una rana se ubica en la casilla de la esquina inferior izquierda $S$ del tablero y quiere llegar mediante saltos a la casilla de la esquina superior derecha $F$ del tablero. La rana solo puede saltar a la casilla vecina derecha o a la casilla vecina de arriba de la casilla en la que se encuentra.

	Algunas casillas están pegajosas y la rana queda atrapada si llega a cualquiera de ellas. El conjunto $X$ de casillas pegajosas se denomina \emph{barricada} si la rana no puede llegar a $F$ partiendo de $S$. Una barricada es \emph{minimal} si no contiene una barricada más pequeña.
	\begin{enumerate}[a)]
		\ii Pruebe que existe una barricada minimal conteniendo al menos $3m^2-3m$ casillas.
		\ii Pruebe que cualquier barricada minimal contiene a lo más $3m^2$ casillas.
	\end{enumerate}
	\emph{Nota:} En la figura se muestra un ejemplo de una barricada minimal $X$ para $m=2$. Las casillas en $X$ han sido marcadas con $\clubsuit$.
	\[\begin{tblr}{}
		         &         &         &         &         &F\\
		\clubsuit&\clubsuit&         &         &         & \\
		         &         &\clubsuit&         &         & \\
		         &         &         &\clubsuit&         & \\
		         &         &         &         &\clubsuit& \\
		S        &         &\clubsuit&         &         &
	\end{tblr}\]
\end{probMB}

\begin{proof}
	Digamos que $X$ es el conjunto de todas las casillas marcadas en el siguiente tablero de $3m\times 3m$.
	\[\begin{tblr}{row{5}={28pt},column{7}={28pt}}
		         &         &         &         &         &         &      &         &         &F        \\
		\clubsuit&         &         &\clubsuit&         &         &      &\clubsuit&         &         \\
		         &\clubsuit&         &         &\clubsuit&         &\cdots&         &\clubsuit&         \\
		         &\clubsuit&         &         &\clubsuit&         &      &         &\clubsuit&         \\
		         &\vdots   &         &         &\vdots   &         &\ddots&         &\vdots   &         \\
		         &\clubsuit&         &         &\clubsuit&         &      &         &\clubsuit&         \\
		         &\clubsuit&         &         &\clubsuit&         &\cdots&         &\clubsuit&         \\
		         &         &\clubsuit&         &         &\clubsuit&      &         &         &\clubsuit\\
		S        &         &         &         &         &         &      &         &         &
	\end{tblr}\]
	Es fácil ver que $X$ es una barricada minimal con $\abs{X}=3m^2-2m>3m^2-3m$. Ahora, para probar que una barricada minimal $X$ contiene a lo más $3m^2$ casillas, asignaremos dos casillas vecinas a cada casilla pegajosa, y de esta manera $3\abs{X}\le 3m\times 3m$ de donde $\abs{X}\le 3m^2$.
\end{proof}

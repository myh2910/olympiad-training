\usepackage[sexy,nothm,hints]{evan}
\usepackage[english,spanish,es-nosectiondot,es-lcroman]{babel}
\addto\captionsspanish{\renewcommand{\proofname}{Solución}}
\usepackage{tabularray}
\SetTblrInner{%
  hlines,%
  vlines,%
  rows={16pt,rowsep=0pt},%
  columns={16pt,colsep=0pt},%
  cells={c,m}%
}
\reversemarginpar

\DeclareMathOperator{\mcd}{mcd}
\DeclareMathOperator{\mcm}{mcm}

%%fakesection New commands
\usepackage{ifthen}
\newcommand{\notecolor}{green!35}
\newcommand{\notestyle}{\fontfamily{qpl}\selectfont}
\newcommand{\note}[2][]{%
  \ifthenelse{\equal{#1}{}}%
    {\todo[color=\notecolor]{\notestyle#2}}%
    {\ifthenelse{\equal{#2}{}}%
      {\todo[color=\notecolor]{\notestyle\bfseries\color{blue}#1}}%
      {\todo[color=\notecolor]{\notestyle#2\\[-6pt]\rule{\textwidth}{0.3pt}\\\bfseries\color{blue}#1}}%
    }%
}
\newcommand{\forum}[2][]{%
  \par\vspace{1pt}\hspace{\fill}%
  \ifthenelse{\equal{#1}{aops}}%
    {\href{https://artofproblemsolving.com/community/p#2}{\ttfamily[aops:#2]}}%
    {\ifthenelse{\equal{#1}{oma}}%
      {\href{https://omaforos.com.ar/viewtopic.php?t=#2}{\ttfamily[omaforos:#2]}}%
      {\url{#2}}%
    }%
}

\usepackage{bold-extra}
\newcommand{\cgreen}[1]{\bgroup\ttfamily\color{green!40!black}#1\egroup}
\newcommand{\cblue}[1]{\bgroup\ttfamily\color{blue!70!black}#1\egroup}
\newcommand{\cred}[1]{\bgroup\ttfamily\color{red!80!black}#1\egroup}

%%fakesection Theorems
\theoremstyle{definition}
\declaretheorem[style=thmbluebox,name=Teorema,numberwithin=section]{theorem}
\declaretheorem[style=thmbluebox,name=Proposición,sibling=theorem]{proposition}
\declaretheorem[name=Definición,sibling=theorem]{definition}
\declaretheorem[name=Problema,sibling=theorem]{problem}

\newtheorem{probEG}[problem]{\cgreen{[E]} Problema}
\newtheorem{probMG}[problem]{\cgreen{[M]} Problema}
\newtheorem{probHG}[problem]{\cgreen{[H]} Problema}
\newtheorem{probEB}[problem]{\cblue{[E]} Problema}
\newtheorem{probMB}[problem]{\cblue{[M]} Problema}
\newtheorem{probHB}[problem]{\cblue{[H]} Problema}
\newtheorem{probER}[problem]{\cred{[E]} Problema}
\newtheorem{probMR}[problem]{\cred{[M]} Problema}
\newtheorem{probHR}[problem]{\cred{[H]} Problema}
\newtheorem{probE}[problem]{\texttt{[E]} Problema}
\newtheorem{probM}[problem]{\texttt{[M]} Problema}
\newtheorem{probH}[problem]{\texttt{[H]} Problema}

%%fakesection Bibliography
\usepackage[backend=biber,style=alphabetic]{biblatex}
\usepackage{csquotes}
\DeclareLabelalphaTemplate{%
  \labelelement{%
    \field[final]{shorthand}%
    \field{label}%
    \field[strwidth=2,strside=left]{labelname}%
  }
  \labelelement{%
    \field[strwidth=2,strside=right]{year}%
  }
}
\DeclareFieldFormat{labelalpha}{\textbf{\scriptsize #1}}
\addbibresource{references.bib}

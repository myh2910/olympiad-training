\section{Semana 4 (04/04 -- 04/10)}

\note[Entrenamien-to EGMO]{Lunes\\2022-04-04}

\begin{probMR}[Balkan MO 2005/4]
	Sea $n\ge 2$ un entero. Sea $S$ un subconjunto de $\{1,2,\dots,n\}$ tal que $S$ no contiene dos elementos tal que uno divida al otro, ni tampoco contiene dos elementos que sean coprimos. Determine el mayor número posible de elementos de $S$.
	\begin{hint}
		Let $n\ge 2$ be an integer. Let $S$ be a subset of $\{1,2,\dots,n\}$ such that $S$ neither contains two elements one of which divides the other, nor contains two elements which are coprime. What is the maximal possible number of elements of such a set $S$?
	\end{hint}
	\forum[aops]{225001}
\end{probMR}

\begin{proof}
	La respuesta es $\floor{\frac{n+2}{4}}$.
\end{proof}

\begin{probMB}[Croatian MO 2018/5]
	Sea $n$ un entero positivo y $\SC$ una circunferencia. En la región interior de $\SC$ se ubican los puntos $A_1,A_2,\dots,A_n$ y en $\SC$ se ubican los puntos $B_1,B_2,\dots,B_n$, de modo que $A_1B_1,A_2B_2,\dots,A_nB_n$ son disjuntos dos a dos. Un saltamontes puede saltar del vértice $A_i$ al vértice $A_j$, donde $i\ne j$, si $A_iA_j$ no interseca al interior de ninguno de los segmentos $A_1B_1,A_2B_2,\dots,A_nB_n$.

	Demuestre que el saltamontes puede trasladarse de un vértice $A_i$ a cualquier otro vértice $A_j$ en un número finito de saltos.
	\begin{hint}
		Let $n$ be a positive integer. $A_1$, $A_2$, $\dots$, $A_n$ are points inside a circle and $B_1$, $B_2$, $\dots$, $B_n$ points on that circle such that the segments $A_iB_i$ are pairwise disjoint for $1\le i\le n$. A grasshopper can move from point $A_i$ to $A_j$ where $i\ne j$ if and only if segment $A_iA_j$ doesn't pass through any of segments $A_kB_k$ for $1\le k\le n$.

		Prove that the grasshopper can move from any point $A_i$ to $A_j$ in a finite sequence of moves.
	\end{hint}
	\forum[aops]{12394250}
\end{probMB}

\note[Entrenamien-to EGMO]{Martes\\2022-04-05}

\begin{probEG}[Spain \cite{ref:pedro}]
	Demostrar que $x$ es racional si y solo si la sucesión
	\[x,\,x+1,\,x+2,\,\dots\]
	contiene al menos tres términos en progresión geométrica.
\end{probEG}

\begin{proof}
	Supongamos que $x$ es un número racional. Si $x$ no es positivo, vamos a sumarle $1$ hasta que sea positivo. Sea $x=\frac pq$ donde $p,q\in\ZZ^+$. Luego, los números $x$, $x+p$ y $x+p(q+2)$ están en una progresión geométrica. Ahora, supongamos que existen tres enteros $i,j,k\ge 0$ tales que $x+i$, $x+j$ y $x+k$ están en una progresión geométrica. En efecto, $(x+i)(x+k)=(x+j)^2$ de donde $x(i+k-2j)=j^2-ik$. Si $i+k=2j$, tenemos que $(i-k)^2=(i+k)^2-4ik=4(j^2-ik)=0$ de donde $i=k$, lo cual es un absurdo. Por ende, $x$ es racional.
\end{proof}

\begin{probEG}[CGMO 2008/8]
	Sea $f_n=\floor{2^n\sqrt{2008}}+\floor{2^n\sqrt{2009}}$ para todo $n\in\ZZ^+$. Demuestre que la secuencia $(f_n)_{n\ge 1}$ contiene infinitos pares y también infinitos impares.
	\begin{hint}
		Let $f_n=\floor{2^n\sqrt{2008}}+\floor{2^n\sqrt{2009}}$ for all $n\in\ZZ^+$. Prove there are infinitely many odd numbers and infinitely many even numbers in the sequence $f_1,f_2,\dots$.
	\end{hint}
	\forum[aops]{1236876}
\end{probEG}

\begin{proof}
	Sea $a_n=1$ si $\left\{2^n\sqrt{2008}\right\}>\half$ y $0$ de lo contrario. Análogamente se define $b_n$ con respecto a $2009$. Si solamente hay finitos números de alguna paridad, tenemos que $f_{n+1}-2f_n=a_n+b_n$ tiene la misma paridad para $n$ suficientemente grande. Luego, $a_n=b_n$ o $a_n+b_n=1$ para todo $n\ge N$ donde $N\in\ZZ^+$. Por ende, $2^N\sqrt{2008}\pm2^N\sqrt{2009}$ es racional, lo cual es un absurdo. Finalmente, existen infinitos números pares e impares en la secuencia.
\end{proof}

\note{Miércoles\\2022-04-06}

\begin{probEB}[ISL 2003/A1]
	Let $a_{ij}$ ($i=1,2,3$, $j=1,2,3$) be real numbers such that $a_{ij}$ is positive for $i=j$ and negative for $i\ne j$.

	Prove the existence of positive real numbers $c_{1}$, $c_{2}$, $c_{3}$ such that the numbers
	\[a_{11}c_{1}+a_{12}c_{2}+a_{13}c_{3},\quad a_{21}c_{1}+a_{22}c_{2}+a_{23}c_{3},\quad a_{31}c_{1}+a_{32}c_{2}+a_{33}c_{3}\]
	are either all negative, all positive, or all zero.
\end{probEB}

\begin{probEG}
	Si $a^2<b^2$ son dos números de $1001$ dígitos, demuestre que existe un número capicúa en el intervalo $(a^2,b^2)$.
\end{probEG}

\begin{proof}
	Es claro que entre $a^2+1,a^2+2,\dots,a^2+10^{500}$ existe un múltiplo de $10^{500}$ y sea $\overline{a_{500}a_{499}\dots a_1a_0}\cdot 10^{500}$ dicho número. Luego,
	\[a^2<\overline{a_{500}a_{499}\dots a_1a_0a_1\dots a_{499}a_{500}}\le a^2+2\cdot 10^{500}<(a+1)^2\le b^2\]
	y aquí se termina la prueba.
\end{proof}

\begin{probEG}[IMO 2014/1]
	Let $a_0<a_1<a_2<\cdots$ be an infinite sequence of positive integers. Prove that there exists a unique integer $n\ge 1$ such that
	\[a_n<\frac{a_0+a_1+a_2+\dots+a_n}{n}\le a_{n+1}.\]
	\forum[aops]{3542095}
\end{probEG}

\begin{probEG}
	Sean $a_1,a_2,\dots,a_n$, $k$ y $M$ enteros positivos tales que
	\[\frac{1}{a_1}+\frac{1}{a_2}+\dots+\frac{1}{a_n}=k\quad\text{y}\quad M=a_1a_2\cdots a_n.\]
	Si $M>1$, demuestre que el polinomio
	\[P(x)=M(x+1)^k-(x+a_1)(x+a_2)\cdots(x+a_n)\]
	no tiene raíces positivos.
\end{probEG}

\begin{proof}
	Si $r>0$ es una raíz de $P(x)$, tenemos que
	\[\prod_{i=1}^n\frac{a_i(r+1)^\frac{1}{a_i}}{r+a_i}=1\]
	pero note que
	\[a^a(r+1)=\binom{a}{1}ra^{a-1}+a^a\le(r+a)^a\]
	de donde
	\[\frac{a(r+1)^\frac{1}{a}}{r+a}\le 1\]
	para todo $a\in\ZZ^+$. Es decir, $a_i=1$ para todo $1\le i\le n$ de donde $M=1$ lo cual es un absurdo.
\end{proof}

\begin{probEG}
	Determine todas las parejas $(x,y)$ de enteros positivos tales que
	\[\sqrt[3]{7x^2-13xy+7y^2}=\abs{x-y}+1.\]
\end{probEG}

\begin{proof}
	Sin pérdida de generalidad, supongamos que $x\ge y$. No es difícil ver que
	\[(x-y-2)^2(4x-4y+1)=(x+y)^2\]
	de donde $x-y=k^2+k$ para algún $k\in\ZZ^+_0$. Si $k=0$, tenemos que $x-y=0$ y $x+y=2$ de donde $x=y=1$. Si $k\ge 1$, note que
	\[x+y=(k^2+k-2)(2k+1)=2k^3+3k^2-3k-2\]
	de donde $(x,y)=(k^3+2k^2-k-1,k^3+k^2-2k-1)$ para algún $k\ge 2$.
\end{proof}

\begin{problem}
	Para cada entero positivo $n$, el Banco de Ciudad del Cabo produce monedas de valor $\frac1n$. Dada una colección finita de tales monedas (no necesariamente de distintos valores) cuyo valor total no supera $99+\frac12$, demostrar que es posible separar esta colección en $100$ o menos montones, de modo que el valor total de cada montón sea como máximo $1$.
\end{problem}

\note[EGMO 2022 Día 1]{Viernes\\2022-04-08}

\begin{probEG}[EGMO 2022/1]
	Sea $ABC$ un triángulo acutángulo con $BC<AB$ y $BC<AC$. Considere los puntos $P$ y $Q$ en los segmentos $AB$ y $AC$, respectivamente, tales que $P\ne B$, $Q\ne C$ y $BQ=BC=CP$. Sea $T$ el circuncentro del triángulo $APQ$, $H$ el ortocentro del triángulo $ABC$ y $S$ el punto de intersección de las rectas $BQ$ y $CP$. Pruebe que los puntos $T$, $H$ y $S$ están en una misma recta.
	\begin{hint}
		Let $ABC$ be an acute-angled triangle in which $BC<AB$ and $BC<CA$. Let point $P$ lie on segment $AB$ and point $Q$ lie on segment $AC$ such that $P\ne B$, $Q\ne C$ and $BQ=BC=CP$. Let $T$ be the circumcentre of triangle $APQ$, $H$ the orthocentre of triangle $ABC$, and $S$ the point of intersection of the lines $BQ$ and $CP$. Prove that $T$, $H$ and $S$ are collinear.
	\end{hint}
\end{probEG}

\begin{proof}
	Como $BH$ y $CH$ son las bisectrices del triángulo $SBC$, la recta $HS$ biseca al ángulo $\angle BSC$. Note que
	\[\angle PSQ=\angle BSC=2\angle BHC-180\dg=2(180\dg-\angle BAC)-180\dg=180\dg-\angle PTQ\]
	de donde $TPSQ$ es cíclico. Como $T$ es el punto medio del arco $PQ$ que no contiene a $S$, la recta $TS$ biseca al ángulo $\angle PSQ$. Por ende, $T$, $H$ y $S$ pertenecen a la bisectriz del ángulo $\angle BSC$.
\end{proof}

\begin{probMG}[EGMO 2022/2]
	Determine todas las funciones $f:\NN\to\NN$ tales que para cualquier pareja de enteros positivos $a$ y $b$, se cumplen las siguientes dos condiciones:
	\begin{enumerate}[(1)]
		\ii $f(ab)=f(a)f(b)$, y
		\ii al menos dos de los números $f(a)$, $f(b)$ y $f(a+b)$ son iguales.
	\end{enumerate}
	\begin{hint}
		Find all functions $f:\NN\to\NN$ such that for any positive integers $a$ and $b$, the following two conditions hold:
		\begin{enumerate}[(1)]
			\ii $f(ab)=f(a)f(b)$, and
			\ii at least two of the numbers $f(a)$, $f(b)$ and $f(a+b)$ are equal.
		\end{enumerate}
	\end{hint}
\end{probMG}

\begin{proof}
	Si $p\in\ZZ^+$ es el menor tal que $f(p)>1$, es claro que $p$ es primo. Si $a>p$ es el menor número coprimo con $p$ tal que $f(a)>1$, considerando a $f(p)>1$, $f(a-p)=1$ y $f(a)>1$ tenemos que $f(a)=f(p)$. Si $a>p^2$, considerando a $f(p^2)=f(p)^2>1$, $f(a-p^2)=1$ y $f(a)>1$ tenemos que $f(p)=f(a)=f(p)^2$ lo cual es un absurdo. Ahora, sea $k=\floor{p^2/a}$ donde $0<k<p$. Como $0<p^2-ka<a$, considerando a $f(ka)=f(p)$, $f(p^2-ka)=1$ y $f(p^2)=f(p)^2$ tenemos un absurdo. Por ende, $f(a)=1$ para todo $a\in\ZZ^+$ coprimo con $p$ de donde $f(n)=a^{\nu_p(n)}$ para algún $a\in\ZZ^+$. Por lo tanto, $f(n)=1$ o $f(n)=a^{\nu_p(n)}$ para algún entero $a>1$ y un primo $p$.
\end{proof}

\begin{probEG}[EGMO 2022/3]
	Se dice que una sucesión infinita de enteros positivos $a_1,a_2,\dots$ es \emph{húngara} si
	\begin{enumerate}[(1)]
		\ii $a_1$ es un cuadrado perfecto, y
		\ii para todo entero $n\ge 2$, $a_n$ es el menor entero positivo tal que
		\[na_1+(n-1)a_2+\dots+2a_{n-1}+a_n\]
		es un cuadrado perfecto.
	\end{enumerate}
	Pruebe que si $a_1,a_2,\dots$ es una sucesión húngara, entonces existe un entero positivo $k$ tal que $a_n=a_k$ para todo entero $n\ge k$.
	\begin{hint}
		An infinite sequence of positive integers $a_1,a_2,\dots$ is called \emph{good} if
		\begin{enumerate}[(1)]
			\ii $a_1$ is a perfect square, and
			\ii for any integer $n\ge 2$, $a_n$ is the smallest positive integer such that
			\[na_1+(n-1)a_2+\dots+2a_{n-1}+a_n\]
			is a perfect square.
		\end{enumerate}
		Prove that for any good sequence $a_1,a_2,\dots$, there exists a positive integer $k$ such that $a_n=a_k$ for all integers $n\ge k$.
	\end{hint}
\end{probEG}

\begin{proof}
	Sea $c_n^2=na_1+(n-1)a_2+\dots+2a_{n-1}+a_n$. Note que $c_n-c_{n-1}$ es decreciente y $c_n$ es estrictamente creciente, de donde existe un $k\in\ZZ^+$ tal que $(c_n-c_{n-1}-1)^2\le c_n$ para todo $n\ge k$. Es decir,
	\[(2c_n-c_{n-1}-1)^2-c_n^2\le c_n^2-c_{n-1}^2<(2c_n-c_{n-1})^2-c_n^2\]
	de donde $c_{n+1}-c_n=c_n-c_{n-1}=d$ donde $d\in\ZZ^+$ es fijo. Es decir,
	\[a_{n+1}=(c_{n+1}^2-c_n^2)-(c_n^2-c_{n-1}^2)=2d^2\]
	es fijo para todo $n\ge k$.
\end{proof}

\note[Miscelánea]{}

\begin{probEG}
	Determine todos los polinomios $f(x)$ y $g(x)$ con coeficientes enteros tales que
	\[f(g(x))=x^{2015}+2x+1\]
	para todo $x\in\RR$.
\end{probEG}

\begin{probEG}
	Determine si existe una secuencia $(a_i)_{i\ge 1}$ de enteros positivos tales que cada entero positivo aparece exactamente una vez en las secuencias $(a_i)_{i\ge 1}$ y $(\abs{a_i-a_{i-1}})_{i\ge 1}$.
\end{probEG}

\note[EGMO 2022 Día 2]{Sábado\\2022-04-09}

\begin{probEG}[EGMO 2022/4]
	Para cada entero positivo $n\ge 2$, determine el mayor entero positivo $N$ con la propiedad de que existen $N+1$ números reales $a_0,a_1,\dots,a_N$ tales que
	\begin{enumerate}[(1)]
		\ii $a_0+a_1=-\frac1n$, y
		\ii $(a_k+a_{k-1})(a_k+a_{k+1})=a_{k-1}-a_{k+1}$ para todo $1\le k\le N-1$.
	\end{enumerate}
	\begin{hint}
		Given a positive integer $n\ge 2$, determine the largest positive integer $N$ for which there exist $N+1$ real numbers $a_0,a_1,\dots,a_N$ such that
		\begin{enumerate}[(1)]
			\ii $a_0+a_1=-\frac1n$, and
			\ii $(a_k+a_{k-1})(a_k+a_{k+1})=a_{k-1}-a_{k+1}$ for $1\le k\le N-1$.
		\end{enumerate}
	\end{hint}
\end{probEG}

\begin{proof}
	Sea $S_i=a_i+a_{i+1}$ para todo $0\le i\le N-1$. Luego, $(1-S_k)(1+S_{k-1})=1$ para todo $1\le k\le N-1$ de donde si $S_{k-1}=-\frac{1}{n-k+1}$ entonces $S_k=-\frac{1}{n-k}$. Es fácil ver que el mayor valor de $N$ es $n$.
\end{proof}

\begin{probEG}[EGMO 2022/5]
	Dados $n$ y $k$ enteros positivos, sea $f(n,2k)$ el número de formas en que un tablero de tamaño $n\times 2k$ puede ser completamente cubierto por $nk$ fichas de dominó de tamaño $2\times 1$. Encuentre todos los enteros positivos $n$ tales que para todo entero positivo $k$, el número $f(n,2k)$ es impar.
	\begin{hint}
		For all positive integers $n,k$, let $f(n,2k)$ be the number of ways an $n\times 2k$ board can be fully covered by $nk$ dominoes of size $2\times 1$. (For example, $f(2,2)=2$ and $f(3,2)=3$.) Find all positive integers $n$ such that for every positive integer $k$, the number $f(n,2k)$ is odd.
	\end{hint}
\end{probEG}

\begin{proof}
	Si $n$ es impar, por simetría con respecto a la fila central, podemos notar que $f(n,2k)\equiv f(\frac{n-1}{2},2k)\pmod 2$. Si $n$ es par, por simetría con respecto a la diagonal principal, podemos notar que $f(n,n)$ es par. Ahora, sea $n=2^tn_1-1$ donde $n_1$ es impar. Si $n_1>1$,
	\[f(2^tn_1-1,n_1-1)\equiv f(2^{t-1}n_1-1,n_1-1)\equiv\dots\equiv f(n_1-1,n_1-1)\equiv 0\pmod 2\]
	de donde $n_1=1$. Por ende,
	\[f(n,2k)=f(2^t-1,2k)\equiv f(2^{t-1}-1,2k)\equiv\dots\equiv f(1,2k)=1\pmod 2\]
	para todo $k\in\ZZ^+$. Aquí se termina la prueba.
\end{proof}

\begin{probEG}[EGMO 2022/6]
	Sea $ABCD$ un cuadrilátero cíclico con circuncentro $O$. Sea $X$ el punto de intersección de las bisectrices de los ángulos $\angle DAB$ y $\angle ABC$; sea $Y$ el punto de intersección de las bisectrices de los ángulos $\angle ABC$ y $\angle BCD$; sea $Z$ el punto de intersección de las bisectrices de los ángulos $\angle BCD$ y $\angle CDA$; y sea $W$ el punto de intersección de las bisectrices de los ángulos $\angle CDA$ y $\angle DAB$. Sea $P$ el punto de intersección de las rectas $AC$ y $BD$. Suponga que los puntos $O$, $P$, $X$, $Y$, $Z$ y $W$ son distintos. Pruebe que $O$, $X$, $Y$, $Z$ y $W$ están sobre una misma circunferencia si y solo si $P$, $X$, $Y$, $Z$ y $W$ están sobre una misma circunferencia.
	\begin{hint}
		Let $ABCD$ be a cyclic quadrilateral with circumcentre $O$. Let the internal angle bisectors at $A$ and $B$ meet at $X$, the internal angle bisectors at $B$ and $C$ meet at $Y$, the internal angle bisectors at $C$ and $D$ meet at $Z$, and the internal angle bisectors at $D$ and $A$ meet at $W$. Further, let $AC$ and $BD$ meet at $P$. Suppose that the points $X$, $Y$, $Z$, $W$, $O$ and $P$ are distinct. Prove that $O$, $X$, $Y$, $Z$ and $W$ lie on the same circle if and only if $P$, $X$, $Y$, $Z$ and $W$ lie on the same circle.
	\end{hint}
\end{probEG}

\begin{proof}
	Si $Q=AB\cap CD$, $R=AD\cap BC$, $S=OQ\cap PR$ y $T=OR\cap PQ$ entonces $QS\cdot QO=QA\cdot QB=QY\cdot QW$. Luego, si $\odot(XYZW)\neq\odot(OSPT)$ entonces $QR$ es el eje radical de ellas. Es decir, como $O$ y $P$ no pertenecen a $QR$, entonces $O$ y $P$ no están sobre $\odot(XYZW)$. Si $\odot(XYZW)=\odot(OSPT)$ entonces $O$ y $P$ están sobre $\odot(XYZW)$. Con esto se termina la prueba.
\end{proof}

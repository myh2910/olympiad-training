\section{Week 3 (03/28 -- 04/01)}

\begin{probEG}[Peru EGMO TST 2022/5]
	Sean $x_0,x_1,x_2,\dots,x_n$ números reales distintos entre sí tales que
	\[x_{k-1}x_k\le x_k\le x_kx_{k+1},\quad\text{para todo }k=1,2,\dots,n-1.\]
	Determine el mayor valor posible de $n$.
\end{probEG}

\begin{proof}
	El mayor valor de $n$ es $6$ y un ejemplo que cumple es
	\[(x_0,x_1,\dots,x_6)=(5/2,-1/2,0,1/2,1,3/2,2).\]
	Si $n\ge 7$, supongamos que $x_i>0$ para algún $1\le i\le 3$. Considerando índices $i\le k\le i+3$ tenemos que $x_{i+1}\ge 1$, $x_{i+2}\ge 1$, $x_{i+1}\le 1$ y $x_{i+2}\le 1$ de donde $x_{i+1}=x_{i+2}=1$, lo cual es un absurdo. Por ende, $x_1,x_2,x_3\le 0$. Si $x_i<0$ para algún $2\le i\le 3$, como $x_{i-1}x_i\le x_i$ tenemos que $x_{i-1}\ge 1>0$, lo cual es una contradicción. Luego, $x_2=x_3=0$ lo cual es un absurdo.
\end{proof}

\begin{probEG}[Peru EGMO TST 2022/6]
	Para cada entero positivo $n$, sea $S(n)$ la suma de todos los dígitos de $n$. Pruebe que para todo entero positivo $n\ge 2022$ se cumple que $\floor{\frac{n}{S(n)}}\ne\floor{\frac{n+1}{S(n+1)}}$.
\end{probEG}

\begin{proof}
	Supongamos que $\floor{\frac{n}{S(n)}}=\floor{\frac{n+1}{S(n+1)}}=k\in\ZZ^+_0$. Si $1\le a\le 9$ y $\alpha\ge 3$ son enteros tales que $a\cdot 10^\alpha\le n<(a+1)\cdot 10^\alpha$, tenemos
	\begin{align*}
		n
		&\ge a\cdot 10^\alpha>a\cdot 100\alpha^2>a\cdot(81\alpha^2+27\alpha+10) \\
		&\ge (a+9\alpha)(a+9\alpha+1) \\
		&\ge S(n)\cdot(S(n)+1)
	\end{align*} 
	de donde $k>\frac{n}{S(n)}-1>S(n)$. Luego,
	\[S(n)-S(n+1)<\frac{k\cdot (S(n)+1)}{k+1}-\frac{n+1}{k+1}\le\frac{k-1}{k+1}<1\]
	de donde $S(n+1)=S(n)+1$. Por ende,
	\[0<k\cdot(S(n)+1)-(k+1)\cdot S(n)<(n+1)-n=1\]
	lo cual es un absurdo.
\end{proof}

\begin{probEG}[Peru EGMO TST 2022/7]
	Sea $n\ge 3$ un número entero. Se tienen $n$ colores distintos $C_1,C_2,\dots,C_n$ y una cantidad ilimitada de fichas de cada uno de dichos colores. Decimos que un entero $m\ge n+1$ es \emph{n-colorido} si es posible colocar $m$ fichas alrededor de un círculo de modo que en cualquier grupo de $n+1$ fichas consecutivas haya al menos una ficha de cada uno de los colores $C_1,C_2,\dots,C_n$. Pruebe que sólo existe una cantidad finita de enteros $m$ que no son $n$-coloridos y encuentre el mayor de ellos.
\end{probEG}

\begin{proof}
	Si $m=n(n-1)-1$ es un $n$-colorido, existe un color $C_i$ que se repite a lo sumo $n-2$ veces. Si consideramos una ficha de color $C_i$, podemos dividir las fichas restantes en $n-2$ sectores de $n+1$ fichas consecutivos. Entonces, existe un sector que no contiene al color $C_i$, lo cual es un absurdo. Ahora, sea $m=nq+r$ un entero positivo donde $q$ y $0\le r<n$ son enteros. Si $m\ge n(n-1)$, probaremos que $m$ es $n$-colorido. Primero, vamos a dividir el círculo en $q$ sectores, donde en cada sector colocaremos $n$ fichas de colores $C_1,C_2,\dots,C_n$ en ese orden. Como $q\ge n-1\ge r$, es posible elegir $r$ pares de sectores adyacentes $(S_i,S_{i+1})$ e insertar una ficha de cualquier color entre $S_i$ y $S_{i+1}$. Es fácil ver que esto cumple, así que $m$ es un $n$-colorido. Finalmente, $n^2-n-1$ es el mayor entero que no es $n$-colorido.
\end{proof}

\begin{probEG}[Kyiv City MO 2022 Round 2/11.3]
	Find the largest $k$ for which there exists a permutation $(a_1,a_2,\dots,a_{2022})$ of integers from $1$ to $2022$ such that for at least $k$ distinct $i$ with $1\le i\le 2022$ the number
	\[\frac{a_1+a_2+\dots+a_i}{1+2+\dots+i}\]
	is an integer larger than $1$.
	\aops{24269928}
\end{probEG}

\begin{proof}
	El mayor valor de $k$ es $1011$ y un ejemplo es
	\[a_i=\begin{cases}
		2i&\text{si }i\le 1011, \\
		2i-2023&\text{si }i>1011.
	\end{cases}\]
	Ahora, supongamos que $k\ge 1012$. Sea
	\[A_i=\frac{a_1+a_2+\dots+a_i}{1+2+\dots+i}\]
	para todo índice $i$. Luego, existen dos índices $i>j\ge 1011$ tales que $A_i,A_j>1$ son enteros. Si $A_i\ge 3$, tenemos que
	\begin{align*}
		2023i-\frac{i(i+1)}{2}
		&=2022+2021+\dots+(2023-i) \\
		&\ge a_1+a_2+\dots+a_i \\
		&\ge 3(1+2+\dots+i) \\
		&=\frac{3i(i+1)}{2}
	\end{align*}
	de donde $i\le 1010$, lo cual es un absurdo. Por ende, $A_i=2$ y análogamente $A_j=2$. Luego,
	\begin{align*}
		a_{j+1}+a_{j+2}+\dots+a_i
		&=(a_1+a_2+\dots+a_i)-(a_1+a_2+\dots+a_j) \\
		&=2(1+2+\dots+i)-2(1+2+\dots+j) \\
		&=2((j+1)+(j+2)+\dots+i) \\
		&>\underbrace{2022+2022+\dots+2022}_{i-j\text{ veces}}
	\end{align*}
	lo cual es un absurdo.
\end{proof}

\begin{probMB}[MEMO 2021 T-2]
	Given a positive integer $n$, we say that a polynomial $P$ with real coefficients is \emph{n-pretty} if the equation $P(\floor{x})=\floor{P(x)}$ has exactly $n$ real solutions. Show that for each positive integer $n$
	\begin{enumerate}[(a)]
		\ii there exists an $n$-pretty polynomial;
		\ii any $n$-pretty polynomial has a degree of at least $\frac{2n+1}{3}$.
	\end{enumerate}
	\aops[nobreak]{23091282}
\end{probMB}

\begin{proof}
	Sea $A$ el conjunto de las soluciones reales a la ecuación $P(\floor x)=\floor{P(x)}$. Note que si $x\in A$ entonces $\floor x\in A$. Luego,
	\[A=I\cup\bigcup_{i\in I}S_i\]
	donde $I=\{x\in\ZZ:P(x)\in\ZZ\}$ y $S_i=\{x\in(i,i+1):P(x)\in[P(i),P(i)+1)\}$. Por ende,
	\[n=\abs A=\abs I+\sum_{i\in I}\abs{S_i}.\]
	Si $\abs I>\deg P$, es posible construir un polinomio $Q\in\QQ[x]$ tal que $\deg Q\le\deg P$ y $P(x)-Q(x)$ tiene más de $\deg P$ raíces reales. Luego, $P(x)\equiv Q(x)$ de donde existen infinitos $x\in\ZZ$ tales que $P(x)\in\ZZ$, lo cual es un absurdo ya que $\abs I\le n$. Es decir, $\abs I\le\deg P$. Sea $\delta\in(i,i+1)$ para algún $i\in I$. Si $P(\delta)>P(i)$, para todo $\epsilon\in(P(i),P(\delta))$ existe algún $x\in(i,\delta)$ tal que $P(x)=\epsilon$. Luego, existen infinitos $x\in(i,i+1)$ tales que $P(x)\in(P(i),P(i)+1)$, lo cual es un absurdo ya que $\abs{S_i}\le n$. Por ende, $P(x)\le P(i)$ para todo $x\in(i,i+1)$ y $P(x)=P(i)$ para todo $x\in S_i$. Es decir, en el intérvalo $(i,i+1)$ existen $2\abs{S_i}$ ceros reales de la derivada de $P$. Luego,
	\[n=\abs I+\sum_{i\in I}\abs{S_i}<\deg P+\frac12\deg P=\frac32\deg P\]
	de donde $\deg P\ge\frac{2n+1}{3}$. Un ejemplo de un polinomio $n$-bonito es
	\[P(x)=-\pi(x-1)^2(x-2)^2\cdots(x-n)^2\]
	donde $I=\{1,2,\dots,n\}$ y $S_i=\varnothing$ para todo $i\in I$.
\end{proof}

\begin{probMG}[MEMO 2020 I-4]
	Find all positive integers $n$ for which there exist positive integers $x_1,x_2,\dots,x_n$ such that
	\[\frac{1}{x_1^2}+\frac{2}{x_2^2}+\frac{4}{x_3^2}+\dots+\frac{2^{n-1}}{x_n^2}=1.\]
	\aops[nobreak]{17377469}
\end{probMG}

\begin{proof}
	Probaremos que $\ZZ^+\setminus\{2\}$ es el conjunto de valores de $n$.
	\begin{itemize}
		\ii Si $n=1$, es suficiente con $x_1=1$.
		\ii Si $n=2$, es claro que $x_1,x_2\ge 2$. Luego, $1=\frac{1}{x_1^2}+\frac{2}{x_2^2}\le\frac34$ lo cual es un absurdo.
		\ii Si $n\ge 3$ es impar, sean $x_n=2^{n-1}$ y $x_i=2^\frac{n-1}{2}$ para todo $1\le i\le n-1$. Luego,
		\[\sum_{i=1}^n\frac{2^{i-1}}{x_i^2}=\sum_{i=1}^{n-1}\frac{2^{i-1}}{2^{n-1}}+\frac{2^{n-1}}{(2^{n-1})^2}=\frac{2^{n-1}-1}{2^{n-1}}+\frac{1}{2^{n-1}}=1.\]
		\ii Si $n\ge 4$ es par, sean $(x_1,x_{n-1},x_n)=\left(3\cdot 2^\frac{n-2}{2},2^{n-2},3\cdot 2^{n-3}\right)$ y $x_i=2^\frac{n-2}{2}$ para todo $2\le i\le n-2$. Luego,
		\begin{align*}
			\sum_{i=1}^n\frac{2^{i-1}}{x_i^2}
			&=\frac{1}{9\cdot 2^{n-2}}+\sum_{i=2}^{n-2}\frac{2^{i-1}}{2^{n-2}}+\frac{2^{n-2}}{(2^{n-2})^2}+\frac{2^{n-1}}{9\cdot(2^{n-3})^2} \\
			&=\frac{1}{2^{n-2}}+\frac{2^{n-2}-2}{2^{n-2}}+\frac{1}{2^{n-2}} \\
			&=1.
		\end{align*}
	\end{itemize}
\end{proof}

\begin{probEG}[MEMO 2018 I-4]
	\begin{enumerate}[(a)]
		\ii Prove that for every positive integer $m$ there exists an integer $n\ge m$ such that
		\begin{equation}\label{eq:floor_product}\tag{$\ast$}
			\floor{\frac n1}\cdot\floor{\frac n2}\cdots\floor{\frac nm}=\binom nm.
		\end{equation}
		\ii Denote by $p(m)$ the smallest integer $n\ge m$ such that the equation \eqref{eq:floor_product} holds. Prove that $p(2018)=p(2019)$.
	\end{enumerate}
	\aops[nobreak]{10959197}
\end{probEG}

\begin{proof}
	Note que
	\[\prod_{i=1}^m\floor{\frac ni}\ge\prod_{i=1}^m\frac{n-i+1}{i}=\binom nm\]
	de donde $i\mid n-i+1$ para todo $1\le i\le m$. Es decir, $n=k\cdot\mcm(1,2,\dots,m)-1$ para algún $k\in\ZZ^+$. De esto tenemos que $p(m)=\mcm(1,2,\dots,m)-1$ para todo $m>1$. Como $2019=3\times 673$ divide a $\mcm(1,2,\dots,2018)$, tenemos que
	\[p(2018)=\mcm(1,2,\dots,2018)-1=\mcm(1,2,\dots,2019)-1=p(2019).\]
	Con esto se termina la prueba.
\end{proof}

\begin{probEG}[MEMO 2018 T-7]
	Let $a_1,a_2,a_3,\dots$ be the sequence of positive integers such that
	\[a_1=1\quad\text{and}\quad a_{k+1}=a^3_k+1,\text{ for all positive integers }k.\]
	Prove that for every prime number $p$ of the form $3\ell+2$, where $\ell$ is a non-negative integer, there exists a positive integer $n$ such that $a_n$ is divisible by $p$.
	\aops{10935321}
\end{probEG}

\begin{proof}
	Note que si $x^3\equiv y^3\pmod p$, tenemos que
	\[x\equiv(x^3)^\frac{2p-1}{3}\equiv(y^3)^\frac{2p-1}{3}\equiv y\pmod p.\]
	Como el mapeo $x\mapsto x^3+1\pmod p$ es biyectivo, la secuencia $(a_i)$ es periódica en módulo $p$. Es decir, $a_{n+1}\equiv a_1\equiv 1\pmod p$ para algún $n\in\ZZ^+$, de donde $a_n\equiv 0\pmod p$.
\end{proof}

\begin{probMB}[MEMO 2018 T-8]
	An integer $n$ is called \emph{Silesian} if there exist positive integers $a$, $b$ and $c$ such that
	\[n=\frac{a^2+b^2+c^2}{ab+bc+ca}.\]
	\begin{enumerate}[(a)]
		\ii Prove that there are infinitely many Silesian integers.
		\ii Prove that not every positive integer is Silesian.
	\end{enumerate}
	\aops[nobreak]{10931715}
\end{probMB}

\begin{proof}
	Si $n=4$, digamos que $\mcd(a,b,c)=1$. Luego, $4\mid a^2+b^2+c^2$ de donde $a,b,c$ son pares, lo cual es un absurdo. Por lo tanto, $4$ no es Silesio. Ahora, supongamos que $n=a^2+(1-a)^2+(a^2-a+1)^2$, donde $b=1-a$ y $c=a^2-a+1$ para algún valor de $a$. Podemos comprobar que esto cumple, por lo que $b=1-a$ es una raíz de la ecuación $b^2-n(c+a)b+(c^2+a^2-nca)=0$. Es decir, la otra raíz es igual a $n(c+a)-(1-a)>0$. Por ende, $a^2+(1-a)^2+(a^2-a+1)^2$ es Silesio para todo $a\in\ZZ^+$. Aquí se termina la prueba.
\end{proof}

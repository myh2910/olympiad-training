\section{Semana 7 (04/25 -- 05/01)}

\note[Álgebra]{Lunes\\2022-04-25}

\begin{probEG}
	Determine todas las parejas $(a,b)$ de números reales tales que
	\[a\floor{bn}=b\floor{an},\]
	para todo entero positivo $n$.
\end{probEG}

\begin{proof}
	Respuesta: $a,b\in\ZZ$ o $(a,b)=(0,k),(k,0),(k,k)$ donde $k\in\RR$.
\end{proof}

\begin{probEG}[Indonesia MO Shortlist 2014 A1]
	Sean $a$ y $b$ números reales positivos tales que $\floor{a^k}+\floor{b^k}=\floor{a}^k+\floor{b}^k$ para infinitos enteros positivos $k$. Demuestre que
	\[\floor{a^{2014}}+\floor{b^{2014}}=\floor{a}^{2014}+\floor{b}^{2014}.\]
	\begin{hint}
		Let $a,b$ be positive real numbers such that there exist infinite number of natural numbers $k$ such that $\floor{a^k}+\floor{b^k}=\floor{a}^k+\floor{b}^k$. Prove that
		\[\floor{a^{2014}}+\floor{b^{2014}}=\floor{a}^{2014}+\floor{b}^{2014}.\]
	\end{hint}
	\aops{12396258}
\end{probEG}

\begin{proof}
	Note que $a^k\ge\floor{a}^k$ de donde $\floor{a^k}\ge\floor{a}^k$ para todo $k\in\ZZ^+$. Luego, se cumple la igualdad para infinitos $k\in\ZZ^+$ de donde $a,b\in\ZZ^+\cup(0,1)$. Finalmente, $\floor{a^k}+\floor{b^k}=\floor{a}^k+\floor{b}^k$ para todo $k\in\ZZ^+$.
\end{proof}

\begin{probEG}
	Determine todas las parejas $(a,b)$ de números reales tales que
	\[\floor{a\floor{bn}}=n-1,\]
	para todo entero positivo $n$.
\end{probEG}

\begin{proof}
	Si $a>0$ tenemos que $-1\le n(ab-1)<a$ y si $a<0$ tenemos que $0>n(ab-1)>a-1$ de donde $n\abs{ab-1}$ es acotado. Por ende, $ab=1$ y $a>0$.
\end{proof}

\begin{probMG}
	Sea $n\ge 2$ un número entero. Los $n$ conjuntos finitos $A_1,A_2,\dots,A_n$ satisfacen:
	\[\abs{A_i\,\triangle\,A_j}=\abs{i-j},\quad\forall\,i,j=1,2,\dots,n.\]
	Determine el mínimo valor de
	\[\sum_{i=1}^n\abs{A_i}.\]
\end{probMG}

\begin{proof}
	La respuesta es $\floor{n^2/4}$ y un ejemplo es cuando $A_i$ es el conjunto de todos los enteros $x\ne\floor{\frac{n+1}{2}}$ que están entre $i$ y $\floor{\frac{n+1}{2}}$ (tendremos $\abs{A_i}=\abs{\floor{\frac{n+1}{2}}-i}$ en este caso). Ahora, note que
	\[\abs{A_i}+\abs{A_{n+1-i}}\ge\abs{A_i\,\triangle\,A_{n+1-i}}=\abs{n+1-2i}\]
	para todo $1\le i\le n$ de donde
	\[\sum_{i=1}^n\abs{A_i}=\half\sum_{i=1}^n(\abs{A_i}+\abs{A_{n+1-i}})\ge\half\sum_{i=1}^n\abs{n+1-2i}=\floor{\frac{n^2}{4}}.\]
\end{proof}

\begin{problem}
	Sea $\alpha\ge 1$ un número real y sea $n$ un entero positivo tal que
	\[\floor{\alpha^{n+1}},\floor{\alpha^{n+2}},\dots,\floor{\alpha^{4n}}\]
	son todos cuadrados perfectos. Demuestre que $\floor{\alpha}$ es un cuadrado perfecto.
\end{problem}

\note[Combinatoria]{Martes\\2022-04-26}

\begin{probEG}
	Sea $A=(a_1,a_2,\dots,a_{2001})$ una secuencia de enteros positivos. Sea $m$ el número de subsecuencias de tres términos $(a_i,a_j,a_k)$ tales que $a_k=a_j+1$ y $a_j=a_i+1$. Considerando todas las secuencias $A$, determine el mayor valor de $m$.
\end{probEG}

\begin{proof}
	Respuesta: $667^3$.
\end{proof}

\begin{probEG}
	Para $i=1,2,\dots,11$, sea $M_i$ un conjunto de $5$ elementos, y asuma que para $1\le i<j\le 11$, $M_i\cap M_j\ne\varnothing$. Sea $m$ el mayor número para el cual existen $m$ conjuntos $M_{x_1},M_{x_2},\dots,M_{x_m}$ tales que $M_{x_1}\cap M_{x_2}\cap\dots\cap M_{x_m}\ne\varnothing$. Determine el mínimo valor de $m$ sobre todos los posibles conjuntos iniciales.
\end{probEG}

\begin{proof}
	Respuesta: $4$.
\end{proof}

\begin{probMR}
	Determine cuantos subconjuntos de $\{1,2,\dots,2000\}$ tienen suma de elemento múltiplo de $5$.
\end{probMR}

\begin{proof}
	Sea
	\[P(x)=\prod_{i=1}^{2000}(x^i+1)=\sum_{j=0}^{\deg P}c_jx^j\]
	un polinomio y sea $\alpha$ una raíz de $x^5-1$ tal que $\alpha^4+\alpha^3+\alpha^2+\alpha+1=0$. Luego,
	\[2^{2000}+4\cdot 2^{400}=\sum_{i=0}^4 P(\alpha^i)=\sum_{j=0}^{\deg P}c_j\cdot\sum_{i=0}^4\alpha^{ij}=5\sum_{5\,\mid\,j}c_j\]
	de donde la respuesta es $\dfrac{2^{2000}+4\cdot 2^{400}}{5}$.
\end{proof}

\begin{probEG}
	Sea $X$ un conjunto finito y no vacío de enteros positivos, y sea $A$ un subconjunto de $X$. Demuestre que existe $B\subset X$ tal que $A$ es el conjunto de todos los elementos de $X$ que dividen a un número impar de elementos de $B$.
\end{probEG}

\begin{proof}
	Nos basta probar que la función $f:X\to X$ tal que
	\[f(A)=\{x\in X:x\text{ divide a un número impar de elementos de }A\}\]
	es biyectiva. Primero, probaremos que $f$ es inyectiva y para eso supongamos que existen $A,B\subset X$ tales que $f(A)=f(B)$. Si $A\ne B$, sea $x$ el mayor elemento de $A\,\triangle\,B$ donde $x\in A$ sin pérdida de generalidad. Luego, $x$ divide a un número más del conjunto $A$ que de $B$, de donde $x\in f(A)\,\triangle\,f(B)=\varnothing$, lo cual es un absurdo. Por lo tanto, $f$ es inyectiva y por ende biyectiva pues $X$ es finito.
\end{proof}

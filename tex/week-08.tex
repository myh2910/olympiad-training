\section{Semana 8 (05/02 -- 05/08)}

\note[Álgebra]{Lunes\\2022-05-02}

\begin{probEG}[ISL 2001/A2]
  Sea $(a_n)_{n\ge 0}$ una secuencia infinita de números reales positivos.
  Demuestre que
  \[1+a_n>a_{n-1}\sqrt[n]{2}\]
  para infinitos $n\in\ZZ^+$.
\end{probEG}

\begin{probMR}[ISL 2001/A3]
  Sean $x_1,x_2,\dots,x_n\in\RR$. Demuestre que
  \[
    \frac{x_1}{1+x_1^2}+\frac{x_2}{1+x_1^2+x_2^2}+\dots+\frac{x_n}{1+x_1^2+\dots+x_n^2}
    <\sqrt{n}.
  \]
\end{probMR}

\note[Combinatoria]{Martes\\2022-05-03}

\begin{probEG}[ISL 2001/C3]
  Define a \emph{k-clique} to be a set of $k$ people such that every pair of
  them are acquainted with each other. At a certain party, every pair of
  $3$-cliques has at least one person in common, and there are no $5$-cliques.
  Prove that there are two or fewer people at the party whose departure leaves
  no $3$-clique remaining.
\end{probEG}

\begin{probMB}[ISL 2001/C4]
  A set of three nonnegative integers $\{x,y,z\}$ with $x<y<z$ is called
  historic if $\{z-y,y-x\}=\{1776,2001\}$. Show that the set of all nonnegative
  integers can be written as the union of pairwise disjoint historic sets.
\end{probMB}

\note[Geometría]{Miércoles\\2022-05-04}

\begin{probEG}[ISL 2001/G1]
  Let $A_1$ be the center of the square inscribed in acute triangle $ABC$ with
  two vertices of the square on side $BC$. Thus one of the two remaining
  vertices of the square is on side $AB$ and the other is on $AC$. Points $B_1$,
  $C_1$ are defined in a similar way for inscribed squares with two vertices on
  sides $AC$ and $AB$, respectively. Prove that lines $AA_1$, $BB_1$, $CC_1$ are
  concurrent.
  \forum[aops]{119194}
\end{probEG}

\begin{probMG}[ISL 2001/G2]
  In acute triangle $ABC$ with circumcenter $O$ and altitude $AP$,
  $\angle C\ge\angle B+30\dg$. Prove that $\angle A+\angle COP<90\dg$.
\end{probMG}

\begin{probMR}[ISL 2001/G3]
  Let $ABC$ be a triangle with centroid $G$. Determine, with proof, the position
  of the point $P$ in the plane of $ABC$ such that
  $AP\cdot AG+BP\cdot BG+CP\cdot CG$ is a minimum, and express this minimum
  value in terms of the side lengths of $ABC$.
\end{probMR}

\note[Teoría de Números]{Viernes\\2022-05-06}

\begin{probEG}
  Demuestre que todo entero positivo coprimo con $3$ posee un múltiplo cuya suma
  de dígitos es un número primo.
\end{probEG}

\begin{proof}
  Como $n$ es coprimo con $3$, existe $m=\ol{a_{k-1}a_{k-2}\cdots a_1a_0}$ tal
  que $m$ es coprimo con $30$ y $n\mid 10^\alpha\cdot m$ para algún
  $\alpha\in\ZZ^+$. Luego,
  \[
    m\cdot(10^t-1)
    =\ol{a_{k-1}a_{k-2}\cdots a_1(a_0-1)\underbrace{99\cdots 9}_{t-k\text{ veces}}(9-a_{k-1})(9-a_{k-2})\cdots(9-a_1)(10-a_0)}
  \]
  de donde $S(m\cdot(10^t-1))=9t$ para todo $t>k$. Por ende, existe un $m\mid M$
  tal que $S(M)$ es coprimo con $S(m)$ y por Dirichlet, la suma de dígitos de
  $\ol{\underbrace{(M)(M)\cdots (M)}_{\beta\text{ veces}}(m)\underbrace{00\cdots 0}_{\alpha\text{ veces}}}$
  es un primo para algún $\beta\in\ZZ^+$. Con esto se termina la prueba, pues
  $M$ y $m$ son múltiplos de $n$.
\end{proof}

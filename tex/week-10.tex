\section{Semana 10 (05/16 -- 05/22)}

\note[Álgebra]{Lunes\\2022-05-16}

\begin{probEG}[ISL 2002/A2]
  Let $a_1,a_2,\dots$ be an infinite sequence of real numbers, for which there
  exists a real number $c$ with $0\le a_i\le c$ for all $i$, such that
  \[\abs{a_i-a_j}\ge\frac{1}{i+j}\quad\text{for all }i,j\text{ with }i\ne j.\]
  Prove that $c\ge 1$.
  \forum[aops]{118699}
\end{probEG}

\begin{probMR}[ISL 2002/A3]
  Let $P$ be a cubic polynomial given by $P(x)=ax^3+bx^2+cx+d$, where $a,b,c,d$
  are integers and $a\ne 0$. Suppose that $xP(x)=yP(y)$ for infinitely many
  pairs $x,y$ of integers with $x\ne y$. Prove that the equation $P(x)=0$ has an
  integer root.
  \forum[aops]{118702}
\end{probMR}

\begin{probMB}[ISL 2002/A4]
  Find all functions $f$ from the reals to the reals such that
  \[\left(f(x)+f(z)\right)\left(f(y)+f(t)\right)=f(xy-zt)+f(xt+yz)\]
  for all real $x,y,z,t$.
  \forum[aops]{118703}
\end{probMB}

\note[Combinatoria]{Martes\\2022-05-17}

\begin{probEG}[ISL 2002/C1]
  Let $n$ be a positive integer. Each point $(x,y)$ in the plane, where $x$ and
  $y$ are non-negative integers with $x+y<n$, is coloured red or blue, subject
  to the following condition: if a point $(x,y)$ is red, then so are all points
  $(x',y')$ with $x'\le x$ and $y'\le y$. Let $A$ be the number of ways to
  choose $n$ blue points with distinct $x$-coordinates, and let $B$ be the
  number of ways to choose $n$ blue points with distinct $y$-coordinates. Prove
  that $A=B$.
  \forum[aops]{118710}
\end{probEG}

\begin{probEG}[ISL 2002/C2]
  For $n$ an odd positive integer, the unit squares of an $n\times n$ chessboard
  are coloured alternately black and white, with the four corners coloured
  black. A \emph{tromino} is an $L$-shape formed by three connected unit
  squares. For which values of $n$ is it possible to cover all the black squares
  with non-overlapping trominos? When it is possible, what is the minimum number
  of trominos needed?
  \forum[aops]{118712}
\end{probEG}

\begin{proof}
  Respuesta: $n\ge 7$ y el mínimo es $\left(\frac{n+1}{2}\right)^2$.
\end{proof}

\begin{probMR}[ISL 2002/C3]
  Let $n$ be a positive integer. A sequence of $n$ positive integers (not
  necessarily distinct) is called \emph{full} if it satisfies the following
  condition: for each positive integer $k\ge 2$, if the number $k$ appears in
  the sequence then so does the number $k-1$, and moreover the first occurrence
  of $k-1$ comes before the last occurrence of $k$. For each $n$, how many full
  sequences are there?
  \forum[aops]{118714}
\end{probMR}

\begin{proof}
  Respuesta: $n!$, considerando una biyección.
\end{proof}

\note[Teoría de Números]{}

\begin{probEG}[Russia 2022\protect\footnote{Kvant/M2693}]
  Demuestre que existe un entero positivo $b$ que tiene la siguiente propiedad:
  para todo entero $n>b$, la suma de los dígitos del número $n!$ es mayor o
  igual que $10^{100}$.
  \forum[aops]{27238158}
  \begin{hint}
    Prove that there exists a natural number $b$ such that for any natural $n>b$
    the sum of the digits of $n!$ is not less than $10^{100}$.
  \end{hint}
\end{probEG}

\begin{proof}
  Note que si $n>b$, entonces $b\mid n!$. Además, $s(x-y)\ge s(x)-s(y)$ para
  todo $x,y\in\ZZ^+$. Sea $b=\underbrace{99\cdots 9}_{N\text{ veces}}$ donde
  $N\in\ZZ^+$ es suficientemente grande, y sea $M\cdot 10^k\cdot b$ un múltiplo
  de $b$ donde $10\nmid M$. Luego,
  \begin{align*}
    s(M\cdot 10^k\cdot b)
    &= s(M\cdot(10^N-1)) \\
    &= s\left(\ol{(M-1)\underbrace{99\cdots 9}_{N\text{ veces}}}-(M-1)\right) \\
    &\ge s(M-1)+9N-s(M-1) \\
    &= 9N
  \end{align*}
  de donde $s(n!)\ge 9N>10^{100}$ para todo $n>b$.
\end{proof}

\note[Geometría]{Miércoles\\2022-05-18}

\begin{probEG}[ISL 2001/G5]
  Let $ABC$ be an acute triangle. Let $DAC$, $EAB$, and $FBC$ be isosceles
  triangles exterior to $ABC$, with $DA=DC$, $EA=EB$, and $FB=FC$, such that
  \[
    \angle ADC=2\angle BAC,\quad
    \angle BEA=2\angle ABC,\quad
    \angle CFB=2\angle ACB.
  \]
  Let $D'$ be the intersection of lines $DB$ and $EF$, let $E'$ be the
  intersection of $EC$ and $DF$, and let $F'$ be the intersection of $FA$ and
  $DE$. Find, with proof, the value of the sum
  \[\frac{DB}{DD'}+\frac{EC}{EE'}+\frac{FA}{FF'}.\]
  \forum[aops]{119201}
\end{probEG}

\begin{proof}
  Si $P$ es un punto interior de $ABC$ tal que
  \[
    \angle CPA=\pi-\angle A,\quad
    \angle APB=\pi-\angle B,\quad
    \angle BPC=\pi-\angle C
  \]
  entonces $P$ es la intersección de circunferencias
  $\odot(D,DA),\odot(E,EB),\odot(F,FC)$ de donde
  \[
    \triangle PDE\equiv\triangle ADE,\quad
    \triangle PEF\equiv\triangle BEF,\quad
    \triangle PFD\equiv\triangle CFD.
  \]
  Luego,
  \[
    \cycsum\frac{DB}{DD'}
    =\cycsum\frac{\abs{DEBF}}{\abs{DEF}}
    =3+\frac{\abs{ADE}+\abs{BEF}+\abs{CFD}}{\abs{PDE}+\abs{PEF}+\abs{PFD}}
    =4.
  \]
\end{proof}

\begin{probEG}[ISL 2001/G6]
  Let $ABC$ be a triangle and $P$ an exterior point in the plane of the
  triangle. Suppose $AP,BP,CP$ meet the sides $BC,CA,AB$ (or extensions thereof)
  in $D,E,F$, respectively. Suppose further that the areas of triangles
  $PBD,PCE,PAF$ are all equal. Prove that each of these areas is equal to the
  area of triangle $ABC$ itself.
  \forum[aops]{119203}
\end{probEG}

\begin{proof}
  Sin pérdida de generalidad, supongamos que $P$ pertenece al lado opuesto de
  $A$ con respecto a $BC$. Si $C$ está entre $B$ y $D$, entonces $\triangle PBD$
  contiene al $\triangle PCE$. Si $B$ está entre $C$ y $D$, entonces
  $\triangle PAF$ contiene al $\triangle PBD$. Por ende, $D$ está en el lado
  $BC$. Si $A$ está entre $E$ y $C$, entonces $\triangle PCE$ contiene al
  $\triangle PBD$. Si $B$ está entre $F$ y $A$, entonces $\triangle PAF$
  contiene al $\triangle PBD$. Es decir, los segmentos dirigidos
  $\frac{PA}{PD},\frac{PB}{PE},\frac{PC}{PF}$ tienen signos $+,-,+$,
  respectivamente. Note que
  \[
    \frac{\abs{PBD}}{\abs{ABC}}=\frac{BD}{BC}\cdot\frac{PD}{AD}\quad\text{y}\quad
    \frac{BD}{BC}\cdot\frac{FC}{FP}\cdot\frac{AP}{AD}=1
  \]
  considerando áreas y segmentos con signos. Es decir,
  \[
    \left(\frac{PC}{PF}-1\right)\cdot\frac{PA}{PD}
    =\frac{\abs{ABC}}{\abs{PBD}}
    =k
  \]
  donde $k\in\RR^-$ es una constante. Con un poco de cálculo obtenemos
  \[(x(x-1)-k)(k+1)=0\]
  donde $x=\frac{PA}{PD}\ge 1$. Por lo tanto, $k=-1$.
\end{proof}

\begin{probEG}[ISL 2001/G7]
  Let $O$ be an interior point of acute triangle $ABC$. Let $A_1$ lie on $BC$
  with $OA_1$ perpendicular to $BC$. Define $B_1$ on $CA$ and $C_1$ on $AB$
  similarly. Prove that $O$ is the circumcenter of $ABC$ if and only if the
  perimeter of $A_1B_1C_1$ is not less than any one of the perimeters of
  $AB_1C_1$, $BC_1A_1$, and $CA_1B_1$.
  \forum[aops]{119204}
\end{probEG}

\begin{proof}
  Digamos que $O'\ne O$ es un punto interior de $\triangle ABC$, siendo $O$ el
  circuncentro de $ABC$. Si $O'$ pertenece en el interior del $\triangle BOC$,
  tenemos que
  \[
    \angle BC_1A_1
    =\angle BO'A_1
    =\frac{\pi}{2}-\angle A_1BO'
    \ge\frac{\pi}{2}-\angle CBO
    =\angle A
  \]
  y análogamente $\angle CB_1A_1\ge\angle A$, de donde
  $A_1B_1+A_1C_1\le AB_1+AC_1$. Es decir, $p(A_1B_1C_1)<p(AB_1C_1)$.
\end{proof}

\begin{probEG}[ISL 2001/G8]
  Let $ABC$ be a triangle with $\angle BAC=60\dg$. Let $AP$ bisect $\angle BAC$
  and let $BQ$ bisect $\angle ABC$, with $P$ on $BC$ and $Q$ on $AC$. If
  $AB+BP=AQ+QB$, what are the angles of the triangle?
  \forum[aops]{119207}
\end{probEG}

\begin{proof}
  Sea $R$ un punto en la recta $AB$ tal que $BR=BP$ y $B$ está entre $R$ y $A$.
  Si $S$ es un punto del rayo $AC$ tal que $AR=AS$, entonces
  \[QS=AS-AQ=AR-AQ=AB+BR-AQ=AB+BP-AQ=QB\]
  y
  \[\angle QSP=\angle ASP=\angle ARP=\half\angle ABP=\angle QBP.\]
  Como $BPSQ$ no es un paralelogramo, $QP$ es bisectriz de $\angle BQC$. Es
  decir,
  \[\frac{BQ}{QC}=\frac{BP}{PC}=\frac{BA}{AC}=\frac{c}{b}\]
  de donde $BQ=\frac{ca}{c+a}$. Además, como $\angle A=60\dg$, tenemos que
  \[
    \left(\frac{ca}{c+a}\right)^2
    =BQ^2
    =AB^2+AQ^2-AB\cdot AQ
    =c^2+\left(\frac{bc}{c+a}\right)^2-\frac{bc^2}{c+a}
  \]
  y
  \[a^2=BC^2=AB^2+AC^2-AB\cdot AC=b^2+c^2-bc.\]
  Simplificando nos queda que $a=b-2c$, y de esto no es difícil ver que los
  ángulos $\angle A,\angle B,\angle C$ son $60\dg,105\dg,15\dg$.
\end{proof}

\note[Punto\\Humpty (HM) \cite{ref:humpty}]{Jueves\\2022-05-19}

\begin{probEG}
  In $\triangle ABC$ with orthocenter $H$, the circle with diameter $\ol{AH}$
  and $\odot(BHC)$ intersect again on the $A$-median at a point $X_A$.
\end{probEG}

\begin{proof}
  Si $A'$ es la reflexión de $A$ con respecto a $BC$, entonces $HA'$ es el
  diámetro de $\odot(BHC)$ de donde $X_A$ pertenece a $AA'$, que claramente es
  la $A$-mediana.
\end{proof}

\begin{probEG}[ELMO Shortlist 2013/G3]
  In $\triangle ABC$, a point $D$ lies on line $BC$. The circumcircle of $ABD$
  meets $AC$ at $F$ (other than $A$), and the circumcircle of $ADC$ meets $AB$
  at $E$ (other than $A$). Prove that as $D$ varies, the circumcircle of $AEF$
  always passes through a fixed point other than $A$, and that this point lies
  on the median from $A$ to $BC$.
  \forum[aops]{3151962}
\end{probEG}

\begin{proof}
  Si $P=BF\cap EC$ y $A'$ la reflexión de $A$ con respecto a $BC$, entonces $P$
  pertenece a $\odot(AEF)$ y $\odot(BHC)$. Como $AA'$ pasa por $X_A$,
  $\angle AX_AP=\angle A'CP=\angle AEP$ de donde $X_A\in\odot(AEF)$.
\end{proof}

\begin{probEG}[ELMO 2014/5]
  Let $ABC$ be a triangle with circumcenter $O$ and orthocenter $H$. Let
  $\omega_1$ and $\omega_2$ denote the circumcircles of triangles $BOC$ and
  $BHC$, respectively. Suppose the circle with diameter $\ol{AO}$ intersects
  $\omega_1$ again at $M$, and line $AM$ intersects $\omega_1$ again at $X$.
  Similarly, suppose the circle with diameter $\ol{AH}$ intersects $\omega_2$
  again at $N$, and line $AN$ intersects $\omega_2$ again at $Y$. Prove that
  lines $MN$ and $XY$ are parallel.
  \forum[aops]{3534946}
\end{probEG}

\begin{proof}
  Pista: con angulitos tenemos que $M$ es el punto medio de la cuerda
  $A$-simediana, $N$ es el $A$-punto Humpty, $X$ es la intersección de las
  tangentes a $\odot(ABC)$ por $B$ y $C$, $Y$ es la reflexión de $A$ con
  respecto a $BC$.
\end{proof}

\begin{probEG}[USA TST 2005/6]
  Let $ABC$ be an acute scalene triangle with $O$ as its circumcenter. Point $P$
  lies inside triangle $ABC$ with $\angle PAB=\angle PBC$ and
  $\angle PAC=\angle PCB$. Point $Q$ lies on line $BC$ with $QA=QP$. Prove that
  $\angle AQP=2\angle OQB$.
  \forum[aops]{734440}
\end{probEG}

\begin{probEG}[Brazil National Olympiad 2015/6]
  Let $\triangle ABC$ be a scalene triangle and $X$, $Y$ and $Z$ be points on
  the lines $BC$, $AC$ and $AB$, respectively, such that
  $\dang AXB=\dang BYC=\dang CZA$. The circumcircles of $BXZ$ and $CXY$
  intersect at $P$. Prove that $P$ is on the circumference which diameter has
  ends in the orthocenter $H$ and in the barycenter $G$ of $\triangle ABC$.
  \forum[aops]{5469201}
\end{probEG}

\begin{proof}
  Se prueba que $P$ es el centro de roto-homotecia que manda
  $\triangle ABC\mapsto\triangle A_1B_1C_1$ donde $A_1=BY\cap CZ$ y análogamente
  para $B_1$ y $C_1$. Con esto se prueba que $\dang BAP=\dang CBP$ y
  $\dang CAP=\dang BCP$, de donde $P$ es el $A$-punto Humpty, entonces
  $\angle HPA=\angle HPG=90\dg$.
\end{proof}

\begin{probEG}[Sharygin Geometry Olympiad 2015 Final Round/10.3]
  Let $A_1$, $B_1$ and $C_1$ be the midpoints of sides $BC$, $CA$ and $AB$ of
  triangle $ABC$, respectively. Points $B_2$ and $C_2$ are the midpoints of
  segments $BA_1$ and $CA_1$ respectively. Point $B_3$ is symmetric to $C_1$
  with respect to $B$, and $C_3$ is symmetric to $B_1$ with respect to $C$.
  Prove that one of common points of circles $BB_2B_3$ and $CC_2C_3$ lies on the
  circumcircle of triangle $ABC$.
  \forum[aops]{10667375}
\end{probEG}

\begin{proof}
  Considere al centro de roto-homotecia que manda $AB\mapsto B_1B_2$ y otra que
  manda $AC\mapsto C_1C_2$, y sea $P$ ese punto. Luego,
  \[\angle BB_3B_2=\angle(AB,B_1B_2)=\angle BPB_2\]
  de donde $P\in\odot(BB_2B_3)$ y análogamente $P\in\odot(CC_2C_3)$. Además,
  \[\angle PBB_2+\angle PCC_2=\angle PAB_1+\angle PAC_1=\angle A\]
  de donde $P\in\odot(ABC)$.
\end{proof}

\note{Viernes\\2022-05-20}

\begin{problem}[ISL 2014/G6]
  Let $ABC$ be a fixed acute-angled triangle. Consider some points $E$ and $F$
  lying on the sides $AC$ and $AB$, respectively, and let $M$ be the midpoint of
  $EF$. Let the perpendicular bisector of $EF$ intersect the line $BC$ at $K$,
  and let the perpendicular bisector of $MK$ intersect the lines $AC$ and $AB$
  at $S$ and $T$, respectively. We call the pair $(E,F)$ \emph{interesting}, if
  the quadrilateral $KSAT$ is cyclic. Suppose that the pair $(E_1,F_1)$ and
  $(E_2,F_2)$ are interesting. Prove that
  \[\frac{E_1E_2}{AB}=\frac{F_1F_2}{AC}.\]
\end{problem}

\begin{proof}
  Como $M$ pertenece a la mediana del $\triangle AST$ y su reflexión con
  respecto a $BC$ pertenece a $\odot(AST)$, entonces $AK$ es la simediana del
  $\triangle AST$. Como $K$ pertenece a la mediatriz del segmento $EF$, tenemos
  que $KE$ y $KF$ son tangentes a $\odot(AEF)$. Luego,
  \begin{align*}
    \frac{AE}{AB}
    &= \frac{AE}{EF}\cdot\frac{EF}{KE}\cdot\frac{KE}{CK}\cdot\frac{CK}{BC}\cdot\frac{BC}{AB} \\
    &= \frac{\sin{\alpha}}{\sin{\angle A}}\cdot\frac{\sin{2\angle A}}{\sin{\angle A}}\cdot\frac{\sin{\angle C}}{\sin{\alpha}}\cdot\frac{CK}{BC}\cdot\frac{\sin{\angle A}}{\sin{\angle C}} \\
    &= 2\cos{\angle A}\cdot\frac{CK}{BC}
  \end{align*}
  donde $\alpha=\angle AFE=\angle CEK$, y análogamente
  $\frac{AF}{AC}=2\cos{\angle A}\cdot\frac{BK}{BC}$. Por ende,
  \[\frac{AE}{AB}+\frac{AF}{AC}=2\cos{\angle A}\]
  es una constante. Con esto es suficiente.
\end{proof}

\begin{probEG}[USA TSTST 2015/2]
  Let $ABC$ be a scalene triangle. Let $K_a$, $L_a$ and $M_a$ be the respective
  intersections with $BC$ of the internal angle bisector, external angle
  bisector, and the median from $A$. The circumcircle of $AK_aL_a$ intersects
  $AM_a$ a second time at point $X_a$ different from $A$. Define $X_b$ and $X_c$
  analogously. Prove that the circumcenter of $X_aX_bX_c$ lies on the Euler line
  of $ABC$.
  \forum[aops]{5017915}
\end{probEG}

\begin{proof}
  Como $X_a$ es un punto de la $A$-mediana que pertenece a la $A$-circunferencia
  de Apolonio, entonces $X_a$ es el $A$-punto Humpty del $\triangle ABC$. Por
  ende, $\angle HX_aG=90\dg$ y el circuncentro del $\triangle X_aX_bX_c$ es el
  punto medio del segmento $HG$.
\end{proof}

\begin{probEG}[WOOT 2013 Practice Olympiad 3/5]
  A semicircle has center $O$ and diameter $AB$. Let $M$ be a point on $AB$
  extended past $B$. A line through $M$ intersects the semicircle at $C$ and
  $D$, so that $D$ is closer to $M$ than $C$. The circumcircles of triangles
  $AOC$ and $DOB$ intersect at $O$ and $K$. Show that $\angle MKO=90\dg$.
\end{probEG}

\begin{probEG}[IMO 2010/4, modified]
  In $\triangle ABC$ with orthocenter $H$, suppose $P$ is the projection of $H$
  onto the $C$-median; let the second intersections of $AP,BP,CP$ with
  $\odot(ABC)$ be $K,L,M$ respectively. Show that $MK=ML$.
\end{probEG}

\begin{probEG}[EGMO 2016/4]
  Two circles $\omega_1$ and $\omega_2$, of equal radius intersect at different
  points $X_1$ and $X_2$. Consider a circle $\omega$ externally tangent to
  $\omega_1$ at $T_1$ and internally tangent to $\omega_2$ at point $T_2$. Prove
  that lines $X_1T_1$ and $X_2T_2$ intersect at a point lying on $\omega$.
\end{probEG}

\begin{probEG}[USA TST 2008/7]
  Let $ABC$ be a triangle with $G$ as its centroid. Let $P$ be a variable point
  on segment $BC$. Points $Q$ and $R$ lie on sides $AC$ and $AB$ respectively,
  such that $PQ\parallel AB$ and $PR\parallel AC$. Prove that, as $P$ varies
  along segment $BC$, the circumcircle of triangle $AQR$ passes through a fixed
  point $X$ such that $\angle BAG=\angle CAX$.
  \forum[aops]{1247506}
\end{probEG}

\begin{probEG}[Mathematical Reflections/O371]
  Let $ABC$ be a triangle with $AB<AC$. Let $D,E$ be the feet of altitudes from
  $B,C$ to sides $AC,AB$ respectively. Let $M,N,P$ be the midpoints of the
  segments $BC,MD,ME$ respectively. Let $NP$ intersect $BC$ again at a point $S$
  and let the line through $A$ parallel to $BC$ intersect $DE$ again at point
  $T$. Prove that $ST$ is tangent to the circumcircle of triangle $ADE$.
\end{probEG}

\begin{problem}[ELMO Shortlist 2012/G7]
  Let $\triangle ABC$ be an acute triangle with circumcenter $O$ such that
  $AB<AC$, let $Q$ be the intersection of the external bisector of $\angle A$
  with $BC$, and let $P$ be a point in the interior of $\triangle ABC$ such that
  $\triangle BPA$ is similar to $\triangle APC$. Show that
  $\angle QPA+\angle OQB=90\dg$.
  \forum[aops]{2728473}
\end{problem}

\begin{problem}[Iranian Geometry Olympiad 2014/S4]
  The tangent to the circumcircle of an acute triangle $ABC$ (with $AB<AC$) at
  $A$ meets $BC$ at $P$. Let $X$ be a point on line $OP$ such that
  $\angle AXP=90\dg$. Points $E$ and $F$ lie on sides $AB$ and $AC$,
  respectively, and are on the same side of line $OP$ such that
  $\dang EXP=\dang ACX$ and $\dang FXO=\dang ABX $. Let $EF$ meet the
  circumcircle of triangle $ABC$ at points $K,L$. Prove that the line $OP$ is
  tangent to the circumcircle of triangle $KLX$.
  \forum[aops]{3758542}
\end{problem}

\note[Teoría de Números]{}

\begin{probEG}[ISL 2001/N4]
  Let $p\ge 5$ be a prime number. Prove that there exists an integer $a$ with
  $1\le a\le p-2$ such that neither $a^{p-1}-1$ nor $(a+1)^{p-1}-1$ is divisible
  by $p^2$.
\end{probEG}

\begin{probMG}[ISL 2001/N5]
  Let $a>b>c>d$ be positive integers and suppose that
  \[ac+bd=(b+d+a-c)(b+d-a+c).\]
  Prove that $ab+cd$ is not prime.
\end{probMG}

\begin{probHR}[ISL 2001/N6]
  Is it possible to find $100$ positive integers not exceeding $25000$, such
  that all pairwise sums of them are different?
\end{probHR}

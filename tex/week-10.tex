\section{Semana 10 (05/16 -- 05/22)}

\note[Álgebra]{Lunes\\2022-05-16}

\begin{probEG}[ISL 2002/A2]
	Let $a_1,a_2,\dots$ be an infinite sequence of real numbers, for which there exists a real number $c$ with $0\le a_i\le c$ for all $i$, such that
	\[\abs{a_i-a_j}\ge\frac{1}{i+j}\quad\text{for all }i,j\text{ with }i\ne j.\]
	Prove that $c\ge 1$.
	\aops{118699}
\end{probEG}

\begin{probMR}[ISL 2002/A3]
	Let $P$ be a cubic polynomial given by $P(x)=ax^3+bx^2+cx+d$, where $a,b,c,d$ are integers and $a\ne 0$. Suppose that $xP(x)=yP(y)$ for infinitely many pairs $x,y$ of integers with $x\ne y$. Prove that the equation $P(x)=0$ has an integer root.
	\aops{118702}
\end{probMR}

\begin{probMB}[ISL 2002/A4]
	Find all functions $f$ from the reals to the reals such that
	\[\left(f(x)+f(z)\right)\left(f(y)+f(t)\right)=f(xy-zt)+f(xt+yz)\]
	for all real $x,y,z,t$.
	\aops{118703}
\end{probMB}

\note[Combinatoria]{Martes\\2022-05-17}

\begin{probEG}[ISL 2002/C1]
	Let $n$ be a positive integer. Each point $(x,y)$ in the plane, where $x$ and $y$ are non-negative integers with $x+y<n$, is coloured red or blue, subject to the following condition: if a point $(x,y)$ is red, then so are all points $(x',y')$ with $x'\le x$ and $y'\le y$. Let $A$ be the number of ways to choose $n$ blue points with distinct $x$-coordinates, and let $B$ be the number of ways to choose $n$ blue points with distinct $y$-coordinates. Prove that $A=B$.
	\aops{118710}
\end{probEG}

\begin{probEG}[ISL 2002/C2]
	For $n$ an odd positive integer, the unit squares of an $n\times n$ chessboard are coloured alternately black and white, with the four corners coloured black. A \emph{tromino} is an $L$-shape formed by three connected unit squares. For which values of $n$ is it possible to cover all the black squares with non-overlapping trominos? When it is possible, what is the minimum number of trominos needed?
	\aops{118712}
\end{probEG}

\begin{proof}
	Respuesta: $n\ge 7$ y el mínimo es $\left(\frac{n+1}{2}\right)^2$.
\end{proof}

\begin{probMR}[ISL 2002/C3]
	Let $n$ be a positive integer. A sequence of $n$ positive integers (not necessarily distinct) is called \emph{full} if it satisfies the following condition: for each positive integer $k\ge 2$, if the number $k$ appears in the sequence then so does the number $k-1$, and moreover the first occurrence of $k-1$ comes before the last occurrence of $k$. For each $n$, how many full sequences are there?
	\aops{118714}
\end{probMR}

\begin{proof}
	Respuesta: $n!$, considerando una biyección.
\end{proof}

\note[Teoría de Números]{}

\begin{probEG}[Russia 2022]
	Demuestre que existe un entero positivo $b$ que tiene la siguiente propiedad: para todo entero $n>b$, la suma de los dígitos del número $n!$ es mayor o igual que $10^{100}$.
\end{probEG}

\begin{proof}
	Note que si $n>b$, entonces $b\mid n!$. Además, $s(x-y)\ge s(x)-s(y)$ para todo $x,y\in\ZZ^+$. Sea $b=\underbrace{99\cdots 9}_{N\text{ veces}}$ donde $N\in\ZZ^+$ es suficientemente grande, y sea $M\cdot 10^k\cdot b$ un múltiplo de $b$ donde $10\nmid M$. Luego,
	\begin{align*}
		s(M\cdot 10^k\cdot b)
		&=s(M\cdot(10^N-1))\\
		&=s\left(\ol{(M-1)\underbrace{99\cdots 9}_{N\text{ veces}}}-(M-1)\right)\\
		&\ge s(M-1)+9N-s(M-1)\\
		&=9N
	\end{align*}
	de donde $s(n!)\ge 9N>10^{100}$ para todo $n>b$.
\end{proof}

\note[Geometría]{Miércoles\\2022-05-18}

\begin{probEG}[ISL 2001/G5]
	Let $ABC$ be an acute triangle. Let $DAC$, $EAB$, and $FBC$ be isosceles triangles exterior to $ABC$, with $DA=DC$, $EA=EB$, and $FB=FC$, such that
	\[\angle ADC=2\angle BAC,\quad\angle BEA=2\angle ABC,\quad\angle CFB=2\angle ACB.\]
	Let $D'$ be the intersection of lines $DB$ and $EF$, let $E'$ be the intersection of $EC$ and $DF$, and let $F'$ be the intersection of $FA$ and $DE$. Find, with proof, the value of the sum
	\[\frac{DB}{DD'}+\frac{EC}{EE'}+\frac{FA}{FF'}.\]
	\aops{119201}
\end{probEG}

\begin{proof}
	Si $P$ es un punto interior de $ABC$ tal que
	\[\angle CPA=\pi-\angle A,\quad\angle APB=\pi-\angle B,\quad\angle BPC=\pi-\angle C\]
	entonces $P$ es la intersección de circunferencias $\odot(D,DA),\odot(E,EB),\odot(F,FC)$ de donde
	\[\triangle PDE\equiv\triangle ADE,\quad\triangle PEF\equiv\triangle BEF,\quad\triangle PFD\equiv\triangle CFD.\]
	Luego,
	\[\cycsum\frac{DB}{DD'}=\cycsum\frac{\abs{DEBF}}{\abs{DEF}}=3+\frac{\abs{ADE}+\abs{BEF}+\abs{CFD}}{\abs{PDE}+\abs{PEF}+\abs{PFD}}=4.\]
\end{proof}

\begin{probEG}[ISL 2001/G6]
	Let $ABC$ be a triangle and $P$ an exterior point in the plane of the triangle. Suppose $AP,BP,CP$ meet the sides $BC,CA,AB$ (or extensions thereof) in $D,E,F$, respectively. Suppose further that the areas of triangles $PBD,PCE,PAF$ are all equal. Prove that each of these areas is equal to the area of triangle $ABC$ itself.
	\aops{119203}
\end{probEG}

\begin{proof}
	Sin pérdida de generalidad, supongamos que $P$ pertenece al lado opuesto de $A$ con respecto a $BC$. Si $C$ está entre $B$ y $D$, entonces $\triangle PBD$ contiene al $\triangle PCE$. Si $B$ está entre $C$ y $D$, entonces $\triangle PAF$ contiene al $\triangle PBD$. Por ende, $D$ está en el lado $BC$. Si $A$ está entre $E$ y $C$, entonces $\triangle PCE$ contiene al $\triangle PBD$. Si $B$ está entre $F$ y $A$, entonces $\triangle PAF$ contiene al $\triangle PBD$. Es decir, los segmentos dirigidos $\frac{PA}{PD},\frac{PB}{PE},\frac{PC}{PF}$ tienen signos $+,-,+$, respectivamente. Note que
	\[\frac{\abs{PBD}}{\abs{ABC}}=\frac{BD}{BC}\cdot\frac{PD}{AD}\quad\text{y}\quad\frac{BD}{BC}\cdot\frac{FC}{FP}\cdot\frac{AP}{AD}=1\]
	considerando áreas y segmentos con signos. Es decir,
	\[\left(\frac{PC}{PF}-1\right)\cdot\frac{PA}{PD}=\frac{\abs{ABC}}{\abs{PBD}}=k\]
	donde $k\in\RR^-$ es una constante. Con un poco de cálculo obtenemos
	\[(x(x-1)-k)(k+1)=0\]
	donde $x=\frac{PA}{PD}\ge 1$. Por lo tanto, $k=-1$.
\end{proof}

\begin{probEG}[ISL 2001/G7]
	Let $O$ be an interior point of acute triangle $ABC$. Let $A_1$ lie on $BC$ with $OA_1$ perpendicular to $BC$. Define $B_1$ on $CA$ and $C_1$ on $AB$ similarly. Prove that $O$ is the circumcenter of $ABC$ if and only if the perimeter of $A_1B_1C_1$ is not less than any one of the perimeters of $AB_1C_1$, $BC_1A_1$, and $CA_1B_1$.
	\aops{119204}
\end{probEG}

\begin{proof}
	Digamos que $O'\ne O$ es un punto interior de $\triangle ABC$, siendo $O$ el circuncentro de $ABC$. Si $O'$ pertenece en el interior del $\triangle BOC$, tenemos que
	\[\angle BC_1A_1=\angle BO'A_1=\frac{\pi}{2}-\angle A_1BO'\ge\frac{\pi}{2}-\angle CBO=\angle A\]
	y análogamente $\angle CB_1A_1\ge\angle A$, de donde $A_1B_1+A_1C_1\le AB_1+AC_1$. Es decir, $p(A_1B_1C_1)<p(AB_1C_1)$.
\end{proof}

\begin{probEG}[ISL 2001/G8]
	Let $ABC$ be a triangle with $\angle BAC=60\dg$. Let $AP$ bisect $\angle BAC$ and let $BQ$ bisect $\angle ABC$, with $P$ on $BC$ and $Q$ on $AC$. If $AB+BP=AQ+QB$, what are the angles of the triangle?
	\aops{119207}
\end{probEG}

\begin{proof}
	Sea $R$ un punto en la recta $AB$ tal que $BR=BP$ y $B$ está entre $R$ y $A$. Si $S$ es un punto del rayo $AC$ tal que $AR=AS$, entonces
	\[QS=AS-AQ=AR-AQ=AB+BR-AQ=AB+BP-AQ=QB\]
	y
	\[\angle QSP=\angle ASP=\angle ARP=\half\angle ABP=\angle QBP.\]
	Como $BPSQ$ no es un paralelogramo, $QP$ es bisectriz de $\angle BQC$. Es decir,
	\[\frac{BQ}{QC}=\frac{BP}{PC}=\frac{BA}{AC}=\frac{c}{b}\]
	de donde $BQ=\frac{ca}{c+a}$. Además, como $\angle A=60\dg$, tenemos que
	\[\left(\frac{ca}{c+a}\right)^2=BQ^2=AB^2+AQ^2-AB\cdot AQ=c^2+\left(\frac{bc}{c+a}\right)^2-\frac{bc^2}{c+a}\]
	y
	\[a^2=BC^2=AB^2+AC^2-AB\cdot AC=b^2+c^2-bc.\]
	Simplificando nos queda que $a=b-2c$, y de esto no es difícil ver que los ángulos $\angle A,\angle B,\angle C$ son $60\dg,105\dg,15\dg$.
\end{proof}

\note[Punto HM]{Jueves\\2022-05-19}

\begin{probEG}
	In $\triangle ABC$ with orthocenter $H$, the circle with diameter $\ol{AH}$ and $\odot(BHC)$ intersect again on the $A$-median at a point $X_A$.
\end{probEG}

\begin{proof}
	Si $A'$ es la reflexión de $A$ con respecto a $BC$, entonces $HA'$ es el diámetro de $\odot(BHC)$ de donde $X_A$ pertenece a $AA'$, que claramente es la $A$-mediana.
\end{proof}

\begin{probEG}[ELMO SL 2013/G3]
	In $\triangle ABC$, a point $D$ lies on line $BC$. The circumcircle of $ABD$ meets $AC$ at $F$ (other than $A$), and the circumcircle of $ADC$ meets $AB$ at $E$ (other than $A$). Prove that as $D$ varies, the circumcircle of $AEF$ always passes through a fixed point other than $A$, and that this point lies on the median from $A$ to $BC$.
	\aops{3151962}
\end{probEG}

\begin{proof}
	Si $P=BF\cap EC$ y $A'$ la reflexión de $A$ con respecto a $BC$, entonces $P$ pertenece a $\odot(AEF)$ y $\odot(BHC)$. Como $AA'$ pasa por $X_A$, $\angle AX_AP=\angle A'CP=\angle AEP$ de donde $X_A\in\odot(AEF)$.
\end{proof}

\begin{probEG}[ELMO 2014/5]
	Let $ABC$ be a triangle with circumcenter $O$ and orthocenter $H$. Let $\omega_1$ and $\omega_2$ denote the circumcircles of triangles $BOC$ and $BHC$, respectively. Suppose the circle with diameter $\ol{AO}$ intersects $\omega_1$ again at $M$, and line $AM$ intersects $\omega_1$ again at $X$. Similarly, suppose the circle with diameter $\ol{AH}$ intersects $\omega_2$ again at $N$, and line $AN$ intersects $\omega_2$ again at $Y$. Prove that lines $MN$ and $XY$ are parallel.
	\aops{3534946}
\end{probEG}

\begin{proof}
	Pista: con angulitos tenemos que $M$ es el punto medio de la cuerda $A$-simediana, $N$ es el punto $HM$ con respecto a $A$, $X$ es la intersección de las tangentes a $\odot(ABC)$ por $B$ y $C$, $Y$ es la reflexión de $A$ con respecto a $BC$.
\end{proof}

\begin{probEG}[USA TST 2005/6]
	Let $ABC$ be an acute scalene triangle with $O$ as its circumcenter. Point $P$ lies inside triangle $ABC$ with $\angle PAB=\angle PBC$ and $\angle PAC=\angle PCB$. Point $Q$ lies on line $BC$ with $QA=QP$. Prove that $\angle AQP=2\angle OQB$.
	\aops{734440}
\end{probEG}

\begin{probEG}[Brazil National Olympiad 2015/6]
	Let $\triangle ABC$ be a scalene triangle and $X$, $Y$ and $Z$ be points on the lines $BC$, $AC$ and $AB$, respectively, such that $\dang AXB=\dang BYC=\dang CZA$. The circumcircles of $BXZ$ and $CXY$ intersect at $P$. Prove that $P$ is on the circumference which diameter has ends in the orthocenter $H$ and in the barycenter $G$ of $\triangle ABC$.
	\aops{5469201}
\end{probEG}

\begin{proof}
	Se prueba que $P$ es el centro de roto-homotecia que manda $\triangle ABC\mapsto\triangle A_1B_1C_1$ donde $A_1=BY\cap CZ$ y análogamente para $B_1$ y $C_1$. Con esto se prueba que $\dang BAP=\dang CBP$ y $\dang CAP=\dang BCP$, de donde $P$ es el $HM$ punto con respecto a $A$, entonces $\angle HPA=\angle HPG=90\dg$.
\end{proof}

\begin{probEG}[Sharygin Final Round 2015/10.3]
	Let $A_1$, $B_1$ and $C_1$ be the midpoints of sides $BC$, $CA$ and $AB$ of triangle $ABC$, respectively. Points $B_2$ and $C_2$ are the midpoints of segments $BA_1$ and $CA_1$ respectively. Point $B_3$ is symmetric to $C_1$ with respect to $B$, and $C_3$ is symmetric to $B_1$ with respect to $C$.
Prove that one of common points of circles $BB_2B_3$ and $CC_2C_3$ lies on the circumcircle of triangle $ABC$.
	\aops{10667375}
\end{probEG}

\begin{proof}
	Considere al centro de roto-homotecia que manda $AB\mapsto B_1B_2$ y otra que manda $AC\mapsto C_1C_2$, y sea $P$ ese punto. Luego,
	\[\angle BB_3B_2=\angle(AB,B_1B_2)=\angle BPB_2\]
	de donde $P\in\odot(BB_2B_3)$ y análogamente $P\in\odot(CC_2C_3)$. Además,
	\[\angle PBB_2+\angle PCC_2=\angle PAB_1+\angle PAC_1=\angle A\]
	de donde $P\in\odot(ABC)$.
\end{proof}

\begin{problem}[ISL 2014/G6]
	Let $ABC$ be a fixed acute-angled triangle. Consider some points $E$ and $F$ lying on the sides $AC$ and $AB$, respectively, and let $M$ be the midpoint of $EF$. Let the perpendicular bisector of $EF$ intersect the line $BC$ at $K$, and let the perpendicular bisector of $MK$ intersect the lines $AC$ and $AB$ at $S$ and $T$, respectively. We call the pair $(E,F)$ \emph{interesting}, if the quadrilateral $KSAT$ is cyclic. Suppose that the pair $(E_1,F_1)$ and $(E_2,F_2)$ are interesting. Prove that
	\[\frac{E_1E_2}{AB}=\frac{F_1F_2}{AC}.\]
\end{problem}

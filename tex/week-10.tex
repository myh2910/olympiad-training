\section{Semana 10 (05/16 -- 05/22)}

\note[Álgebra]{Lunes\\2022-05-16}

\begin{probEG}[IMO Shortlist 2002 A2]
	Let $a_1,a_2,\dots$ be an infinite sequence of real numbers, for which there exists a real number $c$ with $0\le a_i\le c$ for all $i$, such that
	\[\abs{a_i-a_j}\ge\frac{1}{i+j}\quad\text{for all }i,j\text{ with }i\ne j.\]
	Prove that $c\ge 1$.
	\aops{118699}
\end{probEG}

\begin{probMR}[IMO Shortlist 2002 A3]
	Let $P$ be a cubic polynomial given by $P(x)=ax^3+bx^2+cx+d$, where $a,b,c,d$ are integers and $a\ne 0$. Suppose that $xP(x)=yP(y)$ for infinitely many pairs $x,y$ of integers with $x\ne y$. Prove that the equation $P(x)=0$ has an integer root.
	\aops{118702}
\end{probMR}

\begin{probMB}[IMO Shortlist 2002 A4]
	Find all functions $f$ from the reals to the reals such that
	\[\left(f(x)+f(z)\right)\left(f(y)+f(t)\right)=f(xy-zt)+f(xt+yz)\]
	for all real $x,y,z,t$.
	\aops{118703}
\end{probMB}

\note[Combinatoria]{Martes\\2022-05-17}

\begin{probEG}[IMO Shortlist 2002 C1]
	Let $n$ be a positive integer. Each point $(x,y)$ in the plane, where $x$ and $y$ are non-negative integers with $x+y<n$, is coloured red or blue, subject to the following condition: if a point $(x,y)$ is red, then so are all points $(x',y')$ with $x'\le x$ and $y'\le y$. Let $A$ be the number of ways to choose $n$ blue points with distinct $x$-coordinates, and let $B$ be the number of ways to choose $n$ blue points with distinct $y$-coordinates. Prove that $A=B$.
\end{probEG}

\begin{probEG}[IMO Shortlist 2002 C2]
	For $n$ an odd positive integer, the unit squares of an $n\times n$ chessboard are coloured alternately black and white, with the four corners coloured black. A \emph{tromino} is an $L$-shape formed by three connected unit squares. For which values of $n$ is it possible to cover all the black squares with non-overlapping trominos? When it is possible, what is the minimum number of trominos needed?
\end{probEG}

\begin{proof}
	Respuesta: $n\ge 7$ y el mínimo es $\left(\frac{n+1}{2}\right)^2$.
\end{proof}

\begin{probMR}[IMO Shortlist 2002 C3]
	Let $n$ be a positive integer. A sequence of $n$ positive integers (not necessarily distinct) is called \emph{full} if it satisfies the following condition: for each positive integer $k\ge 2$, if the number $k$ appears in the sequence then so does the number $k-1$, and moreover the first occurrence of $k-1$ comes before the last occurrence of $k$. For each $n$, how many full sequences are there?
\end{probMR}

\begin{proof}
	Respuesta: $n!$, considerando una biyección.
\end{proof}

\note[Teoría de Números]{}

\begin{probEG}
	Demuestre que existe un entero positivo $b$ que tiene la siguiente propiedad: para todo entero $n>b$, la suma de los dígitos del número $n!$ es mayor o igual que $10^{100}$.
\end{probEG}

\begin{proof}
	Note que si $n>b$, entonces $b\mid n!$. Además, $s(x-y)\ge s(x)-s(y)$ para todo $x,y\in\ZZ^+$. Sea $b=\underbrace{99\cdots 9}_{N\text{ veces}}$ donde $N\in\ZZ^+$ es suficientemente grande, y sea $M\cdot 10^k\cdot b$ un múltiplo de $b$ donde $10\nmid M$. Luego,
	\begin{align*}
		s(M\cdot 10^k\cdot b)
		&=s(M\cdot(10^N-1))\\
		&=s\left(\ol{(M-1)\underbrace{99\cdots 9}_{N\text{ veces}}}-(M-1)\right)\\
		&\ge s(M-1)+9N-s(M-1)\\
		&=9N
	\end{align*}
	de donde $s(n!)\ge 9N>10^{100}$ para todo $n>b$.
\end{proof}

\note[Geometría]{Miércoles\\2022-05-18}

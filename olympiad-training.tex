\documentclass[11pt]{scrartcl}

\usepackage[sexy,nothm]{evan}
\usepackage[english,spanish,es-nosectiondot,es-lcroman]{babel}
\usepackage{bold-extra}
\usepackage{ifthen}
\usepackage{tabularray}

\reversemarginpar
\addto\captionsspanish{\renewcommand{\proofname}{Solución}}
\newcommand{\note}[2][]{%
	\ifthenelse{\equal{#1}{}}%
		{\todo{#2}}%
		{\todo{#2\\[-6pt]\rule{\textwidth}{0.3pt}\vspace{-1em}\begin{flushright}#1\end{flushright}}}%
}
\SetTblrInner{hlines,vlines,rows={16pt,rowsep=0pt},columns={16pt,colsep=0pt}}

\DeclareMathOperator{\mcd}{mcd}
\DeclareMathOperator{\mcm}{mcm}

\newcommand{\aops}[1]{%
	\par\vspace{1pt}\hspace{\fill}%
	\href{https://artofproblemsolving.com/community/p#1}{[AoPS:#1]}%
}
\newcommand{\customgreen}[1]{\bgroup\color{green!40!black}#1\egroup}
\newcommand{\customblue}[1]{\bgroup\color{blue!70!black}#1\egroup}
\newcommand{\customred}[1]{\bgroup\color{red!80!black}#1\egroup}
\newcommand{\clubG}[1]{\customgreen{[\texttt{#1}]}}
\newcommand{\clubB}[1]{\customblue{[\texttt{#1}]}}
\newcommand{\clubR}[1]{\customred{[\texttt{#1}]}}

\newenvironment{probSM}{\hrulebar\color{Maroon}\small}{}

\theoremstyle{definition}
\declaretheorem[style=thmbluebox,name=Teorema,numberwithin=section]{theorem}
\declaretheorem[name=Problema,sibling=theorem]{problem}

\newtheorem{probEG}[problem]{\clubG{E} Problema}
\newtheorem{probMG}[problem]{\clubG{M} Problema}
\newtheorem{probHG}[problem]{\clubG{H} Problema}
\newtheorem{probZG}[problem]{\clubG{Z} Problema}

\newtheorem{probEB}[problem]{\clubB{E} Problema}
\newtheorem{probMB}[problem]{\clubB{M} Problema}
\newtheorem{probHB}[problem]{\clubB{H} Problema}
\newtheorem{probZB}[problem]{\clubB{Z} Problema}

\newtheorem{probER}[problem]{\clubR{E} Problema}
\newtheorem{probMR}[problem]{\clubR{M} Problema}
\newtheorem{probHR}[problem]{\clubR{H} Problema}
\newtheorem{probZR}[problem]{\clubR{Z} Problema}


\title{Entrenamiento para la IMO}
\author{Yohan Min}

\begin{document}

\maketitle

\begin{abstract}
	Esta es una colección de problemas de Olimpiadas Matemáticas y sus soluciones, donde la mayoría de ellas fueron resueltas en las clases de mi profesor \href{https://www.facebook.com/emerson.sorianoperez}{Emerson Soriano}. Las letras \verb|E| (easy), \verb|M| (medium) y \verb|H| (hard) indican los niveles de los problemas, y sus colores indican lo siguiente:
	\begin{description}[labelwidth=\widthof{\texttt{aaaaa}}+1.2em,leftmargin=\widthof{\texttt{aaaaa}a}+1.2em,align=right]
		\ii[\cgreen{verde}] Resolví el problema solo.
		\ii[\cblue{azul}] Resolví el problema usando una pista, o me tomó mucho tiempo para resolverlo.
		\ii[\cred{rojo}] No pude resolver el problema y vi su solución.
		\ii[\texttt{negro}] Aún no resolví el problema porque no lo intenté mucho.
	\end{description}
\end{abstract}

\tableofcontents

\section{Semana 1 (03/14 -- 03/20)}

\note{Miércoles\\2022-03-16}

\begin{probEG}[IMO Shortlist 2006 A5]
	If $a,b,c$ are the sides of a triangle, prove that
	\[\frac{\sqrt{b+c-a}}{\sqrt b+\sqrt c-\sqrt a}+\frac{\sqrt{c+a-b}}{\sqrt c+\sqrt a-\sqrt b}+\frac{\sqrt{a+b-c}}{\sqrt a+\sqrt b-\sqrt c}\le 3.\]
\end{probEG}

\begin{proof}
	Sea $x=\sqrt b+\sqrt c-\sqrt a>0$ y análogamente se definen $y$ y $z$. Por la desigualdad de Schur tenemos que
	\[\cycsum\frac{(x-y)(x-z)}{x^2}\ge 0.\]
	Note que
	\[b+c-a=\left(\frac{x+y}{2}\right)^2+\left(\frac{x+z}{2}\right)^2-\left(\frac{y+z}{2}\right)^2=\frac{x^2+xy+xz-yz}{2}\]
	de donde
	\begin{align*}
		\frac13\left(\cycsum\frac{\sqrt{b+c-a}}{\sqrt b+\sqrt c-\sqrt a}\right)^2
		&\le\cycsum\left(\frac{\sqrt{b+c-a}}{\sqrt b+\sqrt c-\sqrt a}\right)^2\\
		&=\frac{x^2+xy+xz-yz}{2x^2}\\
		&=3-\cycsum\frac{(x-y)(x-z)}{2x^2}\\
		&\le 3
	\end{align*}
	y esto resuelve el problema.
\end{proof}

\begin{probEB}[IMO Shortlist 2006 N7]
	Demuestre que para todo entero positivo $n$, existe un entero positivo $m$ tal que $n\mid 2^m+m$.
	\begin{hint}
		For all positive integers $n$, show that there exists a positive integer $m$ such that $n$ divides $2^m+m$.
	\end{hint}
	\aops{867486}
\end{probEB}

\begin{proof}
	El problema es trivial cuando $n$ es una potencia de $2$. Ahora, supongamos que para algún $n\in\ZZ^+$ existe un $m\in\ZZ^+$ tal que $n\mid 2^m+m$. En efecto, sea $k=\frac{2^m+m}{n}$ y sea $p$ un primo impar arbitrario mayor o igual que todos los factores primos de $n$. Sea $n=p^en_0$ donde $p\nmid n_0$. Sea $m_0=m+\phi(p^{e+1}n_0)t$ para algún $n_0\mid t$ tal que $n_0k\equiv\phi(n_0)t\pmod p$. Luego,
	\begin{align*}
		2^{m_0}+m_0
		&\equiv 2^m+m+p^e(p-1)\phi(n_0)t\\
		&=p^e\left(n_0k+(p-1)\phi(n_0)t\right)\\
		&\equiv 0\pmod{p^{e+1}n_0}
	\end{align*}
	y aquí terminamos por inducción.
\end{proof}

\begin{probMR}[IMO Shortlist 2007 A4]
	Determine todas las funciones $f:\RR^+\to\RR^+$ tales que $f(x+f(y))=f(x+y)+f(y)$ para todo $x,y\in\RR^+$.
	\begin{hint}
		Find all functions $f:\RR^+\to\RR^+$ satisfying $f(x+f(y))=f(x+y)+f(y)$ for all pairs of positive reals $x$ and $y$.
	\end{hint}
	\aops{1165901}
\end{probMR}

\begin{probMR}[IMO Shortlist 2007 N3]
	Sea $X$ un conjunto de $10000$ enteros no divisibles por $47$. Demuestre que existe $Y\subset X$ con $\abs{Y}=2007$, tal que $47\nmid a-b+c-d+e$ para todo $a,b,c,d,e\in Y$.
	\begin{hint}
		Let $X$ be a set of $10000$ integers, none of them is divisible by $47$. Prove that there exists a $2007$-element subset $Y$ of $X$ such that $a-b+c-d+e$ is not divisible by $47$ for any $a,b,c,d,e\in Y$.
	\end{hint}
	\aops{1187204}
\end{probMR}

\begin{probEG}[Iran MO 2000 3rd Round]
	La secuencia $(c_i)_{i\ge 1}$ de enteros positivos satisface la siguiente condición: para todo $m,n\in\ZZ^+$ con $1\le m\le c_1+c_2+\dots+c_n$, existen enteros positivos $a_1,a_2,\dots,a_n$ tales que
	\[m=\frac{c_1}{a_1}+\frac{c_2}{a_2}+\dots+\frac{c_n}{a_n}.\]
	Para cada índice $i$, hallar el mayor valor de $c_i$.
	\begin{hint}
		A sequence of natural numbers $c_1,c_2,\dots$ is called \emph{perfect} if every natural number $m$ with $1\le m\le c_1+c_2+\dots+c_n$ can be represented as
		\[m=\frac{c_1}{a_1}+\frac{c_2}{a_2}+\dots+\frac{c_n}{a_n}.\]
		Given $n$, find the maximum possible value of $c_n$ in a perfect sequence $(c_i)$.
	\end{hint}
	\aops{389506}
\end{probEG}

\begin{probMR}[IMO Shortlist 2008 N3]
	Sea $(a_i)_{i\ge 0}$ una secuencia de enteros positivos tal que $\mcd(a_i,a_{i+1})>a_{i-1}$ para todo $i\ge 1$. Demuestre que $a_n\ge 2^n$ para todo $n\ge 0$.
	\begin{hint}
		Let $a_0,a_1,a_2,\dots$ be a sequence of positive integers such that $\gcd(a_i,a_{i+1})>a_{i-1}$. Prove that $a_n\ge 2^n$ for all $n\ge 0$.
	\end{hint}
	\aops{1555931}
\end{probMR}

\note{Viernes\\2022-03-18}

\begin{probEG}[IMO Shortlist 2018 A1]
	Determine todas las funciones $f:\QQ^+\to\QQ^+$ tales que
	\[f(x^2f(y)^2)=f(x)^2f(y)\]
	para todo $x,y\in\QQ^+$.
	\begin{hint}
		Determine all functions $f:\QQ^+\to\QQ^+$ satisfying
		\[f(x^2f(y)^2)=f(x)^2f(y)\]
		for all $x,y\in\QQ^+$.
	\end{hint}
	\aops{12752810}
\end{probEG}

\begin{proof}
	Si $x^2f(y)^2=y$ entonces $f(x)=1$. Si $f(y)=1$ tenemos que $f(x^2)=f(x)^2$ para todo $x\in\QQ^+$. Es decir, $f(xf(y))^2=f(x)^2f(y)$ de donde existe una función $f_1:\QQ^+\to\QQ^+$ tal que $f(x)=f_1(x)^2$. Luego, $f_1(xf(y))^2=f_1(x)^2f_1(y)$ de donde existe una función $f_2:\QQ^+\to\QQ^+$ tal que $f_1(x)=f_2(x)^2$. Luego, $f_2(xf(y))^2=f_2(x)^2f_2(y)$ y así sucesivamente. Por lo tanto, $f(x)\equiv 1$.
\end{proof}

\begin{probMR}[IMO Shortlist 2018 A2]
	Determine todos los enteros $n\ge 3$ para los cuales existen números reales $a_1,a_2,\dots,a_{n+2}$ donde $a_{n+1}=a_1$ y $a_{n+2}=a_2$, tales que
	\[a_ia_{i+1}+1=a_{i+2}\]
	para todo $i=1,2,\dots,n$.
	\begin{hint}
		Find all integers $n\ge 3$ for which there exist real numbers $a_1,a_2,\dots,a_n,a_{n+1}=a_1,a_{n+2}=a_2$ such that
		\[a_ia_{i+1}+1=a_{i+2}\]
		for all $i=1,2,\dots,n$.
	\end{hint}
	\aops{10626524}
\end{probMR}

\begin{proof}
	Respuesta: múltiplos de $3$.
\end{proof}

\begin{probMB}[IMO Shortlist 2018 A3]
	Dado un conjunto $S$ de enteros positivos, demuestre que al menos una de las siguientes dos proposiciones es verdadera:
	\begin{enumerate}[(1)]
		\ii \label{enumi:finite_subsets} Existen dos subconjuntos finitos y distintos $F$ y $G$ de $S$ tales que $\sum_{x\in F}1/x=\sum_{x\in G}1/x$;
		\ii \label{enumi:rational_number} Existe un número racional positivo $r<1$ tal que $\sum_{x\in F}1/x\ne r$ para todo subconjunto finito $F$ de $S$.
	\end{enumerate}
	\begin{hint}
		Given any set $S$ of positive integers, show that at least one of the following two assertions holds:
		\begin{enumerate}[(1)]
			\ii There exist distinct finite subsets $F$ and $G$ of $S$ such that $\sum_{x\in F}1/x=\sum_{x\in G}1/x$;
			\ii There exists a positive rational number $r<1$ such that $\sum_{x\in F}1/x\ne r$ for all finite subsets $F$ of $S$.
		\end{enumerate}
	\end{hint}
\end{probMB}

\begin{proof}
	Si $S$ es un conjunto finito, es claro que la \ref{enumi:rational_number} es verdadera. Ahora, supongamos que $S$ es infinito y la \ref{enumi:rational_number} no se cumple. Sean $a_1<a_2<\cdots$ los elementos de $S$. Sea $i\ge 1$ un índice cualquiera. Si $\frac{1}{a_i}\le\frac{1}{a_{i+1}}+\frac{1}{a_{i+2}}+\dots+\frac{1}{a_j}<\frac{2}{a_i}$, sea $r=\frac{1}{a_{i+1}}+\frac{1}{a_{i+2}}+\dots+\frac{1}{a_j}-\frac{1}{a_i}<\frac{1}{a_i}\le 1$. Es decir, $r=\frac{1}{a_{x_1}}+\dots+\frac{1}{a_{x_k}}<\frac{1}{a_i}$ de donde
	\[\frac{1}{a_i}+\frac{1}{a_{x_1}}+\dots+\frac{1}{a_{x_k}}=\frac{1}{a_i}+r=\frac{1}{a_{i+1}}+\frac{1}{a_{i+2}}+\dots+\frac{1}{a_j}\]
	y la \ref{enumi:finite_subsets} es verdadera. Ahora, supongamos que
	\[\frac{1}{a_{i+1}}+\frac{1}{a_{i+2}}+\dots+\frac{1}{a_j}<\frac{1}{a_i}\]
	para todo $j>i>0$. Si $a_{i+1}>2a_i$, sea $\frac{1}{a_{i+1}}<r=\frac{2}{a_{i+1}}<\frac{1}{a_i}\le 1$. Es decir, $r=\frac{1}{a_{x_1}}+\dots+\frac{1}{a_{x_k}}<\frac{1}{a_i}$. Si $x_1,\dots,x_k>i+1$ entonces $r<\frac{1}{a_{i+1}}$ lo cual es una contradicción. Por ende, $r=\frac{1}{a_{i+1}}+\frac{1}{a_{y_1}}+\dots+\frac{1}{a_{y_l}}<\frac{2}{a_{i+1}}$ lo cual es un absurdo. Por lo tanto, $a_{i+1}\le 2a_i$ para todo $i\in\ZZ^+$. Luego, \[\frac{1}{a_{i+1}}\left(2-\frac{1}{2^{j-i-1}}\right)=\sum_{k=0}^{j-i-1}\frac{1}{2^ka_{i+1}}\le\frac{1}{a_{i+1}}+\dots+\frac{1}{a_j}<\frac{1}{a_i}\]
	de donde $a_{i+1}\ge 2a_i$ para todo $i\in\ZZ^+$. Es decir, $a_{i+1}=2a_i$ para todo $i\in\ZZ^+$. Si $r=\frac{1}{3a_1}<\frac{1}{a_1}<1$, tenemos que
	\[\frac{1}{3}=\frac{a_1}{a_{x_1}}+\dots+\frac{a_1}{a_{x_k}}=\frac{1}{2^{x_1-1}}+\dots+\frac{1}{2^{x_k-1}}=\frac{M}{N}\]
	donde $M$ es impar y $N$ es par, lo cual es un absurdo.
\end{proof}

\begin{problem}[IMO Shortlist 2018 A4]
	Sea $(a_i)_{i\ge 0}$ una secuencia de números reales tal que $a_0=0$, $a_1=1$, y para todo $n\ge 2$ existe un $1\le k\le n$ tal que
	\[a_n=\frac{a_{n-1}+\dots+a_{n-k}}{k}.\]
	Determine el máximo valor de $a_{2018}-a_{2017}$.
	\begin{hint}
		Let $a_0,a_1,a_2,\dots$ be a sequence of real numbers such that $a_0=0$, $a_1=1$, and for every $n\ge 2$ there exists $1\le k\le n$ satisfying
		\[a_n=\frac{a_{n-1}+\dots+a_{n-k}}{k}.\]
		Find the maximal possible value of $a_{2018}-a_{2017}$.
	\end{hint}
\end{problem}

\begin{proof}
	La respuesta es $\frac{2016}{2017^2}$, cuando $a_1=a_2=\dots=a_{2016}=1$ y $(a_{2017},a_{2018})=\left(\frac{2016}{2017},1-\frac{1}{2017^2}\right)$.
\end{proof}

\section{Week 2 (03/21 -- 03/27)}

\begin{problem}[IMO Shortlist 2007 N5]
	Find all surjective functions $f:\NN\to\NN$ such that for every $m,n\in\NN$ and every prime $p$, the number $f(m+n)$ is divisible by $p$ if and only if $f(m)+f(n)$ is divisible by $p$.
\end{problem}

\begin{problem}[IMO Shortlist 2007 N6]
	Let $k$ be a positive integer. Prove that the number $(4\cdot k^2-1)^2$ has a positive divisor of the form $8kn-1$ if and only if $k$ is even.
\end{problem}

\begin{probER}[Petrozavodsk Winter 2021, UPC Contest/B]
	We say that a permutation $a_0,\dots,a_{n-1}$ of $0,\dots,n-1$ is \emph{beautiful} if the sequence $b_0,\dots,b_{n-1}$ given by $b_i=\abs{a_i-i}$ is also a permutation of $0,\dots,n-1$.

	Determine for which $n$ there exists a beautiful permutation of $0,\dots,n-1$.
	\aops{21584017}
\end{probER}

\begin{proof}
	Note que
	\[\frac{n(n-1)}{2}=\sum_{i=0}^{n-1}\abs{a_i-i}\equiv\sum_{i=0}^{n-1}a_i-\sum_{i=0}^{n-1}i=0\pmod 2\]
	de donde $n=4k$ o $n=4k+1$. Ahora, si $n=4k+c$ donde $c\in\{0,1\}$ sea $a_k=k$. Considerando el resto de los números, sea $(C)$ un ciclo que alterna entre el mayor y el menor de los números que sobran. Por ejemplo, si $n=4\times 2=8$, $(C)=(7\;0\;6\;1\;5\;3\;4)$. Note que el conjunto de las diferencias consecutivas de $(C)$ es $1,2,\dots,n-1$ con ningún número repetido. Entonces, podemos definir $a=(k)(C)$ y esta satisface lo requerido.
\end{proof}

\begin{problem}[IMO Shortlist 2007 N4]
	For every integer $k\ge 2$, prove that $2^{3k}$ divides the number
	\[\binom{2^{k+1}}{2^k}-\binom{2^k}{2^{k-1}}\]
	but $2^{3k+1}$ does not.
\end{problem}

\begin{problem}[IMO Shortlist 2008 A6]
	Let $f:\RR\to\NN$ be a function which satisfies
	\[f\left(x+\frac{1}{f(y)}\right)=f\left(y+\frac{1}{f(x)}\right)\]
	for all $x,y\in\RR$. Prove that there is a positive integer which is not a value of $f$.
\end{problem}

\begin{probEG}[RMM Shortlist 2018 N1]
	Determine all polynomials $f$ with integer coefficients such that $f(p)$ is a divisor of $2^p-2$ for every odd prime $p$.
	\aops{11822580}
\end{probEG}

\begin{proof}
	Sea $f(x)=x^n\cdot g(x)$ donde $g(0)\ne 0$.
	\begin{itemize}
		\ii Si $g(x)=c$ es una función constante, tenemos que $3^n\cdot c\mid 2^3-2=6$. Si $n=0$, tenemos que $c\mid 6$. Si $n=1$, tenemos que $c\mid 2$. Por lo tanto, $f(x)=\pm 1,\pm 2,\pm 3,\pm 6,\pm x,\pm 2x$.
		\ii Si $g$ no es constante, por Schur existe un primo $q>3$ suficientemente grande tal que $q\nmid g(0)$ y $q\mid g(m)$ para algún $m\in\ZZ^+$. Como $m\mid g(m)-g(0)$, es claro que $q\nmid m$. Como $q(q-1)$ y $m(q-1)+q$ son coprimos, por Dirichlet existe un primo impar $p$ de la forma $q(q-1)k-(m(q-1)+q)\equiv m\pmod q$. Es decir, $q\mid g(p)\mid 2^p-2$ donde $\mcd(p-1,q-1)=\mcd(q-1,2)=2$. Por ende, $q\mid\mcd(2^{p-1}-1,2^{q-1}-1)=2^2-1=3$, lo cual es un absurdo.
	\end{itemize}
\end{proof}

\begin{probEG}[All-Russian Olympiad 1998/9.8]
	Two distinct positive integers $a,b$ are written on the board. The smaller of them is erased and replaced with the number $\frac{ab}{\abs{a-b}}$. This process is repeated as long as the two numbers are not equal. Prove that eventually the two numbers on the board will be equal.
	\aops{2621673}
\end{probEG}

\begin{proof}
	Digamos que $a>b$. Si $a=bq+r$ donde $0\le r<b$, tenemos que
	\[(a,b)\to\left(a,\frac{ab}{a-b}\right)\to\dots\to\left(a,\frac{ab}{a-bq}\right)=\left(\frac{a}{r}\cdot r,\frac{a}{r}\cdot b\right)\]
	y así podemos continuar con la pareja $(b,r)$ hasta obtener la pareja $(d,d)$ (multiplicado por algún número racional) siendo $d=\mcd(a,b)$.
\end{proof}

\begin{probEG}
	En una fila, que tiene la forma de un tablero infinito (en ambas direcciones), hay varios caramelos. Un \emph{movimiento} consiste en elegir una casilla que contenga al menos cuatro caramelos, luego, se extrae cuatro caramelos de esa casilla y se coloca $2$ caramelos en la casilla anterior y $2$ caramelos en la casilla posterior. ¿Es posible que después de un número finito de movimientos se pueda regresar a la configuración inicial?
\end{probEG}

\begin{proof}
	La respuesta es no. Sea $X$ la casilla que está más a la izquierda tal que el número de caramelos en ella es mayor que el número inicial. En cada operación, $X$ no se mueve o se mueve más a la izquierda. Por lo tanto, siempre existe una casilla con el número de caramelos mayor que el inicial.
\end{proof}

\begin{probEG}
	Inicialmente hay $2022$ osos de peluche repartidos aleatoriamente en $127$ cajas. Un \emph{movimiento} consiste en elegir una caja que no contenga a todos los osos de peluche, retirar un oso de peluche, y colocarlo en otra caja cuyo número de peluches sea mayor o igual que el de la caja elegida. Demuestre que eventualmente todos los osos de peluches estarán en una misma caja.
\end{probEG}

\begin{proof}
	Sea $C$ el conjunto de los números de osos de peluche (mayores que $0$) en cada caja y sea $P$ el producto de todos los elementos de $C$. En cada movimiento, elegimos dos cajas con números de osos $a\le b$ y tendremos $(a,b)\to(a-1,b+1)$. Como $(a-1)(b+1)<ab$, el producto $P$ o el cardinal $\abs{C}$ disminuye, por lo que al final siempre tendremos una sola caja conteniendo todos los osos de peluche.
\end{proof}

\begin{probMR}[IMO Shortlist 2005 C5]
	There are $n$ markers, each with one side white and the other side black. In the beginning, these $n$ markers are aligned in a row so that their white sides are all up. In each step, if possible, we choose a marker whose white side is up (but not one of the outermost markers), remove it, and reverse the closest marker to the left of it and also reverse the closest marker to the right of it. Prove that, by a finite sequence of such steps, one can achieve a state with only two markers remaining if and only if $n-1$ is not divisible by $3$.
\end{probMR}

\begin{probER}
	En $n-1$ casillas de un tablero de $n\times n$ se ha escrito el número $1$ y en las casillas restantes se ha escrito el número cero. Un \emph{movimiento} consiste en elegir una casilla, restarle $1$ al número escrito en ella y sumarle $1$ a todos los números que están en casillas de su misma fila y a todos los números que están en casillas de su misma columna. ¿Es posible, luego de algunos movimientos, obtener un tablero cuyos números son todos iguales?
\end{probER}

\begin{probMG}[IberoAmerican 1995/5]
	The incircle of a triangle $ABC$ touches the sides $BC$, $CA$, $AB$ at the points $D$, $E$, $F$ respectively. Let the line $AD$ intersect this incircle of triangle $ABC$ at a point $X$ (apart from $D$). Assume that this point $X$ is the midpoint of the segment $AD$, this means, $AX=XD$. Let the line $BX$ meet the incircle of triangle $ABC$ at a point $Y$ (apart from $X$), and let the line $CX$ meet the incircle of triangle $ABC$ at a point $Z$ (apart from $X$). Show that $EY=FZ$.
	\aops{15699}
\end{probMG}

\begin{proof}
	Sea $P$ el punto medio de $FD$. Como $XY$ es simediana de $\triangle XFD$, tenemos que
	\[\angle BFY=\angle FXY=\angle PXD=\angle FAD\]
	de donde $FY\parallel AD$ y análogamente $EZ\parallel AD$. Por lo tanto, $FY\parallel EZ$ de donde $EY=FZ$.
\end{proof}

\begin{probEB}[Peru EGMO TST 2022/1]
	En cada casilla de un tablero $4\times 4$ se escribe un número entero positivo de modo que el número escrito en cada casilla es igual a la cantidad de números distintos entre sí escritos en sus casillas vecinas. ¿Cuántos números distintos como máximo pueden haber en el tablero? \\[4pt]
	\emph{Nota: Casillas vecinas son las que comparten un lado en común o un vértice en común.}
\end{probEB}

\begin{proof}
	Supongamos que hay $n$ números distintos en el tablero, donde $M$ es el máximo de ellos. Luego, $M=n$ y $1,2,\dots,n$ son los números que están en el tablero. Si $n\ge 3$, existe un $1$ que es vecino de $n$. Considerando la posición del $1$, es fácil ver que $1$ está en una esquina del tablero. Es decir, tenemos lo siguiente.

	\[\begin{tblr}{cccc}
		1 & n & & \\
		n & n & & \\
		& & & \\
		& & & \\
	\end{tblr}\]

	Es fácil ver que $n\in\{3,4\}$, pero analizando por casos tenemos un absurdo. Por ende, la máxima cantidad de números distintos en el tablero es $2$. Como ejemplo considere el siguiente tablero.

	\[\begin{tblr}{cccc}
		1 & 2 & 2 & 1 \\
		2 & 2 & 2 & 2 \\
		2 & 2 & 2 & 2 \\
		1 & 2 & 2 & 1 \\
	\end{tblr}\]
\end{proof}

\begin{probEG}[Peru EGMO TST 2022/2]
	Encuentre todos los enteros positivos $a$, $b$ y $c$ tales que
	\[a^{b!}+b^{a!}=c^{c!}.\]
\end{probEG}

\begin{proof}
	Si $a=1$ o $b=1$, tenemos que $(a,b,c)=\left(1,c^{c!}-1,c\right),\left(c^{c!}-1,1,c\right)$ para algún valor de $c>1$. Ahora, si $a,b\ge 2$ tenemos que $c\ge 2$. Note que si $a=b$, tenemos que $2a^{a!}=c^{c!}$ lo cual es un absurdo pues $a!$ y $c!$ son exponentes pares. Sin pérdida de generalidad, supongamos que $a>b\ge 2$. Luego, $c\ge 3$. Si $b\ge 3$, los exponentes $b!$, $a!$ y $c!$ son múltiplos de $3$, lo cual es un absurdo por el último teorema de Fermat. Por ende, $b=2$. Si $a=3$, tenemos que $73=c^{c!}$ lo cual es un absurdo. Por ende, $a\ge 4$. Si $c=3$, $1024=2^{10}<2^{a!}<c^{c!}=729$, lo cual es un absurdo. Por ende, $c\ge 4$. Luego, los exponentes $a!$ y $c!$ son múltiplos de $4$, de donde tenemos una ecuación de la forma $x^2=y^4-z^4$, lo cual es un absurdo.
\end{proof}

\begin{probEG}[Peru EGMO TST 2022/3]
	Los puntos $A_2,B_2,C_2$ son los puntos medios de las alturas $AA_1,BB_1,CC_1$ del triángulo acutángulo $ABC$. Demuestre que la suma de los ángulos $\angle B_2A_1C_2$, $\angle C_2B_1A_2$ y $\angle A_2C_1B_2$ es $180\dg$.
\end{probEG}

\begin{proof}
	Sea $H$ el ortocentro del triángulo $ABC$. Si $M$ es el punto medio del lado $BC$, tenemos que $\angle HB_2M=\angle HC_2M=\angle HA_1M=90\dg$, de donde $\angle B_2A_1C_2=\angle B_2HC_2$ y análogamente $\angle C_2B_1A_2=\angle C_2HA_2$ y $\angle A_2C_1B_2=\angle A_2HB_2$. Aquí se termina la prueba.
\end{proof}

\begin{probMB}[Peru EGMO TST 2022/4]
	Encuentre todos los enteros positivos $m$, $n$ y $a$ tales que
	\[\abs{(a^n+1)^n-a^m}\le a^2.\]
\end{probMB}

\begin{proof}
	Supongamos que $n=1$. Como $a=1$ cumple, supongamos que $a>1$. Si $m=1$, es trivial. Si $m=2$, $0<a^2-a-1<a^2$. Si $m\ge 3$,
	\[a^m-a-1\ge a^3-a-1=(a-1)^2(a+1)-2+a^2>a^2.\]
	Ahora, supongamos que $n>1$. Note que si $a=1$ tenemos $2^n-1>1$, de donde $a>1$. Si $m\le n^2$, tenemos que
	\[(a^n+1)^n-a^m\ge \left(a^{n^2}-a^m\right)+a^n+1>a^2\]
	de donde $m\ge n^2+1$. Si $n=2$, tenemos que
	\[a^{n^2+1}-(a^n+1)^n=(a-1)(a^4-3a-3)-4+a^2>a^2.\]
	Si $n\ge 3$, note que
	\begin{align*}
		a^{n^2+1}-(a^n+1)^n
		&\ge a^{n^2+1}-\left(a^{n^2}+\sum_{i=0}^{n-1}\binom{n}{i}a^{n(n-1)}\right) \\
		&=a^{n^2+1}-a^{n^2}-(2^n-1)a^{n(n-1)} \\
		&=a^{n(n-1)}\left(a^n(a-1)-2^n+1\right) \\
		&>a^2
	\end{align*}
	de donde $a^m-(a^n+1)^n\ge a^{n^2+1}-(a^n+1)^n>a^2$. Finalmente, concluimos que las ternas $(m,n,a)=(m,1,1),(1,1,a),(2,1,a)$ cumplen, donde $m\ge 1$ y $a>1$ son enteros positivos.
\end{proof}

\begin{probEG}[JBMO Shortlist 2016 C4]
	A splitting of a planar polygon is a finite set of triangles whose interiors are pairwise disjoint, and whose union is the polygon in question. Given an integer $n\ge 3$, determine the largest integer $m$ such that no planar $n$-gon splits into less than $m$ triangles.
	\aops{9180563}
\end{probEG}

\begin{proof}
	La respuesta es $m=\ceiling{n/3}$. La figura \ref{fig:splitting_hexagon} muestra un ejemplo con $n=6$.

	\begin{figure}[ht!]
		\centering
		\begin{asy}
			size(4cm);
			pair A = dir(-30);
			pair B = dir(90);
			pair C = dir(210);
			pair D = (2 * C + A) / 3;
			pair E = dir(-90);
			pair F = (C + 2 * A) / 3;

			filldraw(A--B--C--cycle, paleblue, blue);
			filldraw(D--E--F--cycle, pink, red);

			dot(A);
			dot(B);
			dot(C);
			dot(D);
			dot(E);
			dot(F);
		\end{asy}
		\caption{Hexágono cóncavo cuvierto por $2$ triángulos.}
		\label{fig:splitting_hexagon}
	\end{figure}

	En este caso, el mínimo número requerido de triángulos es claramente $2$. Generalmente, es posible construir un $n$-ágono compuesto por $\ceiling{n/3}$ triángulos. Ahora, si $t$ es el número de triángulos, tenemos que probar que $n\le 3t$. Pero claramente, cada vértice del polígono corresponde a al menos un vértice de un triángulo, y cada triángulo tiene $3$ vértices. Con esto se termina la prueba.
\end{proof}

\begin{probEB}[JBMO Shortlist 2016 N4]
	Find all triples of integers $(a,b,c)$ such that the number
	\[N=\frac{(a-b)(b-c)(c-a)}{2}+2\]
	is a power of $2016$.
	\aops{6565545}
\end{probEB}

\begin{proof}
	Sea $N=2016^n$ donde $n\in\ZZ^+_0$. Si $n=0$, $(a-b)(b-c)(c-a)=-2$ de donde $(a,b,c)=(k+2,k+1,k),(k+1,k,k+2),(k,k+2,k+1)$ para algún $k\in\ZZ$. Si $n\ge 1$, nos va a quedar una ecuación como
	\[(x+y)xy=2(2016^n-2)\equiv -4\pmod 9\]
	donde $x\equiv y\pmod 3$. Analizando por casos sale un absurdo.
\end{proof}

\begin{probMB}[IberoAmerican 2021/3]
	Sea $a_1,a_2,a_3,\dots$ una sucesión de enteros positivos y sea $b_1,b_2,b_3,\dots$ la sucesión de números reales dada por
	\[b_n=\frac{a_1a_2\cdots a_n}{a_1+a_2+\dots+a_n},\quad\text{para }n\ge 1.\]
	Demuestre que si entre cada millón de términos consecutivos de la sucesión $b_1,b_2,b_3,\dots$ existe al menos uno que es entero, entonces existe algún $k$ tal que $b_k>2021^{2021}$.
	\aops{23437726}
\end{probMB}

\begin{proof}
	Supongamos que existe un entero $M$ tal que $b_i\le M$ para todo $i\ge 1$. Sea $i_0>10^6\cdot M$ un índice tal que $b_{i_0}$ es entero. Luego, existe un índice $i_1>i_0$ tal que $i_1\le i_0+10^6$ y $b_{i_1}$ es entero. Sea $t$ el número de índices $i_0<i\le i_1$ tales que $a_i>1$. Si $t=0$, claramente $b_{i_0}>b_{i_1}$ de donde
	\[\frac{a_1a_2\cdots a_{i_0}(i_1-i_0)}{(a_1+a_2+\dots+a_{i_0})(a_1+a_2+\dots+a_{i_1})}=b_{i_0}-b_{i_1}\ge 1\]
	y
	\[b_{i_0}=\frac{a_1a_2\cdots a_{i_0}}{a_1+a_2+\dots+a_{i_0}}\ge\frac{a_1+a_2+\dots+a_{i_1}}{i_1-i_0}\ge\frac{i_1}{10^6}>M\]
	lo cual es una contradicción. Por ende, $t\ge 1=\frac{10^6}{10^6}>\frac{i_1-i_0}{i_0-1}$ de donde $2^t-1\ge t>\frac{2t+(i_1-i_0-t)}{i_0}$. Sea $P=a_{i_0+1}a_{i_0+2}\cdots a_{i_1}\ge 2^t$. Luego,
	\begin{align*}
		1
		&>\frac{1}{2^t}+\frac{1}{i_0}\left(\frac{t}{2^{t-1}}+\frac{i_1-i_0-t}{2^t}\right) \\
		&\ge\frac{1}{P}+\frac{1}{a_1+a_2+\dots+a_{i_0}}\left(\frac{a_{i_0+1}+a_{i_0+2}+\dots+a_{i_1}}{P}\right) \\
		&=\frac{b_{i_0}}{b_{i_1}}
	\end{align*}
	de donde $b_{i_1}>b_{i_0}$. Por lo tanto, existe una secuencia estrictamente creciente de índices $\{i_j\}_{j\ge 0}$ tal que $\{b_{i_j}\}_{j\ge 0}$ es una secuencia estrictamente creciente de enteros. Luego, existe un $b_i$ tal que $b_i>M$, lo cual es una contradicción. Es decir, la secuencia $\{b_i\}_{i\ge 1}$ no es acotada superiormente. Aquí se termina la prueba.
\end{proof}

\begin{probEB}[IberoAmerican 2021/5]
	Para un conjunto finito $C$ de enteros, se define $S(C)$ como la suma de los elementos de $C$. Encuentre dos conjuntos no vacíos $A$ y $B$ cuya intersección es vacía, cuya unión es el conjunto $\{1,2,\dots,2021\}$ y tales que el producto $S(A)S(B)$ es un cuadrado perfecto.
	\aops{23437791}
\end{probEB}

\begin{proof}
	Considere a los números $(a,b)=(6063\cdot 9^2,6063\cdot 16^2)$. Note que $a+b=6063\cdot 337=\frac{2021\cdot 2022}{2}$ y $ab=(6063\cdot 9\cdot 16)^2$. Es claro que existen subconjuntos no vacíos y disjuntos $A$ y $B$ de $\{1,2,\dots,2021\}$ tales que $|A|=a$ y $|B|=b$. Aquí se termina la prueba.
\end{proof}

\begin{problem}[IberoAmerican 2021/6]
	Considere un polígono regular de $n$ lados, $n\ge 4$, y sea $V$ un subconjunto de $r$ vértices del polígono. Demuestre que si $r(r-3)\ge n$, entonces existen al menos dos triángulos congruentes cuyos vértices pertenecen a $V$.
	\aops{23437854}
\end{problem}

\begin{problem}[RMM 2021/1]
	Let $T_1,T_2,T_3,T_4$ be pairwise distinct collinear points such that $T_2$ lies between $T_1$ and $T_3$, and $T_3$ lies between $T_2$ and $T_4$. Let $\omega_1$ be a circle through $T_1$ and $T_4$; let $\omega_2$ be the circle through $T_2$ and internally tangent to $\omega_1$ at $T_1$; let $\omega_3$ be the circle through $T_3$ and externally tangent to $\omega_2$ at $T_2$; and let $\omega_4$ be the circle through $T_4$ and externally tangent to $\omega_3$ at $T_3$. A line crosses $\omega_1$ at $P$ and $W$, $\omega_2$ at $Q$ and $R$, $\omega_3$ at $S$ and $T$, and $\omega_4$ at $U$ and $V$, the order of these points along the line being $P,Q,R,S,T,U,V,W$. Prove that $PQ+TU=RS+VW$.
	\aops{23374851}
\end{problem}

\begin{probEG}[Balkan MO 2016/3]
	Find all monic polynomials $f$ with integer coefficients satisfying the following condition: there exists a positive integer $N$ such that $p$ divides $2(f(p)!)+1$ for every prime $p>N$ for which $f(p)$ is a positive integer.
	\aops{6316984}
\end{probEG}

\begin{proof}
	Es claro que $f(p)<p$ para todo primo $p>N$. Es decir, $\deg f=1$. En efecto, sea $f(x)=x-c$ donde $c\in\ZZ^+$. Si $c\le 2$, claramente $2(p-1)!+1\equiv -1\pmod p$ y $2(p-2)!+1\equiv 3\pmod p$ para $p>3$ lo cual es un absurdo. Si $c>3$, note que
	\[2(p-3)!+1\equiv 0\equiv 2(p-c)!+1\pmod p\]
	de donde $(p-c+1)(p-c+2)\cdots(p-3)\equiv 1\pmod p$. Luego, $3\cdot 4\cdots(c-1)\equiv(-1)^{c-3}\pmod p$ para todo $p>3$ de donde
	\[3\le 3\cdot 4\cdots(c-1)=(-1)^{c-3}\le 1\]
	lo cual es un absurdo. Por lo tanto, $c=3$ y $f(x)=x-3$.
\end{proof}

\begin{probEG}[RMM 2016/1]
	Let $ABC$ be a triangle and let $D$ be a point on the segment $BC$, $D\ne B$ and $D\ne C$. The circle $ABD$ meets the segment $AC$ again at an interior point $E$. The circle $ACD$ meets the segment $AB$ again at an interior point $F$. Let $A'$ be the reflection of $A$ in the line $BC$. The lines $A'C$ and $DE$ meet at $P$, and the lines $A'B$ and $DF$ meet at $Q$. Prove that the lines $AD$, $BP$ and $CQ$ are concurrent (or all parallel).
\end{probEG}

\begin{proof}
	Sean $E'$ y $F'$ los simétricos de $E$ y $F$ con respecto a la recta $BC$. Note que
	\[\angle BDF'=\angle FDB=\angle FAC\equiv\angle BAE=\angle CDE\]
	de donde $F'\in DE$. Análogamente, $E'\in DF$. Como $F'A'CD$ es cíclico, $PA'\cdot PC=PF'\cdot PD$ de donde $BP$ es el eje radical de $\odot(A'BC)$ y $\odot(F'DB)$. Análogamente, $CQ$ es el eje radical de $\odot(A'BC)$ y $\odot(E'DC)$. Como $\triangle DBF'\cong\triangle DE'C$, el punto $R=BE'\cap CF'$ es la intersección de las circunferencias $\odot(F'DB)$ y $\odot(E'DC)$. Luego,
	\[\angle F'DR=\angle F'BR\equiv\angle A'BE'=\angle EBA=\angle EDA\]
	de donde $R\in AD$. Es decir, $AD$ es el eje radical de $\odot(F'DB)$ y $\odot(E'DC)$. Por lo tanto, $AD$, $BP$ y $CQ$ concurren en el centro radical de las circunferencias $\odot(A'BC)$, $\odot(F'DB)$ y $\odot(E'DC)$.
\end{proof}

\begin{probHR}[RMM 2016/2]
	Given positive integers $m$ and $n\ge m$, determine the largest number of dominoes ($1\times 2$ or $2\times 1$ rectangles) that can be placed on a rectangular board with $m$ rows and $2n$ columns consisting of cells ($1\times 1$ squares) so that:
	\begin{enumerate}[(1)]
		\ii each domino covers exactly two adjacent cells of the board;
		\ii no two dominoes overlap;
		\ii no two form a $2\times 2$ square; and
		\ii the bottom row of the board is completely covered by $n$ dominoes.
	\end{enumerate}
\end{probHR}

\begin{proof}
	Sean $C_i$ el conjunto de los dominós verticales en la columna $i$, $H_i$ el conjunto de los dominós horizontales tales que sus dos casillas están en las columnas $i$ y $i+1$, y $X_i$ el conjunto de los dominós en $V_i$ que están en la fila $1$. Ahora, considere el subtablero de $2n\times 2$ formado por las columnas $i$ y $i+1$. Note que los dominós en $V_i$ separan al subtablero en varios subtableros de $2\times t_1,2\times t_2,\dots,2\times t_k$ donde
	\[\sum_{j=1}^k t_j=2n-\abs{V_i}.\]
	Es claro que un dominó que pertenece en $H_i$ o $H_{i+1}$ está en algún subtablero de $2\times t_j$. Por la tercera condición, en cada subtablero de $2\times t_j$ hay a lo sumo $t_j-1$ dominós horizontales. Por lo tanto,
	\[\abs{H_i}+\abs{H_{i+1}}\le\sum_{j=1}^k(t_j-1)=2n-\abs{V_i}-k\]
	de donde $\abs{H_i}+\abs{H_{i+1}}+2\abs{V_i}\le 2n-1+\abs{V_i}+1-k$. Luego,
	\begin{align*}
		2\sum_{i=1}^m\abs{H_i}+2\sum_{i=1}^m\abs{V_i}
		&\le\abs{H_1}+\abs{H_m}+(2n-1)(m-1) \\
		&=(\abs{H_1}-n)+(\abs{H_m}-n)+2mn-m+1 \\
		&\le 2mn-m+1
	\end{align*}
	de donde $mn-\floor{\frac{m}{2}}$ es el mayor número de fichas de dominó en el tablero.
\end{proof}

\begin{problem}[RMM 2016/3]
	A \emph{cubic sequence} is a sequence of integers given by $a_n=n^3+bn^2+cn+d$, where $b$, $c$ and $d$ are integer constants and $n$ ranges over all integers, including negative integers.
	\begin{enumerate}[(a)]
		\ii \label{enumi:cubic_sequence} Show that there exists a cubic sequence such that the only terms of the sequence which are squares of integers are $a_{2015}$ and $a_{2016}$.
		\ii Determine the possible values of $a_{2015}\cdot a_{2016}$ for a cubic sequence satisfying the condition in part \ref{enumi:cubic_sequence}.
	\end{enumerate}
\end{problem}

\begin{probMG}
	Sea $M$ un conjunto finito de enteros positivos. Demuestre que es posible agregarle elementos (puede ser ninguno) de tal manera que en el nuevo conjunto, la suma de todos sus elementos es igual al mínimo común múltiplo de todos sus elementos.
\end{probMG}

\begin{proof}
	Si $\abs{M}=1$, es trivial. Ahora, supongamos que $\abs{M}>1$. Si para algún entero $n>1$ existen enteros positivos distintos $x_1,x_2,\dots,x_k$ tales que
	\[\mcm(n-1,x_1,x_2,\dots,x_k)=1+x_1+x_2+\dots+x_k\]
	tenemos que
	\[\mcm(n,n-1,nx_1,nx_2,\dots,nx_k)=1+(n-1)+nx_1+nx_2+\dots+nx_k\]
	y
	\[\mcm(n,n^2-1,y_1,y_2,\dots,y_k)=(n+1)+(n^2-1)+y_1+y_2+\dots+y_k\]
	donde $y_i=n(n+1)x_i$ para todo $1\le i\le k$. Sea $n>1$ el mínimo común múltiplo de los elementos de $M$. Como $\mcm(1,2,3)=1+2+3$, por inducción existen enteros positivos distintos $x_1,x_2,\dots,x_k$ mayores que $n$ tales que
	\[\mcm(n,x_1,x_2,\dots,x_k)=n+1+x_1+x_2+\dots+x_k.\]
	Si $S$ es la suma de los elementos de $M$, tenemos que
	\[\mcm(n,Sn,Sx_1,Sx_2,\dots,Sx_k)=S+Sn+Sx_1+Sx_2+\dots+Sx_k.\]
	Es decir, agregando los números $Sn,Sx_1,Sx_2,\dots,Sx_k$ al conjunto $M$ podemos lograr lo deseado.
\end{proof}

\section{Semana 3 (03/28 -- 04/03)}

\note[EGMO TST Día 2]{Lunes\\2022-03-28}

\begin{probEG}[Peru EGMO TST 2022/5]
	Sean $x_0,x_1,x_2,\dots,x_n$ números reales distintos entre sí tales que
	\[x_{k-1}x_k\le x_k\le x_kx_{k+1},\quad\text{para todo }k=1,2,\dots,n-1.\]
	Determine el mayor valor posible de $n$.
\end{probEG}

\begin{proof}
	El mayor valor de $n$ es $6$ y un ejemplo que cumple es
	\[(x_0,x_1,\dots,x_6)=(5/2,-1/2,0,1/2,1,3/2,2).\]
	Si $n\ge 7$, supongamos que $x_i>0$ para algún $1\le i\le 3$. Considerando índices $i\le k\le i+3$ tenemos que $x_{i+1}\ge 1$, $x_{i+2}\ge 1$, $x_{i+1}\le 1$ y $x_{i+2}\le 1$ de donde $x_{i+1}=x_{i+2}=1$, lo cual es un absurdo. Por ende, $x_1,x_2,x_3\le 0$. Si $x_i<0$ para algún $2\le i\le 3$, como $x_{i-1}x_i\le x_i$ tenemos que $x_{i-1}\ge 1>0$, lo cual es una contradicción. Luego, $x_2=x_3=0$ lo cual es un absurdo.
\end{proof}

\begin{probEG}[Peru EGMO TST 2022/6]
	Para cada entero positivo $n$, sea $S(n)$ la suma de todos los dígitos de $n$. Pruebe que para todo entero positivo $n\ge 2022$ se cumple que $\floor{\frac{n}{S(n)}}\ne\floor{\frac{n+1}{S(n+1)}}$.
\end{probEG}

\begin{proof}
	Supongamos que $\floor{\frac{n}{S(n)}}=\floor{\frac{n+1}{S(n+1)}}=k\in\ZZ^+_0$. Si $1\le a\le 9$ y $\alpha\ge 3$ son enteros tales que $a\cdot 10^\alpha\le n<(a+1)\cdot 10^\alpha$, tenemos
	\begin{align*}
		n
		&\ge a\cdot 10^\alpha>a\cdot 100\alpha^2>a\cdot(81\alpha^2+27\alpha+10)\\
		&\ge (a+9\alpha)(a+9\alpha+1)\\
		&\ge S(n)\cdot(S(n)+1)
	\end{align*}
	de donde $k>\frac{n}{S(n)}-1>S(n)$. Luego,
	\[S(n)-S(n+1)<\frac{k\cdot (S(n)+1)}{k+1}-\frac{n+1}{k+1}\le\frac{k-1}{k+1}<1\]
	de donde $S(n+1)=S(n)+1$. Por ende,
	\[0<k\cdot(S(n)+1)-(k+1)\cdot S(n)<(n+1)-n=1\]
	lo cual es un absurdo.
\end{proof}

\begin{probEG}[Peru EGMO TST 2022/7]
	Sea $n\ge 3$ un número entero. Se tienen $n$ colores distintos $C_1,C_2,\dots,C_n$ y una cantidad ilimitada de fichas de cada uno de dichos colores. Decimos que un entero $m\ge n+1$ es \emph{n-colorido} si es posible colocar $m$ fichas alrededor de un círculo de modo que en cualquier grupo de $n+1$ fichas consecutivas haya al menos una ficha de cada uno de los colores $C_1,C_2,\dots,C_n$. Pruebe que solo existe una cantidad finita de enteros $m$ que no son $n$-coloridos y encuentre el mayor de ellos.
\end{probEG}

\begin{proof}
	Si $m=n(n-1)-1$ es un $n$-colorido, existe un color $C_i$ que se repite a lo sumo $n-2$ veces. Si consideramos una ficha de color $C_i$, podemos dividir las fichas restantes en $n-2$ sectores de $n+1$ fichas consecutivos. Entonces, existe un sector que no contiene al color $C_i$, lo cual es un absurdo. Ahora, sea $m=nq+r$ un entero positivo donde $q$ y $0\le r<n$ son enteros. Si $m\ge n(n-1)$, probaremos que $m$ es $n$-colorido. Primero, vamos a dividir el círculo en $q$ sectores, donde en cada sector colocaremos $n$ fichas de colores $C_1,C_2,\dots,C_n$ en ese orden. Como $q\ge n-1\ge r$, es posible elegir $r$ pares de sectores adyacentes $(S_i,S_{i+1})$ e insertar una ficha de cualquier color entre $S_i$ y $S_{i+1}$. Es fácil ver que esto cumple, así que $m$ es un $n$-colorido. Finalmente, $n^2-n-1$ es el mayor entero que no es $n$-colorido.
\end{proof}

\note{Martes\\2022-03-29}

\begin{probEG}[Kyiv City MO 2022 Round 2/11.3]
	Hallar el mayor $k\in\ZZ^+$ para el cual existe una permutación $a_1,a_2,\dots,a_{2022}$ de los números $1,2,\dots,2022$ tal que para al menos $k$ índices $1\le i\le 2022$, el siguiente número
	\[\frac{a_1+a_2+\dots+a_i}{1+2+\dots+i}\]
	es un entero mayor que $1$.
	\begin{hint}
		Find the largest $k$ for which there exists a permutation $(a_1,a_2,\dots,a_{2022})$ of integers from $1$ to $2022$ such that for at least $k$ distinct $i$ with $1\le i\le 2022$ the number
		\[\frac{a_1+a_2+\dots+a_i}{1+2+\dots+i}\]
		is an integer larger than $1$.
	\end{hint}
	\forum[aops]{24269928}
\end{probEG}

\begin{proof}
	El mayor valor de $k$ es $1011$ y un ejemplo es
	\[a_i=\begin{cases}
		2i&\text{si }i\le 1011,\\
		2i-2023&\text{si }i>1011.
	\end{cases}\]
	Ahora, supongamos que $k\ge 1012$. Sea
	\[A_i=\frac{a_1+a_2+\dots+a_i}{1+2+\dots+i}\]
	para todo índice $i$. Luego, existen dos índices $i>j\ge 1011$ tales que $A_i,A_j>1$ son enteros. Si $A_i\ge 3$, tenemos que
	\begin{align*}
		2023i-\frac{i(i+1)}{2}
		&=2022+2021+\dots+(2023-i)\\
		&\ge a_1+a_2+\dots+a_i\\
		&\ge 3(1+2+\dots+i)\\
		&=\frac{3i(i+1)}{2}
	\end{align*}
	de donde $i\le 1010$, lo cual es un absurdo. Por ende, $A_i=2$ y análogamente $A_j=2$. Luego,
	\begin{align*}
		a_{j+1}+a_{j+2}+\dots+a_i
		&=(a_1+a_2+\dots+a_i)-(a_1+a_2+\dots+a_j)\\
		&=2(1+2+\dots+i)-2(1+2+\dots+j)\\
		&=2((j+1)+(j+2)+\dots+i)\\
		&>\underbrace{2022+2022+\dots+2022}_{i-j\text{ veces}}
	\end{align*}
	lo cual es un absurdo.
\end{proof}

\note{Miércoles\\2022-03-30}

\begin{probMB}[MEMO 2021/T2]
	Dado un entero positivo $n$, decimos que un polinomio $P(x)\in\RR[x]$ es \emph{n-bonito} si la ecuación $P(\floor{x})=\floor{P(x)}$ tiene exactamente $n$ soluciones. Demuestre que para todo $n\in\ZZ^+$
	\begin{enumerate}[(a)]
		\ii existe algún polinomio $n$-bonito;
		\ii todo polinomio $n$-bonito tiene grado mayor o igual que $\frac{2n+1}{3}$.
	\end{enumerate}
	\begin{hint}
		Given a positive integer $n$, we say that a polynomial $P$ with real coefficients is \emph{n-pretty} if the equation $P(\floor{x})=\floor{P(x)}$ has exactly $n$ real solutions. Show that for each positive integer $n$
		\begin{enumerate}[(a)]
			\ii there exists an $n$-pretty polynomial;
			\ii any $n$-pretty polynomial has a degree of at least $\frac{2n+1}{3}$.
		\end{enumerate}
	\end{hint}
	\forum[aops]{23091282}
\end{probMB}

\begin{proof}
	Sea $A$ el conjunto de las soluciones reales a la ecuación $P(\floor x)=\floor{P(x)}$. Note que si $x\in A$ entonces $\floor x\in A$. Luego,
	\[A=I\cup\bigcup_{i\in I}S_i\]
	donde $I=\{x\in\ZZ:P(x)\in\ZZ\}$ y $S_i=\{x\in(i,i+1):P(x)\in[P(i),P(i)+1)\}$. Por ende,
	\[n=\abs A=\abs I+\sum_{i\in I}\abs{S_i}.\]
	Si $\abs I>\deg P$, es posible construir un polinomio $Q\in\QQ[x]$ tal que $\deg Q\le\deg P$ y $P(x)-Q(x)$ tiene más de $\deg P$ raíces reales. Luego, $P(x)\equiv Q(x)$ de donde existen infinitos $x\in\ZZ$ tales que $P(x)\in\ZZ$, lo cual es un absurdo ya que $\abs I\le n$. Es decir, $\abs I\le\deg P$. Sea $\delta\in(i,i+1)$ para algún $i\in I$. Si $P(\delta)>P(i)$, para todo $\epsilon\in(P(i),P(\delta))$ existe algún $x\in(i,\delta)$ tal que $P(x)=\epsilon$. Luego, existen infinitos $x\in(i,i+1)$ tales que $P(x)\in(P(i),P(i)+1)$, lo cual es un absurdo ya que $\abs{S_i}\le n$. Por ende, $P(x)\le P(i)$ para todo $x\in(i,i+1)$ y $P(x)=P(i)$ para todo $x\in S_i$. Es decir, en el intérvalo $(i,i+1)$ existen $2\abs{S_i}$ ceros reales de la derivada de $P$. Luego,
	\[n=\abs I+\sum_{i\in I}\abs{S_i}<\deg P+\frac12\deg P=\frac32\deg P\]
	de donde $\deg P\ge\frac{2n+1}{3}$. Un ejemplo de un polinomio $n$-bonito es
	\[P(x)=-\pi(x-1)^2(x-2)^2\cdots(x-n)^2\]
	donde $I=\{1,2,\dots,n\}$ y $S_i=\varnothing$ para todo $i\in I$.
\end{proof}

\begin{probMG}[MEMO 2020/I4]
	Determine todos los enteros positivos $n$ para los cuales existen enteros positivos $x_1,x_2,\dots,x_n$ tales que
	\[\frac{1}{x_1^2}+\frac{2}{x_2^2}+\frac{4}{x_3^2}+\dots+\frac{2^{n-1}}{x_n^2}=1.\]
	\begin{hint}
		Find all positive integers $n$ for which there exist positive integers $x_1,x_2,\dots,x_n$ such that
		\[\frac{1}{x_1^2}+\frac{2}{x_2^2}+\frac{4}{x_3^2}+\dots+\frac{2^{n-1}}{x_n^2}=1.\]
	\end{hint}
	\forum[aops]{17377469}
\end{probMG}

\begin{proof}
	Probaremos que $\ZZ^+\setminus\{2\}$ es el conjunto de valores de $n$.
	\begin{itemize}
		\ii Si $n=1$, es suficiente con $x_1=1$.
		\ii Si $n=2$, es claro que $x_1,x_2\ge 2$. Luego, $1=\frac{1}{x_1^2}+\frac{2}{x_2^2}\le\frac34$ lo cual es un absurdo.
		\ii Si $n\ge 3$ es impar, sean $x_n=2^{n-1}$ y $x_i=2^\frac{n-1}{2}$ para todo $1\le i\le n-1$. Luego,
		\[\sum_{i=1}^n\frac{2^{i-1}}{x_i^2}=\sum_{i=1}^{n-1}\frac{2^{i-1}}{2^{n-1}}+\frac{2^{n-1}}{(2^{n-1})^2}=\frac{2^{n-1}-1}{2^{n-1}}+\frac{1}{2^{n-1}}=1.\]
		\ii Si $n\ge 4$ es par, sean $(x_1,x_{n-1},x_n)=\left(3\cdot 2^\frac{n-2}{2},2^{n-2},3\cdot 2^{n-3}\right)$ y $x_i=2^\frac{n-2}{2}$ para todo $2\le i\le n-2$. Luego,
		\begin{align*}
			\sum_{i=1}^n\frac{2^{i-1}}{x_i^2}
			&=\frac{1}{9\cdot 2^{n-2}}+\sum_{i=2}^{n-2}\frac{2^{i-1}}{2^{n-2}}+\frac{2^{n-2}}{(2^{n-2})^2}+\frac{2^{n-1}}{9\cdot(2^{n-3})^2}\\
			&=\frac{1}{2^{n-2}}+\frac{2^{n-2}-2}{2^{n-2}}+\frac{1}{2^{n-2}}\\
			&=1.
		\end{align*}
	\end{itemize}
\end{proof}

\note{Viernes\\2022-04-01}

\begin{probEG}[MEMO 2018/I4]
	\begin{enumerate}[(a)]
		\ii Demuestre que para todo $m\in\ZZ^+$ existe un entero $n\ge m$ tal que
		\begin{equation}\label{eq:floor_product}\tag{$\ast$}
			\floor{\frac n1}\cdot\floor{\frac n2}\cdots\floor{\frac nm}=\binom nm.
		\end{equation}
		\ii Sea $p(m)$ el menor entero $n\ge m$ tal que la ecuación \eqref{eq:floor_product} cumple. Demuestre que $p(2018)=p(2019)$.
	\end{enumerate}
	\begin{hint}
		\begin{enumerate}[(a)]
			\ii Prove that for every positive integer $m$ there exists an integer $n\ge m$ such that
			\begin{equation}\tag{$\ast$}
				\floor{\frac n1}\cdot\floor{\frac n2}\cdots\floor{\frac nm}=\binom nm.
			\end{equation}
			\ii Denote by $p(m)$ the smallest integer $n\ge m$ such that the equation ($\ast$) holds. Prove that $p(2018)=p(2019)$.
		\end{enumerate}
	\end{hint}
	\forum[aops]{10959197}
\end{probEG}

\begin{proof}
	Note que
	\[\prod_{i=1}^m\floor{\frac ni}\ge\prod_{i=1}^m\frac{n-i+1}{i}=\binom nm\]
	de donde $i\mid n-i+1$ para todo $1\le i\le m$. Es decir, $n=k\cdot\mcm(1,2,\dots,m)-1$ para algún $k\in\ZZ^+$. De esto tenemos que $p(m)=\mcm(1,2,\dots,m)-1$ para todo $m>1$. Como $2019=3\times 673$ divide a $\mcm(1,2,\dots,2018)$, tenemos que
	\[p(2018)=\mcm(1,2,\dots,2018)-1=\mcm(1,2,\dots,2019)-1=p(2019).\]
	Con esto se termina la prueba.
\end{proof}

\begin{probEG}[MEMO 2018/T7]
	Sea $(a_n)_{n\ge 1}$ una secuencia definida por
	\[a_1=1\quad\text{y}\quad a_{k+1}=a_k^3+1,\text{ para todo }k\in\ZZ^+.\]
	Demuestre que para todo número primo $p$ de la forma $3\ell+2$ con $\ell\in\ZZ^+_0$, existe $n\in\ZZ^+$ tal que $p\mid a_n$.
	\begin{hint}
		Let $a_1,a_2,a_3,\dots$ be the sequence of positive integers such that
		\[a_1=1\quad\text{and}\quad a_{k+1}=a_k^3+1,\text{ for all positive integers }k.\]
		Prove that for every prime number $p$ of the form $3\ell+2$, where $\ell$ is a non-negative integer, there exists a positive integer $n$ such that $a_n$ is divisible by $p$.
	\end{hint}
	\forum[aops]{10935321}
\end{probEG}

\begin{proof}
	Note que si $x^3\equiv y^3\pmod p$, tenemos que
	\[x\equiv(x^3)^\frac{2p-1}{3}\equiv(y^3)^\frac{2p-1}{3}\equiv y\pmod p.\]
	Como el mapeo $x\mapsto x^3+1\pmod p$ es biyectivo, la secuencia $(a_i)$ es periódica en módulo $p$. Es decir, $a_{n+1}\equiv a_1\equiv 1\pmod p$ para algún $n\in\ZZ^+$, de donde $a_n\equiv 0\pmod p$.
\end{proof}

\begin{probMB}[MEMO 2018/T8]
	Un entero positivo $n$ es llamado \emph{interesante} si existen enteros positivos $a$, $b$ y $c$ tales que
	\[n=\frac{a^2+b^2+c^2}{ab+bc+ca}.\]
	\begin{enumerate}[(a)]
		\ii Demuestre que existen infinitos números interesantes.
		\ii Demuestre que no todos los enteros positivos son interesantes.
	\end{enumerate}
	\begin{hint}
		An integer $n$ is called \emph{Silesian} if there exist positive integers $a$, $b$ and $c$ such that
		\[n=\frac{a^2+b^2+c^2}{ab+bc+ca}.\]
		\begin{enumerate}[(a)]
			\ii Prove that there are infinitely many Silesian integers.
			\ii Prove that not every positive integer is Silesian.
		\end{enumerate}
	\end{hint}
	\forum[aops]{10931715}
\end{probMB}

\begin{proof}
	Si $n=4$, digamos que $\mcd(a,b,c)=1$. Luego, $4\mid a^2+b^2+c^2$ de donde $a,b,c$ son pares, lo cual es un absurdo. Por lo tanto, $4$ no es Silesio. Ahora, supongamos que $n=a^2+(1-a)^2+(a^2-a+1)^2$, donde $b=1-a$ y $c=a^2-a+1$ para algún valor de $a$. Podemos comprobar que esto cumple, por lo que $b=1-a$ es una raíz de la ecuación $b^2-n(c+a)b+(c^2+a^2-nca)=0$. Es decir, la otra raíz es igual a $n(c+a)-(1-a)>0$. Por ende, $a^2+(1-a)^2+(a^2-a+1)^2$ es Silesio para todo $a\in\ZZ^+$. Aquí se termina la prueba.
\end{proof}

\section{Semana 4 (04/04 -- 04/10)}

\note[Entrenamien-to EGMO]{Lunes\\2022-04-04}

\begin{probMR}[Balkan MO 2005/4]
	Let $n\ge 2$ be an integer. Let $S$ be a subset of $\{1,2,\dots,n\}$ such that $S$ neither contains two elements one of which divides the other, nor contains two elements which are coprime. What is the maximal possible number of elements of such a set $S$?
	\aops{225001}
\end{probMR}

\begin{proof}
	La respuesta es $\floor{\frac{n+2}{4}}$.
\end{proof}

\begin{probMB}[Croatian MO 2018/5]
	Let $n$ be a positive integer. $A_1,A_2,\dots,A_n$ are points inside a circle and $B_1,B_2,\dots,B_n$ points on that circle such that the segments $A_iB_i$ are pairwise disjoint for $1\le i\le n$. A grasshopper can move from point $A_i$ to $A_j$ where $i\ne j$ if and only if segment $A_iA_j$ doesn't pass through any of segments $A_kB_k$ for $1\le k\le n$.

	Prove that the grasshopper can move from any point $A_i$ to $A_j$ in a finite sequence of moves.
	\aops{12394250}
\end{probMB}

\note[Entrenamien-to EGMO]{Martes\\2022-04-05}

\begin{probEG}[Pedro Alegría, Spain]
	Demostrar que $x$ es racional si y solo si la sucesión
	\[x,x+1,x+2,\dots\]
	contiene al menos tres términos en progresión geométrica.
\end{probEG}

\begin{proof}
	Supongamos que $x$ es un número racional. Si $x$ no es positivo, vamos a sumarle $1$ hasta que sea positivo. Sea $x=\frac pq$ donde $p,q\in\ZZ^+$. Luego, los números $x$, $x+p$ y $x+p(q+2)$ están en una progresión geométrica. Ahora, supongamos que existen tres enteros $i,j,k\ge 0$ tales que $x+i$, $x+j$ y $x+k$ están en una progresión geométrica. En efecto, $(x+i)(x+k)=(x+j)^2$ de donde $x(i+k-2j)=j^2-ik$. Si $i+k=2j$, tenemos que $(i-k)^2=(i+k)^2-4ik=4(j^2-ik)=0$ de donde $i=k$, lo cual es un absurdo. Por ende, $x$ es racional.
\end{proof}

\begin{probEG}[CGMO 2008/8]
	Let $f_n=\floor{2^n\sqrt{2008}}+\floor{2^n\sqrt{2009}}$ for all $n\in\ZZ^+$. Prove there are infinitely many odd numbers and infinitely many even numbers in the sequence $f_1,f_2,\dots$.
	\aops{1236876}
\end{probEG}

\begin{proof}
	Sea $a_n=1$ si $\left\{2^n\sqrt{2008}\right\}>\half$ y $0$ de lo contrario. Análogamente se define $b_n$ con respecto a $2009$. Si solamente hay finitos números de alguna paridad, tenemos que $f_{n+1}-2f_n=a_n+b_n$ tiene la misma paridad para $n$ suficientemente grande. Luego, $a_n=b_n$ o $a_n+b_n=1$ para todo $n\ge N$ donde $N\in\ZZ^+$. Por ende, $2^N\sqrt{2008}\pm2^N\sqrt{2009}$ es racional, lo cual es un absurdo. Finalmente, existen infinitos números pares e impares en la secuencia.
\end{proof}

\note[Miscelánea]{Martes\\2022-04-05}

\begin{probEB}[IMO Shortlist 2003 A1]
	Let $a_{ij}$ ($i=1,2,3$, $j=1,2,3$) be real numbers such that $a_{ij}$ is positive for $i=j$ and negative for $i\ne j$.

	Prove the existence of positive real numbers $c_{1}$, $c_{2}$, $c_{3}$ such that the numbers
	\[a_{11}c_{1}+a_{12}c_{2}+a_{13}c_{3},\quad a_{21}c_{1}+a_{22}c_{2}+a_{23}c_{3},\quad a_{31}c_{1}+a_{32}c_{2}+a_{33}c_{3}\]
	are either all negative, all positive, or all zero.
\end{probEB}

\begin{probEG}
	Si $a^2<b^2$ son dos números de $1001$ dígitos, demuestre que existe un número capicúa en el intervalo $(a^2,b^2)$.
\end{probEG}

\begin{proof}
	Es claro que entre $a^2+1,a^2+2,\dots,a^2+10^{500}$ existe un múltiplo de $10^{500}$ y sea $\overline{a_{500}a_{499}\dots a_1a_0}\cdot 10^{500}$ dicho número. Luego,
	\[a^2<\overline{a_{500}a_{499}\dots a_1a_0a_1\dots a_{499}a_{500}}\le a^2+2\cdot 10^{500}<(a+1)^2\le b^2\]
	y aquí se termina la prueba.
\end{proof}

\begin{probEG}[IMO 2014/1]
	Let $a_0<a_1<a_2<\cdots$ be an infinite sequence of positive integers. Prove that there exists a unique integer $n\ge 1$ such that
	\[a_n<\frac{a_0+a_1+a_2+\dots+a_n}{n}\le a_{n+1}.\]
	\aops{3542095}
\end{probEG}

\begin{probEG}
	Sean $a_1,a_2,\dots,a_n$, $k$ y $M$ enteros positivos tales que
	\[\frac{1}{a_1}+\frac{1}{a_2}+\dots+\frac{1}{a_n}=k\quad\text{y}\quad M=a_1a_2\cdots a_n.\]
	Si $M>1$, demuestre que el polinomio
	\[P(x)=M(x+1)^k-(x+a_1)(x+a_2)\cdots(x+a_n)\]
	no tiene raíces positivos.
\end{probEG}

\begin{proof}
	Si $r>0$ es una raíz de $P(x)$, tenemos que
	\[\prod_{i=1}^n\frac{a_i(r+1)^\frac{1}{a_i}}{r+a_i}=1\]
	pero note que
	\[a^a(r+1)=\binom{a}{1}ra^{a-1}+a^a\le(r+a)^a\]
	de donde
	\[\frac{a(r+1)^\frac{1}{a}}{r+a}\le 1\]
	para todo $a\in\ZZ^+$. Es decir, $a_i=1$ para todo $1\le i\le n$ de donde $M=1$ lo cual es un absurdo.
\end{proof}

\begin{probEG}
	Determine todas las parejas $(x,y)$ de enteros positivos tales que
	\[\sqrt[3]{7x^2-13xy+7y^2}=\abs{x-y}+1.\]
\end{probEG}

\begin{proof}
	Sin pérdida de generalidad, supongamos que $x\ge y$. No es difícil ver que
	\[(x-y-2)^2(4x-4y+1)=(x+y)^2\]
	de donde $x-y=k^2+k$ para algún $k\in\ZZ^+_0$. Si $k=0$, tenemos que $x-y=0$ y $x+y=2$ de donde $x=y=1$. Si $k\ge 1$, note que
	\[x+y=(k^2+k-2)(2k+1)=2k^3+3k^2-3k-2\]
	de donde $(x,y)=(k^3+2k^2-k-1,k^3+k^2-2k-1)$ para algún $k\ge 2$.
\end{proof}

\begin{problem}
	Para cada entero positivo $n$, el Banco de Ciudad del Cabo produce monedas de valor $\frac1n$. Dada una colección finita de tales monedas (no necesariamente de distintos valores) cuyo valor total no supera $99+\frac12$, demostrar que es posible separar esta colección en $100$ o menos montones, de modo que el valor total de cada montón sea como máximo $1$.
\end{problem}

\begin{probEG}[EGMO 2022/1]
	Let $ABC$ be an acute-angled triangle in which $BC<AB$ and $BC<CA$. Let point $P$ lie on segment $AB$ and point $Q$ lie on segment $AC$ such that $P\ne B$, $Q\ne C$ and $BQ=BC=CP$. Let $T$ be the circumcentre of triangle $APQ$, $H$ the orthocentre of triangle $ABC$, and $S$ the point of intersection of the lines $BQ$ and $CP$. Prove that $T$, $H$ and $S$ are collinear.
	\begin{probSM}
		Sea $ABC$ un triángulo acutángulo con $BC<AB$ y $BC<AC$. Considere los puntos $P$ y $Q$ en los segmentos $AB$ y $AC$, respectivamente, tales que $P\ne B$, $Q\ne C$ y $BQ=BC=CP$. Sea $T$ el circuncentro del triángulo $APQ$, $H$ el ortocentro del triángulo $ABC$ y $S$ el punto de intersección de las rectas $BQ$ y $CP$. Pruebe que los puntos $T$, $H$ y $S$ están en una misma recta.
	\end{probSM}
\end{probEG}

\begin{proof}
	Como $BH$ y $CH$ son las bisectrices del triángulo $SBC$, la recta $HS$ biseca al ángulo $\angle BSC$. Note que
	\[\angle PSQ=\angle BSC=2\angle BHC-180\dg=2(180\dg-\angle BAC)-180\dg=180\dg-\angle PTQ\]
	de donde $TPSQ$ es cíclico. Como $T$ es el punto medio del arco $PQ$ que no contiene a $S$, la recta $TS$ biseca al ángulo $\angle PSQ$. Por ende, $T$, $H$ y $S$ pertenecen a la bisectriz del ángulo $\angle BSC$.
\end{proof}

\begin{probMG}[EGMO 2022/2]
	Find all functions $f:\NN\to\NN$ such that for any positive integers $a$ and $b$, the following two conditions hold:
	\begin{enumerate}[(1)]
		\ii $f(ab)=f(a)f(b)$, and
		\ii at least two of the numbers $f(a)$, $f(b)$ and $f(a+b)$ are equal.
	\end{enumerate}
	\begin{probSM}
		Determine todas las funciones $f:\NN\to\NN$ tales que para cualquier pareja de enteros positivos $a$ y $b$, se cumplen las siguientes dos condiciones:
		\begin{enumerate}[(1)]
			\ii $f(ab)=f(a)f(b)$, y
			\ii al menos dos de los números $f(a)$, $f(b)$ y $f(a+b)$ son iguales.
		\end{enumerate}
	\end{probSM}
\end{probMG}

\begin{proof}
	Si $p\in\ZZ^+$ es el menor tal que $f(p)>1$, es claro que $p$ es primo. Si $a>p$ es el menor número coprimo con $p$ tal que $f(a)>1$, considerando a $f(p)>1$, $f(a-p)=1$ y $f(a)>1$ tenemos que $f(a)=f(p)$. Si $a>p^2$, considerando a $f(p^2)=f(p)^2>1$, $f(a-p^2)=1$ y $f(a)>1$ tenemos que $f(p)=f(a)=f(p)^2$ lo cual es un absurdo. Ahora, sea $k=\floor{p^2/a}$ donde $0<k<p$. Como $0<p^2-ka<a$, considerando a $f(ka)=f(p)$, $f(p^2-ka)=1$ y $f(p^2)=f(p)^2$ tenemos un absurdo. Por ende, $f(a)=1$ para todo $a\in\ZZ^+$ coprimo con $p$ de donde $f(n)=a^{\nu_p(n)}$ para algún $a\in\ZZ^+$. Por lo tanto, $f(n)=1$ o $f(n)=a^{\nu_p(n)}$ para algún entero $a>1$ y un primo $p$.
\end{proof}

\begin{probEG}[EGMO 2022/3]
	An infinite sequence of positive integers $a_1,a_2,\dots$ is called \emph{good} if
	\begin{enumerate}[(1)]
		\ii $a_1$ is a perfect square, and
		\ii for any integer $n\ge 2$, $a_n$ is the smallest positive integer such that
		\[na_1+(n-1)a_2+\dots+2a_{n-1}+a_n\]
		is a perfect square.
	\end{enumerate}
	Prove that for any good sequence $a_1,a_2,\dots$, there exists a positive integer $k$ such that $a_n=a_k$ for all integers $n\ge k$.
	\begin{probSM}
		Se dice que una sucesión infinita de enteros positivos $a_1,a_2,\dots$ es \emph{húngara} si
		\begin{enumerate}[(1)]
			\ii $a_1$ es un cuadrado perfecto, y
			\ii para todo entero $n\ge 2$, $a_n$ es el menor entero positivo tal que
			\[na_1+(n-1)a_2+\dots+2a_{n-1}+a_n\]
			es un cuadrado perfecto.
		\end{enumerate}
		Pruebe que si $a_1,a_2,\dots$ es una sucesión húngara, entonces existe un entero positivo $k$ tal que $a_n=a_k$ para todo entero $n\ge k$.
	\end{probSM}
\end{probEG}

\begin{proof}
	Sea $c_n^2=na_1+(n-1)a_2+\dots+2a_{n-1}+a_n$. Note que $c_n-c_{n-1}$ es decreciente y $c_n$ es estrictamente creciente, de donde existe un $k\in\ZZ^+$ tal que $(c_n-c_{n-1}-1)^2\le c_n$ para todo $n\ge k$. Es decir,
	\[(2c_n-c_{n-1}-1)^2-c_n^2\le c_n^2-c_{n-1}^2<(2c_n-c_{n-1})^2-c_n^2\]
	de donde $c_{n+1}-c_n=c_n-c_{n-1}=d$ donde $d\in\ZZ^+$ es fijo. Es decir,
	\[a_{n+1}=(c_{n+1}^2-c_n^2)-(c_n^2-c_{n-1}^2)=2d^2\]
	es fijo para todo $n\ge k$.
\end{proof}

\begin{probEG}
	Determine todos los polinomios $f(x)$ y $g(x)$ con coeficientes enteros tales que
	\[f(g(x))=x^{2015}+2x+1\]
	para todo $x\in\RR$.
\end{probEG}

\begin{probEG}
	Determine si existe una secuencia $(a_i)_{i\ge 1}$ de enteros positivos tales que cada entero positivo aparece exactamente una vez en las secuencias $(a_i)_{i\ge 1}$ y $(\abs{a_i-a_{i-1}})_{i\ge 1}$.
\end{probEG}

\begin{probEG}[EGMO 2022/4]
	Given a positive integer $n\ge 2$, determine the largest positive integer $N$ for which there exist $N+1$ real numbers $a_0,a_1,\dots,a_N$ such that
	\begin{enumerate}[(1)]
		\ii $a_0+a_1=-\frac1n$, and
		\ii $(a_k+a_{k-1})(a_k+a_{k+1})=a_{k-1}-a_{k+1}$ for $1\le k\le N-1$.
	\end{enumerate}
	\begin{probSM}
		Para cada entero positivo $n\ge 2$, determine el mayor entero positivo $N$ con la propiedad de que existen $N+1$ números reales $a_0,a_1,\dots,a_N$ tales que
		\begin{enumerate}[(1)]
			\ii $a_0+a_1=-\frac1n$, y
			\ii $(a_k+a_{k-1})(a_k+a_{k+1})=a_{k-1}-a_{k+1}$ para todo $1\le k\le N-1$.
		\end{enumerate}
	\end{probSM}
\end{probEG}

\begin{proof}
	Sea $S_i=a_i+a_{i+1}$ para todo $0\le i\le N-1$. Luego, $(1-S_k)(1+S_{k-1})=1$ para todo $1\le k\le N-1$ de donde si $S_{k-1}=-\frac{1}{n-k+1}$ entonces $S_k=-\frac{1}{n-k}$. Es fácil ver que el mayor valor de $N$ es $n$.
\end{proof}

\begin{probEG}[EGMO 2022/5]
	For all positive integers $n,k$, let $f(n,2k)$ be the number of ways an $n\times 2k$ board can be fully covered by $nk$ dominoes of size $2\times 1$. (For example, $f(2,2)=2$ and $f(3,2)=3$.) Find all positive integers $n$ such that for every positive integer $k$, the number $f(n,2k)$ is odd.
	\begin{probSM}
		Dados $n$ y $k$ enteros positivos, sea $f(n,2k)$ el número de formas en que un tablero de tamaño $n\times 2k$ puede ser completamente cubierto por $nk$ fichas de dominó de tamaño $2\times 1$. Encuentre todos los enteros positivos $n$ tales que para todo entero positivo $k$, el número $f(n,2k)$ es impar.
	\end{probSM}
\end{probEG}

\begin{proof}
	Si $n$ es impar, por simetría con respecto a la fila central, podemos notar que $f(n,2k)\equiv f(\frac{n-1}{2},2k)\pmod 2$. Si $n$ es par, por simetría con respecto a la diagonal principal, podemos notar que $f(n,n)$ es par. Ahora, sea $n=2^tn_1-1$ donde $n_1$ es impar. Si $n_1>1$,
	\[f(2^tn_1-1,n_1-1)\equiv f(2^{t-1}n_1-1,n_1-1)\equiv\dots\equiv f(n_1-1,n_1-1)\equiv 0\pmod 2\]
	de donde $n_1=1$. Por ende,
	\[f(n,2k)=f(2^t-1,2k)\equiv f(2^{t-1}-1,2k)\equiv\dots\equiv f(1,2k)=1\pmod 2\]
	para todo $k\in\ZZ^+$. Aquí se termina la prueba.
\end{proof}

\begin{probEG}[EGMO 2022/6]
	Let $ABCD$ be a cyclic quadrilateral with circumcentre $O$. Let the internal angle bisectors at $A$ and $B$ meet at $X$, the internal angle bisectors at $B$ and $C$ meet at $Y$, the internal angle bisectors at $C$ and $D$ meet at $Z$, and the internal angle bisectors at $D$ and $A$ meet at $W$. Further, let $AC$ and $BD$ meet at $P$. Suppose that the points $X$, $Y$, $Z$, $W$, $O$ and $P$ are distinct. Prove that $O$, $X$, $Y$, $Z$ and $W$ lie on the same circle if and only if $P$, $X$, $Y$, $Z$ and $W$ lie on the same circle.
	\begin{probSM}
		Sea $ABCD$ un cuadrilátero cíclico con circuncentro $O$. Sea $X$ el punto de intersección de las bisectrices de los ángulos $\angle DAB$ y $\angle ABC$; sea $Y$ el punto de intersección de las bisectrices de los ángulos $\angle ABC$ y $\angle BCD$; sea $Z$ el punto de intersección de las bisectrices de los ángulos $\angle BCD$ y $\angle CDA$; y sea $W$ el punto de intersección de las bisectrices de los ángulos $\angle CDA$ y $\angle DAB$. Sea $P$ el punto de intersección de las rectas $AC$ y $BD$. Suponga que los puntos $O$, $P$, $X$, $Y$, $Z$ y $W$ son distintos. Pruebe que $O$, $X$, $Y$, $Z$ y $W$ están sobre una misma circunferencia si y solo si $P$, $X$, $Y$, $Z$ y $W$ están sobre una misma circunferencia.
	\end{probSM}
\end{probEG}

\begin{proof}
	Si $Q=AB\cap CD$, $R=AD\cap BC$, $S=OQ\cap PR$ y $T=OR\cap PQ$ entonces $QS\cdot QO=QA\cdot QB=QY\cdot QW$. Luego, si $\odot(XYZW)\neq\odot(OSPT)$ entonces $QR$ es el eje radical de ellas. Es decir, como $O$ y $P$ no pertenecen a $QR$, entonces $O$ y $P$ no están sobre $\odot(XYZW)$. Si $\odot(XYZW)=\odot(OSPT)$ entonces $O$ y $P$ están sobre $\odot(XYZW)$. Con esto se termina la prueba.
\end{proof}

\section{Week 5 (04/11 -- 04/17)}

\begin{probEG}[IMO Shortlist 2003 A2]
	Find all nondecreasing functions $f:\RR\to\RR$ such that
	\begin{enumerate}[(i)]
		\ii $f(0)=0,f(1)=1$;
		\ii $f(a)+f(b)=f(a)f(b)+f(a+b-ab)$ for all real numbers $a,b$ such that $a<1<b$.
	\end{enumerate}
\end{probEG}

\begin{probEG}[IMO Shortlist 2003 A3]
	Consider pairs of sequences of positive real numbers
	\[a_1\ge a_2\ge a_3\ge\cdots,\quad b_1\ge b_2\ge b_3\ge\cdots\]
	and the sums
	\[A_n=a_1+\dots+a_n,\quad B_n=b_1+\dots+b_n;\quad n=1,2,\dots.\]
	For any pair define $c_i=\min\{a_i,b_i\}$ and $C_n=c_1+\dots+c_n$, $n=1,2,\dots$.
	\begin{enumerate}[(1)]
		\ii \label{enumi:bounded_sequence} Does there exist a pair $(a_i)_{i\ge 1}$, $(b_i)_{i\ge 1}$ such that the sequences $(A_n)_{n\ge 1}$ and $(B_n)_{n\ge 1}$ are unbounded while the sequence $(C_n)_{n\ge 1}$ is bounded?
		\ii Does the answer to question \ref{enumi:bounded_sequence} change by assuming additionally that $b_i=1/i$, $i=1,2,\dots$?
	\end{enumerate}
\end{probEG}

\begin{probMR}[IMO Shortlist 2003 A4]
	Let $n$ be a positive integer and let $x_1\le x_2\le\dots\le x_n$ be real numbers.
	\begin{enumerate}[(1)]
		\ii Prove that
		\[\left(\sum_{i,j=1}^n\abs{x_i-x_j}\right)^2\le\frac{2(n^2-1)}{3}\sum_{i,j=1}^n(x_i-x_j)^2.\]
		\ii Show that the equality holds if and only if $x_1,\dots,x_n$ is an arithmetic sequence.
	\end{enumerate}
\end{probMR}

\begin{probEG}[Kosovo MO 2021/10.4]
	Let $M$ be the midpoint of segment $BC$ of $\triangle ABC$. Let $D$ be a point such that $AD=AB$, $AD\perp AB$ and points $C$ and $D$ are on different sides of $AB$. Prove that
	\[\sqrt{AB\cdot AC+BC\cdot AM}\ge\frac{\sqrt{2}}{2}CD.\]
	\aops{20639935}
\end{probEG}

\begin{probEG}[IMO Shortlist 2003 C4]
	Let $x_1,\dots,x_n$ and $y_1,\dots,y_n$ be real numbers. Let $A=(a_{ij})_{1\le i,j\le n}$ be the matrix with entries
	\[a_{ij}=\begin{cases}1,&\text{if }x_i+y_j\ge 0;\\0,&\text{if }x_i+y_j<0.\end{cases}\]
	Suppose that $B$ is an $n\times n$ matrix with entries $0$, $1$ such that the sum of the elements in each row and each column of $B$ is equal to the corresponding sum for the matrix $A$. Prove that $A=B$.
\end{probEG}

\begin{probMR}[IMO Shortlist 2003 C5]
	Every point with integer coordinates in the plane is the center of a disk with radius $1/1000$.
	\begin{enumerate}[(1)]
		\ii Prove that there exists an equilateral triangle whose vertices lie in different discs.
		\ii Prove that every equilateral triangle with vertices in different discs has side-length greater than $96$.
	\end{enumerate}
\end{probMR}

\begin{problem}[IMO Shortlist 2003 C6]
	Let $f(k)$ be the number of integers $n$ that satisfy the following conditions:
	\begin{enumerate}[(i)]
		\ii $0\le n<10^k$, so $n$ has exactly $k$ digits (in decimal notation), with leading zeroes allowed;
		\ii the digits of $n$ can be permuted in such a way that they yield an integer divisible by $11$.
	\end{enumerate}
	Prove that $f(2m)=10f(2m-1)$ for every positive integer $m$.
\end{problem}

\section{Semana 6 (04/18 -- 04/24)}

\note[Álgebra]{Lunes\\2022-04-18}

\begin{probEG}
	Sean $a,b,c,d,e,f$ números reales no negativos que suman $6$. Determine el máximo valor de
	\[abc+bcd+cde+def+efa+fab.\]
\end{probEG}

\begin{proof}
	La respuesta es $8$ (un ejemplo es $(a,b,c,d,e,f)=(2,2,2,0,0,0)$). Es claro que la expresión dada es igual a $a(bc+ef+fb)+d(bc+ce+ef)$. Analizando por casos, tenemos que $0\in\{a,d\}\cap\{b,e\}\cap\{c,f\}$. Es decir, la expresión dada es igual a $0$ o sin pérdida de generalidad $a,b,c\ne 0$ y $d,e,f=0$. Luego, la expresión dada es igual a $abc\le\left(\frac{a+b+c}{3}\right)^3=8$.
\end{proof}

\begin{probMR}
	Una secuencia $(a_n)$ de números reales no negativos satisface
	\[\abs{a_m-a_n}\ge\frac{1}{m+n},\quad\forall\,m\ne n.\]
	Demuestre que si $a_n<c$ para todo $n\in\ZZ^+$, entonces $c\ge 1$.
\end{probMR}

\begin{proof}
	Note que si $i_1,i_2,\dots,i_n$ es una permutación de $1,2,\dots,n$ tal que $a_{i_1}\ge a_{i_2}\ge\dots\ge a_{i_n}$, entonces
	\begin{align*}
		c&\ge a_{i_1}-a_{i_n}
		=\sum_{j=1}^{n-1}(a_{i_j}-a_{i_{j+1}})\\
		&\ge\sum_{j=1}^{n-1}\frac{1}{i_j+i_{j+1}}
		\ge\frac{(n-1)^2}{2(i_1+i_2+\dots+i_n)-i_1-i_n}\\
		&>\frac{(n-1)^2}{n(n+1)}
	\end{align*}
	para todo $n>1$ de donde $c\ge 1$.
\end{proof}

\note{Martes\\2022-04-19}

\begin{problem}[Iran TST 2019 Test 1/5]
	Find all functions $f:\RR\to\RR$ such that for all $x,y\in\RR$:
	\[f\left(f(x)^2-y^2\right)^2+f(2xy)^2=f\left(x^2+y^2\right)^2.\]
	\aops{12147602}
\end{problem}

\note[Combinatoria]{}

\begin{theorem}[Hall's Marriage Theorem]
	Let $G$ be a finite bipartite graph with bipartite sets $X$ and $Y$. Then, there is an $X$-perfect matching if and only if
	\[\abs{W}\le\abs{N_G(W)}\]
	for every subset $W$ of $X$.
\end{theorem}

\begin{probEG}
	Tenemos dos superficies congruentes compuestas por $2022$ regiones de área $1$. Demostrar que si superponemos las dos superficies, es posible pinchar las $4044$ regiones en solo $2022$ pinchazos.
\end{probEG}

\begin{probEG}
	En cada fila y columna de un tablero de $n\times n$ hay $k$ fichas. Demuestre que es posible elegir $n$ fichas que están en filas diferentes y columnas diferentes.
\end{probEG}

\begin{probEG}
	En un tablero de $4\times 13$ se colocan $52$ cartas de naipe (una en cada casilla). Demuestre que existen $13$ cartas de números distintos que están en columnas distintas.
\end{probEG}

\begin{probEB}
	En cada casilla de un tablero de $n\times n$ está escrito un $0$ o un $1$ de tal manera que entre cualesquiera $n$ casillas que no están en la misma fila ni en la misma columna, al menos una de ellas tiene escrito el número $1$. Demuestre que existen $i$ filas y $j$ columnas de tal manera que $i+j\ge n+1$ y en las $ij$ casillas de sus intersecciones esté escrito el número $1$.
\end{probEB}

\begin{probEG}
	Sea $\cal X$ un conjunto finito y sean
	\[\bigcup_{i=1}^n X_i={\cal X}\quad\bigcup_{j=1}^n Y_j={\cal X}\]
	dos descomposiciones disjuntas de $\cal X$, de tal manera que los $2n$ subconjuntos $X_i$ y $Y_j$ tengan el mismo número de elementos. Demuestre que podemos elegir $n$ elementos $z_1,z_2,\dots,z_n$ de $\cal X$ de tal manera que estos $n$ elementos estén en diferentes conjuntos en cada descomposición.
\end{probEG}

\begin{probEG}
	En cada casilla de un tablero de $n\times n$ se ha escrito un número entero no negativo de tal manera que la suma de los $n$ números en cada fila y en cada columna es $1$. Demuestre que es posible elegir $n$ números de diferentes filas y diferentes columnas que son positivos.
\end{probEG}

\begin{probEB}
	Para cada conjunto finito $\cal S$ de números enteros positivos, sea $\cal S^\ast$ el conjunto que se obtiene al sumar $2$ a cada elemento de $\cal S$. ¿Para cuántos conjuntos $\cal S$ se cumple que la unión de $\cal S$ con $\cal S^\ast$ es el conjunto de todos los enteros positivos del $1$ al $2022$?
\end{probEB}

\begin{proof}
	La respuesta es $f_{1010}^2$. Pista: recurrencias.
\end{proof}

\note[Geometría]{Miércoles\\2022-04-20}

\begin{probEG}[IberoAmerican 1999/6]
	Sean $A$ y $B$ puntos del plano y $C$ un punto de la mediatriz del segmento $AB$. Se construye una secuencia de puntos $(C_i)$ de la siguiente manera: $C_1=C$ y para $n\ge 1$, si $C_n$ no pertenece al segmento $AB$, entonces $C_{n+1}$ es el circuncentro del triángulo $ABC_n$. Determine todos los puntos $C$ para los cuales la secuencia $(C_n)$ está definida para todo entero positivo $n$ y es eventualmente periódica.
\end{probEG}

\begin{probEG}[IberoAmerican 2002/3]
	Un punto $P$, en el interior de un triángulo equilátero $ABC$, es tal que $\angle APC=120\dg$. Sea $M$ el punto de intersección de las rectas $CP$ y $AB$, y sea $N$ el punto de intersección de las rectas $AP$ y $BC$. Determine el lugar geométrico del circuncentro del triángulo $MBN$ al variar $P$.
\end{probEG}

\begin{proof}
	Respuesta: El segmento que une los puntos medios de los arcos $\widehat{AB}$ y $\widehat{BC}$ de $\odot(ABC)$.
\end{proof}

\begin{probMB}
	Sea $X$ un punto variable sobre el arco $AC$, del circuncírculo del triángulo $ABC$, que no contiene al punto $B$. El punto $Y$ está sobre la prolongación de $BA$ por $A$ y es tal que $AY=AX$, donde $A$ está entre $B$ y $Y$. El punto $Z$ está en la prolongación de $BC$ por $C$ y es tal que $CZ=CX$, donde $C$ está entre $B$ y $Z$. Determine el lugar geométrico de todos los puntos medios del segmento $YZ$ al variar $X$.
\end{probMB}

\begin{proof}
	Si $M$ y $P$ son puntos medios de $AC$ y $YZ$ respectivamente, note que
	\begin{align*}
		4\cdot\abs{MP}^2
		&=\norm{2\cdot\ray{MP}}^2=\norm{\ray{AY}+\ray{CZ}}^2\\
		&=\abs{AY}^2+\abs{CZ}^2+2\cdot\abs{AY}\cdot\abs{CZ}\cdot\cos{\angle ABC}\\
		&=\abs{XA}^2+\abs{XC}^2-2\cdot\abs{XA}\cdot\abs{XC}\cdot\cos{\angle AXC}\\
		&=\abs{AC}^2
	\end{align*}
	de donde $P$ pertenece a la semicircunferencia de diámetro $AC$.
\end{proof}

\note{Viernes\\2022-04-22}

\begin{probEG}
	Let $ABCD$ be a trapezoid with parallel sides $AB>CD$. Points $K$ and $L$ lie on the line segments $AB$ and $CD$, respectively, so that $AK/KB=DL/LC$. Suppose that there are points $P$ and $Q$ on the line segment $KL$ satisfying
	\[\angle APB=\angle BCD\quad\text{and}\quad\angle CQD=\angle ABC.\]
	Prove that the points $P$, $Q$, $B$ and $C$ are concyclic.
\end{probEG}

\begin{proof}
	Pista: semejanza y angulitos.
\end{proof}

\note[Teoría de Números]{}

\begin{probEG}
	Para cada entero positivo $n$, denotemos por $P(n)$ al mayor divisor primo del número $n^2+n+1$. Demuestre que
	\[P(n)>2n+\sqrt{2n}\]
	para infinitos enteros positivos $n$.
\end{probEG}

\begin{proof}
	Pista: demuestra primero que $P(n)>2n$ y luego que $2n<P(n)\le 2n+\sqrt{2n}$ no cumple.
\end{proof}

\begin{probEG}
	Demuestre que existe un $c\in\RR^+$ tal que para cualesquiera $a,b,n\in\ZZ^+$, con $n>1$ y $a!b!\mid n!$, se cumple la desigualdad
	\[a+n<n+c\cdot\ln(n).\]
\end{probEG}

\begin{proof}
	Pista: sale con la identidad
	\[\nu_p(n)=\sum_{i=1}^\infty\floor{\frac{n}{p^i}}=\frac{n-s_p(n)}{p-1}\]
	con $p=3$ (también con $p=2$).
\end{proof}

\begin{probEG}
	Demuestre que para todo $n\in\ZZ^+$,
	\[n!\mid\prod_{k=0}^{n-1}(2^n-2^k).\]
\end{probEG}

\begin{proof}
	Pista: para todo primo $p=2$ y $2\nmid p$, demuestra que el $\nu_p$ del lado izquierdo es menor o igual que el del lado derecho.
\end{proof}

\begin{probEG}
	Demuestre que la ecuación
	\[\frac{1}{10^n}=\frac{1}{n_1!}+\frac{1}{n_2!}+\dots+\frac{1}{n_k!}\]
	no tiene soluciones enteras, con $1\le n_1<n_2<\dots<n_k$.
\end{probEG}

\begin{proof}
	Pista: demuestra que si la ecuación es verdadera, se cumple que $n=\nu_5(n_k!)$.
\end{proof}

\begin{probEG}
	Sea $n>1$ un número entero. Demuestre que
	\[n\nmid 2^{n-1}+1.\]
\end{probEG}

\begin{proof}
	Pista: demuestra que $2^{\nu_2(n-1)+1}\mid\ord_p(2)\mid p-1$ para todo $p\mid n$.
\end{proof}

\begin{probMG}
	Sean $a_1,a_2,\dots,a_n\in\ZZ$. Demuestre que
	\[\prod_{1\le i<j\le n}\frac{a_i-a_j}{i-j}\in\ZZ.\]
\end{probMG}

\begin{probMR}
	Sea $n>1$ un número entero, y sean $1<a_1<a_2<\dots<a_k<n$ enteros positivos tales que $\mcm(a_i,a_j)>n$ para todo $i\ne j$. Demuestre que
	\[\frac{1}{a_1}+\frac{1}{a_2}+\dots+\frac{1}{a_k}<\frac32.\]
\end{probMR}

\begin{proof}
	Pista: considere la cantidad de números divisibles por algún $a_i$ menores o iguales que $n$.
\end{proof}

\note[Simulacro]{Sábado\\2022-04-23}

\begin{probEG}
	Miguel tiene una lista de varios subconjuntos de $10$ elementos de $\{1,2,\dots,100\}$. Él le dice a Cecilia: si eliges cualquier subconjunto de $10$ elementos de $\{1,2,\dots,100\}$, será disjunto con al menos un subconjunto de mi lista.

	¿Cuál es la mínima cantidad de subconjuntos que puede tener la lista de Miguel, si lo que le dice a Cecilia es cierto?
\end{probEG}

\begin{proof}
	La respuesta es $13$ y se puede conseguir con subconjuntos
	\[A_1\cup A_2,A_2\cup A_3,A_3\cup A_1,\]
	\[A_4\cup A_5,A_5\cup A_6,A_6\cup A_4,\]
	\[A_7\cup A_8,A_9\cup A_{10},\dots,A_{19}\cup A_{20}\]
	donde $A_1,A_2,\dots,A_{20}\subset\{1,2,\dots,100\}$ son subconjuntos disjuntos de tamaño $5$. Ahora, si la lista de Miguel tuviera $12$ o menos subconjuntos, podemos añadir algún subconjunto que ya está en la lista, así que supongamos que Miguel tiene $12$ subconjuntos. Como hay $12\times 10=120$ números en la lista, existe un número $x$ que pertenece a al menos $2$ subconjuntos de la lista. Además de esos dos subconjuntos, tenemos $10$ subconjuntos de la lista. Si $x$ pertenece a alguno de esos $10$ subconjuntos, de cada uno de los otros $9$ subconjuntos podemos elegir $9$ números (no necesariamente distintos). De lo contrario, existe un número $y$ que pertenece a al menos $2$ de esos $10$ subconjuntos, y de cada uno de los otros $8$ subconjuntos podemos elegir $8$ números. Es decir, existe un subconjunto $B\subset\{1,2,\dots,100\}$ de tamaño $10$ tal que $B$ no es disjunto con ningún subconjunto de la lista, lo cual es una contradicción.
\end{proof}

\begin{probEG}
	Sean $\Omega_1$, $\Omega_2$, $\Omega_3$ y $\Omega_4$ circunferencias distintas tales que $\Omega_1$ y $\Omega_3$ son tangentes externas en el punto $P$, y $\Omega_2$ y $\Omega_4$ son tangentes externas en el mismo punto $P$. Suponga que $\Omega_1$ y $\Omega_2$, $\Omega_2$ y $\Omega_3$, $\Omega_3$ y $\Omega_4$, $\Omega_4$ y $\Omega_1$ se intersectan en los puntos $A,B,C,D$, respectivamente, y que esos cuatro puntos son distintos de $P$. Demuestre que
	\[\frac{AB\cdot BC}{AD\cdot DC}=\frac{PB^2}{PD^2}.\]
\end{probEG}

\begin{proof}
	Pista: inversión centrada en $P$ de radio $1$.
\end{proof}

\begin{probMR}
	La secuencia $a_0,a_1,a_2,\dots$ está definida de la siguiente manera:
	\[a_0=2\quad\text{y}\quad a_{k+1}=2a_k^2-1,\text{ para }k\ge 0.\]
	Demuestre que si un primo impar $p$ divide a $a_n$, entonces $2^{n+3}$ divide a $p^2-1$.
\end{probMR}

\begin{proof}
	Se puede probar que
	\[a_n=\half\left((2+\sqrt{3})^{2^n}+(2-\sqrt{3})^{2^n}\right)=\sum_{i=0}^{2^{n-1}}\binom{2^n}{2k}2^{2k}3^{2^{n-1}-k}\]
	para todo $n\ge 0$.
\end{proof}

\section{Semana 7 (04/25 -- 05/01)}

\note[Álgebra]{Lunes\\2022-04-25}

\begin{probEG}
  Determine todas las parejas $(a,b)$ de números reales tales que
  \[a\floor{bn}=b\floor{an},\]
  para todo entero positivo $n$.
\end{probEG}

\begin{proof}
  Respuesta: $a,b\in\ZZ$ o $(a,b)=(0,k),(k,0),(k,k)$ donde $k\in\RR$.
\end{proof}

\begin{probEG}[Indonesia MO Shortlist 2014/A1]
  Sean $a$ y $b$ números reales positivos tales que
  $\floor{a^k}+\floor{b^k}=\floor{a}^k+\floor{b}^k$ para infinitos enteros
  positivos $k$. Demuestre que
  \[\floor{a^{2014}}+\floor{b^{2014}}=\floor{a}^{2014}+\floor{b}^{2014}.\]
  \begin{hint}
    Let $a,b$ be positive real numbers such that there exist infinite number of
    natural numbers $k$ such that
    $\floor{a^k}+\floor{b^k}=\floor{a}^k+\floor{b}^k$. Prove that
    \[\floor{a^{2014}}+\floor{b^{2014}}=\floor{a}^{2014}+\floor{b}^{2014}.\]
  \end{hint}
  \forum[aops]{12396258}
\end{probEG}

\begin{proof}
  Note que $a^k\ge\floor{a}^k$ de donde $\floor{a^k}\ge\floor{a}^k$ para todo
  $k\in\ZZ^+$. Luego, se cumple la igualdad para infinitos $k\in\ZZ^+$ de donde
  $a,b\in\ZZ^+\cup(0,1)$. Finalmente,
  $\floor{a^k}+\floor{b^k}=\floor{a}^k+\floor{b}^k$ para todo $k\in\ZZ^+$.
\end{proof}

\begin{probEG}
  Determine todas las parejas $(a,b)$ de números reales tales que
  \[\floor{a\floor{bn}}=n-1,\]
  para todo entero positivo $n$.
\end{probEG}

\begin{proof}
  Si $a>0$ tenemos que $-1\le n(ab-1)<a$ y si $a<0$ tenemos que $0>n(ab-1)>a-1$
  de donde $n\abs{ab-1}$ es acotado. Por ende, $ab=1$ y $a>0$.
\end{proof}

\begin{probMG}
  Sea $n\ge 2$ un número entero. Los $n$ conjuntos finitos $A_1,A_2,\dots,A_n$
  satisfacen:
  \[\abs{A_i\,\triangle\,A_j}=\abs{i-j},\quad\forall\,i,j=1,2,\dots,n.\]
  Determine el mínimo valor de
  \[\sum_{i=1}^n\abs{A_i}.\]
\end{probMG}

\begin{proof}
  La respuesta es $\floor{n^2/4}$ y un ejemplo es cuando $A_i$ es el conjunto de
  todos los enteros $x\ne\floor{\frac{n+1}{2}}$ que están entre $i$ y
  $\floor{\frac{n+1}{2}}$ (tendremos $\abs{A_i}=\abs{\floor{\frac{n+1}{2}}-i}$
  en este caso). Ahora, note que
  \[\abs{A_i}+\abs{A_{n+1-i}}\ge\abs{A_i\,\triangle\,A_{n+1-i}}=\abs{n+1-2i}\]
  para todo $1\le i\le n$ de donde
  \[
    \sum_{i=1}^n\abs{A_i}
    =\half\sum_{i=1}^n(\abs{A_i}+\abs{A_{n+1-i}})
    \ge\half\sum_{i=1}^n\abs{n+1-2i}
    =\floor{\frac{n^2}{4}}.
  \]
\end{proof}

\begin{problem}
  Sea $\alpha\ge 1$ un número real y sea $n$ un entero positivo tal que
  \[\floor{\alpha^{n+1}},\floor{\alpha^{n+2}},\dots,\floor{\alpha^{4n}}\]
  son todos cuadrados perfectos. Demuestre que $\floor{\alpha}$ es un cuadrado
  perfecto.
\end{problem}

\note[Combinatoria]{Martes\\2022-04-26}

\begin{probEG}
  Sea $A=(a_1,a_2,\dots,a_{2001})$ una secuencia de enteros positivos. Sea $m$
  el número de subsecuencias de tres términos $(a_i,a_j,a_k)$ tales que
  $a_k=a_j+1$ y $a_j=a_i+1$. Considerando todas las secuencias $A$, determine el
  mayor valor de $m$.
\end{probEG}

\begin{proof}
  Respuesta: $667^3$.
\end{proof}

\begin{probEG}
  Para $i=1,2,\dots,11$, sea $M_i$ un conjunto de $5$ elementos, y asuma que
  para $1\le i<j\le 11$, $M_i\cap M_j\ne\varnothing$. Sea $m$ el mayor número
  para el cual existen $m$ conjuntos $M_{x_1},M_{x_2},\dots,M_{x_m}$ tales que
  $M_{x_1}\cap M_{x_2}\cap\dots\cap M_{x_m}\ne\varnothing$. Determine el mínimo
  valor de $m$ sobre todos los posibles conjuntos iniciales.
\end{probEG}

\begin{proof}
  Respuesta: $4$.
\end{proof}

\begin{probMR}[%
    High-School Mathematics 1994/1, China \cite{ref:titu-102}\protect\footnote{%
      Un video de \href{https://www.youtube.com/c/3blue1brown}{3Blue1Brown}
      cubriendo este problema:
      \url{https://www.youtube.com/watch?v=bOXCLR3Wric}.%
    }%
  ]
  Determine cuantos subconjuntos de $\{1,2,\dots,2000\}$ tienen suma de elemento
  múltiplo de $5$.
  \begin{hint}
    Find the number of subsets of $\{1,\dots,2000\}$, the sum of whose elements
    is divisible by $5$.
  \end{hint}
\end{probMR}

\begin{proof}
  Sea
  \[P(x)=\prod_{i=1}^{2000}(x^i+1)=\sum_{j=0}^{\deg P}c_jx^j\]
  un polinomio y sea $\zeta=e^{2\pi i/5}$ una raíz compleja de la ecuación
  $z^5=1$. Luego, $1+\zeta^1+\zeta^2+\zeta^3+\zeta^4=0$ y
  \[
    2^{2000}+4\cdot 2^{400}
    =\sum_{i=0}^4 P(\zeta^i)
    =\sum_{j=0}^{\deg P}c_j\cdot\sum_{i=0}^4\zeta^{ij}
    =5\sum_{5\,\mid\,j}c_j.
  \]
  Por lo tanto, la respuesta es $\dfrac{2^{2000}+4\cdot 2^{400}}{5}$.
\end{proof}

\begin{probEG}[MOSP 1999]
  Sea $X$ un conjunto finito y no vacío de enteros positivos, y sea $A$ un
  subconjunto de $X$. Demuestre que existe $B\subset X$ tal que $A$ es el
  conjunto de todos los elementos de $X$ que dividen a un número impar de
  elementos de $B$.
  \begin{hint}
    Let $X$ be a finite set of positive integers and $A$ a subset of $X$. Prove
    that there exists a subset $B$ of $X$ such that $A$ equals the set of
    elements of $X$ which divide an odd number of elements of $B$.
  \end{hint}
\end{probEG}

\begin{proof}
  Nos basta probar que la función $f:X\to X$ tal que
  \[f(A)=\{x\in X:x\text{ divide a un número impar de elementos de }A\}\]
  es biyectiva. Primero, probaremos que $f$ es inyectiva y para eso supongamos
  que existen $A,B\subset X$ tales que $f(A)=f(B)$. Si $A\ne B$, sea $x$ el
  mayor elemento de $A\,\triangle\,B$ donde $x\in A$ sin pérdida de generalidad.
  Luego, $x$ divide a un número más del conjunto $A$ que de $B$, de donde
  $x\in f(A)\,\triangle\,f(B)=\varnothing$, lo cual es un absurdo. Por lo tanto,
  $f$ es inyectiva y por ende biyectiva pues $X$ es finito.
\end{proof}

\note[Geometría]{Miércoles\\2022-04-27}

\begin{probEG}
  Las reflexiones de la diagonal $BD$ de un cuadrilátero convexo $ABCD$ (el cual
  no tiene lados iguales), con respecto a las bisectrices de los ángulos
  interiores $\angle B$ y $\angle D$, pasan por el punto medio del lado $AC$.
  Demuestre que las reflexiones de la diagonal $AC$ con respecto a las
  bisectrices interiores $\angle A$ y $\angle C$, pasan por el punto medio de
  $BD$.
\end{probEG}

\begin{probEG}
  Sea $ABC$ un triángulo. Sobre los lados $BC$, $CA$ y $AB$ se toman los puntos
  $X$, $Y$ y $Z$, respectivamente, tales que las rectas $AX$, $BY$ y $CZ$ son
  concurrentes y $AY=AZ$. La bisectriz interior del ángulo $\angle BAC$ corta a
  la recta $XZ$ en el punto $T$. Si $M$ y $N$ son los puntos medios de los lados
  $BC$ y $CA$, respectivamente, demuestre que los puntos $M$, $N$ y $T$ son
  colineales.
\end{probEG}

\begin{probEG}
  Sea $ABC$ un triángulo y sea $A'B'C'$ la reflexión de $ABC$ con respecto a un
  punto cualquiera $P$ del plano. Demuestre que los circuncírculos de $AB'C'$,
  $BC'A'$ y $CA'B'$ pasan por un mismo punto en el circuncírculo del triángulo
  $ABC$.
\end{probEG}

\begin{probEG}
  Sean $P$ y $Q$ dos puntos distintos sobre el circuncírculo del triángulo $ABC$
  de tal manera que las rectas de Simson de $P$ y $Q$ se intersectan
  perpendicularmente en el punto $X$. Demuestre que $X$ es un punto de la
  circunferencia de los nueve puntos del triángulo $ABC$.
\end{probEG}

\begin{probEG}
  Sea $\Omega$ el circuncírculo de un triángulo escaleno $ABC$. Las rectas
  tangentes a $\Omega$ que pasan por $B$ y $C$ se intersectan en el punto $Q$.
  Sea $P$ un punto sobre la semirecta $BC$ de tal manera que $AP$ y $AQ$ son
  perpendiculares. Los puntos $D$ y $E$ están sobre la semirecta $PQ$, con $E$
  entre $D$ y $P$, y son tales que $DQ=BQ=EQ$. Demuestre que los triángulos
  $ABC$ y $ADE$ son semejantes.
\end{probEG}

\begin{probEG}
  Sea $ABCD$ un rectángulo. Los puntos $P$, $Q$, $R$ y $S$ pertenecen a los
  lados $AB$, $BC$, $CD$ y $DA$, respectivamente. Demuestre que el perímetro de
  $PQRS$ es mayor o igual que el doble de la diagonal de $ABCD$.
\end{probEG}

\begin{probEG}
  Sea $\Gamma$ una circunferencia y sea $P$ un punto en su exterior. Las
  tangentes por $P$ a $\Gamma$ tocan a la circunferencia en $A$ y $B$. Sea $M$
  el punto medio del segmento $AB$. La mediatriz de $AM$ intersecta a $\Gamma$
  en un punto $C$ del interior del triángulo $ABP$. La recta $AC$ corta a $PM$
  en $G$, y $PM$ corta a $\Gamma$ en un punto $D$ exterior al triángulo $ABP$.
  Si $BD\parallel AC$, demuestre que $G$ es el baricentro del triángulo $ABP$.
\end{probEG}

\begin{probEG}
  Determine el mayor entero positivo $k$ para el cual existe un polígono
  $P_1P_2\cdots P_{2015}$ tal que exactamente $k$ de los cuadriláteros
  $P_iP_{i+1}P_{i+2}P_{i+3}$, con $i=1,2,\dots,2015$ (con índices en el módulo
  $2015$) tienen una circunferencia inscrita.
\end{probEG}

\begin{proof}
  Respuesta: $1007$. Se puede probar que no existen $\abs{i_1-i_2}=1$ tales que
  $P_iP_{i+1}P_{i+2}P_{i+3}$ tenga una circunferencia inscrita para $i=i_1,i_2$.
  Ahora, si $k=2007$, el polígono debe ser cíclico y se puede probar que existe
  uno que cumple.
\end{proof}

\note[Teoría de Números]{Viernes\\2022-04-29}

\begin{probEG}[ISL 2001/N1]
  Prove that there is no positive integer $n$ such that, for $k=1,2,\dots,9$,
  the leftmost digit (in decimal notation) of $(n+k)!$ equals $k$.
\end{probEG}

\begin{probMR}[ISL 2001/N3]
  Let $a_1=11^{11},a_2=12^{12},a_3=13^{13}$, and
  \[a_n=\abs{a_{n-1}-a_{n-2}}+\abs{a_{n-2}-a_{n-3}},\quad\forall\,n\ge 4.\]
  Determine $a_{14^{14}}$.
\end{probMR}

\begin{proof}
  Respuesta: $1$. Primero, demuestre que $(\abs{a_{i+1}-a_i})_{i\ge 1}$ es
  decreciente y que $a_{7k}$ es impar para todo $k\in\ZZ^+$.
\end{proof}

\begin{problem}[ISL 2001/N4]
  Let $p\ge 5$ be a prime number. Prove that there exists an integer $a$ with
  $1\le a\le p-2$ such that neither $a^{p-1}-1$ nor $(a+1)^{p-1}-1$ is divisible
  by $p^2$.
\end{problem}

\begin{problem}[ISL 2001/N5]
  Let $a>b>c>d$ be positive integers and suppose
  \[ac+bd=(b+d+a-c)(b+d-a+c).\]
  Prove that $ab+cd$ is not prime.
\end{problem}

\begin{problem}[ISL 2001/N6]
  Is it possible to find $100$ positive integers not exceeding $25000$, such
  that all pairwise sums of them are different?
\end{problem}

\section{Semana 8 (05/02 -- 05/08)}

\note[Álgebra]{Lunes\\2022-05-02}

\begin{probEG}[IMO Shortlist 2001 A2]
	Sea $(a_n)_{n\ge 0}$ una secuencia infinita de números reales positivos. Demuestre que
	\[1+a_n>a_{n-1}\sqrt[n]{2}\]
	para infinitos $n\in\ZZ^+$.
\end{probEG}

\begin{probMR}[IMO Shortlist 2001 A3]
	Sean $x_1,x_2,\dots,x_n\in\RR$. Demuestre que
	\[\frac{x_1}{1+x_1^2}+\frac{x_2}{1+x_1^2+x_2^2}+\dots+\frac{x_n}{1+x_1^2+\dots+x_n^2}<\sqrt{n}.\]
\end{probMR}

\note[Combinatoria]{Martes\\2022-05-03}

\begin{probEG}[IMO Shortlist 2001 C3]
	Define a \emph{k-clique} to be a set of $k$ people such that every pair of them are acquainted with each other. At a certain party, every pair of $3$-cliques has at least one person in common, and there are no $5$-cliques. Prove that there are two or fewer people at the party whose departure leaves no $3$-clique remaining.
\end{probEG}

\begin{probMB}[IMO Shortlist 2001 C4]
	A set of three nonnegative integers $\{x,y,z\}$ with $x<y<z$ is called historic if $\{z-y,y-x\}=\{1776,2001\}$. Show that the set of all nonnegative integers can be written as the union of pairwise disjoint historic sets.
\end{probMB}

\note[Geometría]{Miércoles\\2022-05-04}

\begin{probEG}[IMO Shortlist 2001 G1]
	Let $A_1$ be the center of the square inscribed in acute triangle $ABC$ with two vertices of the square on side $BC$. Thus one of the two remaining vertices of the square is on side $AB$ and the other is on $AC$. Points $B_1$, $C_1$ are defined in a similar way for inscribed squares with two vertices on sides $AC$ and $AB$, respectively. Prove that lines $AA_1$, $BB_1$, $CC_1$ are concurrent.
	\aops{119194}
\end{probEG}

\begin{probMG}[IMO Shortlist 2001 G2]
	In acute triangle $ABC$ with circumcenter $O$ and altitude $AP$, $\angle C\ge\angle B+30\dg$. Prove that $\angle A+\angle COP<90\dg$.
\end{probMG}

\begin{probMR}[IMO Shortlist 2001 G3]
	Let $ABC$ be a triangle with centroid $G$. Determine, with proof, the position of the point $P$ in the plane of $ABC$ such that $AP\cdot AG+BP\cdot BG+CP\cdot CG$ is a minimum, and express this minimum value in terms of the side lengths of $ABC$.
\end{probMR}

\note[Teoría de Números]{Viernes\\2022-05-06}

\begin{probEG}
	Demuestre que todo entero positivo coprimo con $3$ posee un múltiplo cuya suma de dígitos es un número primo.
\end{probEG}

\begin{proof}
	Como $n$ es coprimo con $3$, existe $m=\ol{a_{k-1}a_{k-2}\cdots a_1a_0}$ tal que $m$ es coprimo con $30$ y $n\mid 10^\alpha\cdot m$ para algún $\alpha\in\ZZ^+$. Luego,
	\[m\cdot(10^t-1)=\ol{a_{k-1}a_{k-2}\cdots a_1(a_0-1)\underbrace{99\cdots 9}_{t-k\text{ veces}}(9-a_{k-1})(9-a_{k-2})\cdots(9-a_1)(10-a_0)}\]
	de donde $S(m\cdot(10^t-1))=9t$ para todo $t>k$. Por ende, existe un $m\mid M$ tal que $S(M)$ es coprimo con $S(m)$ y por Dirichlet, la suma de dígitos de $\ol{\underbrace{(M)(M)\cdots (M)}_{\beta\text{ veces}}(m)\underbrace{00\cdots 0}_{\alpha\text{ veces}}}$ es un primo para algún $\beta\in\ZZ^+$. Con esto se termina la prueba, pues $M$ y $m$ son múltiplos de $n$.
\end{proof}

\section{Semana 9 (05/09 -- 05/15)}

\note[Álgebra]{Lunes\\2022-05-09}

\begin{probMG}[ISL 2001/A4]
	Find all functions $f:\RR\to\RR$, satisfying
	\[f(xy)(f(x)-f(y))=(x-y)f(x)f(y)\]
	for all $x,y$.
\end{probMG}

\begin{proof}
	Con $(x,y)=(1,0)$ y $(x,1)$ tenemos que $f(0)=0$ y que $f(x)=0$ o $f(x)=xf(1)$. Luego, $f(x)=xc$ para todo $x\in G$, y $f(x)=0$ para todo $x\not\in G$ donde $(G,\cdot)$ es un grupo multiplicativo y $c$ es una constante.
\end{proof}

\begin{probMB}[ISL 2001/A5]
	Find all positive integers $a_1,a_2,\dots,a_n$ such that
	\[\frac{99}{100}=\frac{a_0}{a_1}+\frac{a_1}{a_2}+\dots+\frac{a_{n-1}}{a_n},\]
	where $a_0=1$ and $(a_{k+1}-1)a_{k-1}\ge a_k^2(a_k-1)$ for $k=1,2,\dots,n-1$.
\end{probMB}

\note[Combinatoria]{Martes\\2022-05-10}

\begin{probEG}[ISL 1998/N3]
	Determine the smallest integer $n$, $n\ge 4$, for which one can choose four different numbers $a,b,c,d$ from any $n$ distinct integers such that $a+b-c-d$ is divisible by $20$.
	\forum[aops]{124432}
\end{probEG}

\begin{probER}
	Form a $2000\times 2002$ screen with unit screens. Initially, there are more than $1999\times 2001$ unit screens which are \emph{on}. In any $2\times 2$ screen, as soon as there are $3$ unit screens which are \emph{off}, the $4^\text{th}$ screen turns off automatically. Prove that the whole screen can never be totally off.
\end{probER}

\begin{proof}
	Pista: considere a la cantidad de cuadrados de $2\times 2$.
\end{proof}

\begin{probMR}
	Given an initial sequence $a_1,a_2,\dots,a_n$ of real numbers, we perform a series of steps. At each step, we replace the current sequence $x_1,x_2,\dots,x_n$ with $\abs{x_1-a},\abs{x_2-a},\dots,\abs{x_n-a}$ for some $a$. For each step, the value of $a$ can be different.
	\begin{enumerate}[(a)]
		\ii Prove that it is always possible to obtain the null sequence consisting of all $0$'s.
		\ii Determine with proof the minimum number of steps required, regardless of initial sequence, to obtain the null sequence.
	\end{enumerate}
\end{probMR}

\note[Teoría de Números]{Miércoles\\2022-05-11}

\begin{probEG}
	Sean $1\le a_1,a_2,\dots,a_{10^6}\le 9$ algunos dígitos. Demuestre que el conjunto
	\[A=\left\{\ol{a_1a_2\cdots a_k}:1\le k\le 10^6\right\}\]
	contiene a lo más $100$ cuadrados perfectos.
\end{probEG}

\begin{proof}
	Por el absurdo, supongamos que existen más de $100$ índices $k$ tales que $\ol{a_1a_2\cdots a_k}$ es un cuadrado perfecto. Entonces, hay $50$ de la misma paridad. Como $2^{50}>2^{20}=1024^2>1000^2=10^6$, existen índices $j<i<2j$ tales que $2\mid i-j$ y
	\begin{align*}
		\ol{a_1a_2\cdots a_i}
		&=\ol{a_1a_2\cdots a_j}\cdot 10^{i-j}+\ol{a_{j+1}a_{j+2}\cdots a_i}\\
		&=\left(n\cdot 10^\frac{i-j}{2}\right)^2+\ol{a_{j+1}a_{j+2}\cdots a_i}\\
		&\ge\left(n\cdot 10^\frac{i-j}{2}+1\right)^2
	\end{align*}
	de donde
	\[10^{i-j}>\ol{a_{j+1}a_{j+2}\cdots a_i}>2n\cdot 10^\frac{i-j}{2}>10^\frac{i-1}{2}\]
	y $i\ge 2j$, lo cual es un absurdo.
\end{proof}

\note{Viernes\\2022-05-13}

\begin{probMG}[IberoAmerican 2012/6]
	Demostrar que, para todo entero positivo $n$, existen $n$ enteros positivos consecutivos tales que ninguno es divisible por la suma de sus respectivos dígitos.
	\begin{hint}
		Show that, for every positive integer $n$, there exist $n$ consecutive positive integers such that none is divisible by the sum of its digits.
	\end{hint}
	\forum[aops]{2814652}
	\forum[oma]{1201}
\end{probMG}

\begin{proof}
	Procederemos por inducción sobre $n$. Si $n=1$, es claro que $s(11)\nmid 11$ donde $s(x)$ denota la suma de dígitos de $x$. Ahora, supongamos que existe un $m\in\ZZ^+_0$ tal que $s(m+i)\nmid m+i$ para todo $1\le i\le n$. Sea $m'=M\cdot 10^k+m$ donde $M$ es un número formado por $N!$ bloques consecutivos de $N!$, siendo $k,N\in\ZZ^+$ suficientemente grandes. Si $s(m'+i)\mid m'+i$ para algún $1\le i\le n$, entonces
	\[s(m+i)\mid N!\cdot s(N!)+s(m+i)=s(m'+i)\mid m'+i=M\cdot 10^k+m+i\]
	de donde $s(m+i)\mid m+i$, absurdo. Si $s(m'+n+1)\mid m'+n+1$ para $k$ y $k+1$, entonces $s(M)+s(m+n+1)$ divide simultáneamente a $M\cdot 10^k+m+n+1$ y a $M\cdot 10^{k+1}+m+n+1$. Luego,
	\[N!\cdot s(N!)+s(m+n+1)\mid 9(m+n+1)\]
	lo cual no es posible para un valor grande de $N$. Por lo tanto, existe un $m'\in\ZZ^+_0$ tal que $s(m'+i)\nmid m'+i$ para todo $1\le i\le n+1$.
\end{proof}

\begin{probEG}
	Hallar la menor cantidad de dígitos de un número $N$ que contiene a todas las permutaciones de $1234$.
\end{probEG}

\begin{proof}
	Vamos a dividir las $24$ permutaciones en $6$ ciclos. Digamos que dos bloques de $4$ son \emph{adyacentes} si esos bloques son de la forma $abcd$ y $bcde$ donde los dígitos $a,b,c,d,e$ son consecutivos. Si dos bloques son de distintos ciclos, esos bloques no pueden ser adyacentes. Luego, si $t$ es la cantidad de bloques que no son permutaciones de $1234$, entonces $n-3\ge 24+t$. Como existen a lo sumo $t+1$ ciclos, entonces $t+1\ge 6$ de donde $n\ge 27+t\ge 32$, pero es fácil ver que no se da la igualdad. Un ejemplo con $33$ es \[N=123412314231243121342132413214321.\]
\end{proof}

\begin{probEG}
	Un número es \emph{ascendente} si sus dígitos están en orden creciente, de izquierda a derecha. Por ejemplo, $123$ y $11224$ son ascendentes. Demuestre que para todo entero positivo $n$, es posible encontrar un cuadrado perfecto de exactamente $n$ dígitos que sea ascendente.
\end{probEG}

\begin{proof}
	Note que
	\begin{align*}
		2^2&=4\\
		17^2&=289\\
		167^2&=27889\\
		1667^2&=2778889\\
		&\vdotswithin{=}
	\end{align*}
	y
	\begin{align*}
		4^2&=16\\
		34^2&=1156\\
		334^2&=111556\\
		3334^2&=11115556\\
		&\vdotswithin{=}
	\end{align*}
	así que con esto es suficiente.
\end{proof}

\begin{probEG}
	Demostrar que para cualquier entero positivo $N$, al menos uno de los números $N$ y $N+1$ se puede representar de la forma $k+s(k)$ para algún entero positivo $k$.
\end{probEG}

\begin{proof}
	Note que si $k=\ol{(a-1)\underbrace{99\dots 9}_{t\text{ veces}}}$ donde $1\le a\le 9$ y $t\ge 0$ son enteros, tenemos que
	\begin{align*}
		k+s(k)&=a\cdot 10^t+s(a)+9t-2\\
		k+1+s(k+1)&=a\cdot 10^t+s(a)
	\end{align*}
	y por inducción,
	\[\bigcup_{k=1}^\infty\{k+s(k)-1,k+s(k)\}=\ZZ^+.\]
\end{proof}

\section{Semana 10 (05/16 -- 05/22)}

\note[Álgebra]{Lunes\\2022-05-16}

\begin{probEG}[IMO Shortlist 2002/A2]
	Let $a_1,a_2,\dots$ be an infinite sequence of real numbers, for which there exists a real number $c$ with $0\le a_i\le c$ for all $i$, such that
	\[\abs{a_i-a_j}\ge\frac{1}{i+j}\quad\text{for all }i,j\text{ with }i\ne j.\]
	Prove that $c\ge 1$.
	\forum[aops]{118699}
\end{probEG}

\begin{probMR}[IMO Shortlist 2002/A3]
	Let $P$ be a cubic polynomial given by $P(x)=ax^3+bx^2+cx+d$, where $a,b,c,d$ are integers and $a\ne 0$. Suppose that $xP(x)=yP(y)$ for infinitely many pairs $x,y$ of integers with $x\ne y$. Prove that the equation $P(x)=0$ has an integer root.
	\forum[aops]{118702}
\end{probMR}

\begin{probMB}[IMO Shortlist 2002/A4]
	Find all functions $f$ from the reals to the reals such that
	\[\left(f(x)+f(z)\right)\left(f(y)+f(t)\right)=f(xy-zt)+f(xt+yz)\]
	for all real $x,y,z,t$.
	\forum[aops]{118703}
\end{probMB}

\note[Combinatoria]{Martes\\2022-05-17}

\begin{probEG}[IMO Shortlist 2002/C1]
	Let $n$ be a positive integer. Each point $(x,y)$ in the plane, where $x$ and $y$ are non-negative integers with $x+y<n$, is coloured red or blue, subject to the following condition: if a point $(x,y)$ is red, then so are all points $(x',y')$ with $x'\le x$ and $y'\le y$. Let $A$ be the number of ways to choose $n$ blue points with distinct $x$-coordinates, and let $B$ be the number of ways to choose $n$ blue points with distinct $y$-coordinates. Prove that $A=B$.
	\forum[aops]{118710}
\end{probEG}

\begin{probEG}[IMO Shortlist 2002/C2]
	For $n$ an odd positive integer, the unit squares of an $n\times n$ chessboard are coloured alternately black and white, with the four corners coloured black. A \emph{tromino} is an $L$-shape formed by three connected unit squares. For which values of $n$ is it possible to cover all the black squares with non-overlapping trominos? When it is possible, what is the minimum number of trominos needed?
	\forum[aops]{118712}
\end{probEG}

\begin{proof}
	Respuesta: $n\ge 7$ y el mínimo es $\left(\frac{n+1}{2}\right)^2$.
\end{proof}

\begin{probMR}[IMO Shortlist 2002/C3]
	Let $n$ be a positive integer. A sequence of $n$ positive integers (not necessarily distinct) is called \emph{full} if it satisfies the following condition: for each positive integer $k\ge 2$, if the number $k$ appears in the sequence then so does the number $k-1$, and moreover the first occurrence of $k-1$ comes before the last occurrence of $k$. For each $n$, how many full sequences are there?
	\forum[aops]{118714}
\end{probMR}

\begin{proof}
	Respuesta: $n!$, considerando una biyección.
\end{proof}

\note[Teoría de Números]{}

\begin{probEG}[Russia 2022]
	Demuestre que existe un entero positivo $b$ que tiene la siguiente propiedad: para todo entero $n>b$, la suma de los dígitos del número $n!$ es mayor o igual que $10^{100}$.
\end{probEG}

\begin{proof}
	Note que si $n>b$, entonces $b\mid n!$. Además, $s(x-y)\ge s(x)-s(y)$ para todo $x,y\in\ZZ^+$. Sea $b=\underbrace{99\cdots 9}_{N\text{ veces}}$ donde $N\in\ZZ^+$ es suficientemente grande, y sea $M\cdot 10^k\cdot b$ un múltiplo de $b$ donde $10\nmid M$. Luego,
	\begin{align*}
		s(M\cdot 10^k\cdot b)
		&=s(M\cdot(10^N-1))\\
		&=s\left(\ol{(M-1)\underbrace{99\cdots 9}_{N\text{ veces}}}-(M-1)\right)\\
		&\ge s(M-1)+9N-s(M-1)\\
		&=9N
	\end{align*}
	de donde $s(n!)\ge 9N>10^{100}$ para todo $n>b$.
\end{proof}

\note[Geometría]{Miércoles\\2022-05-18}

\begin{probEG}[IMO Shortlist 2001/G5]
	Let $ABC$ be an acute triangle. Let $DAC$, $EAB$, and $FBC$ be isosceles triangles exterior to $ABC$, with $DA=DC$, $EA=EB$, and $FB=FC$, such that
	\[\angle ADC=2\angle BAC,\quad\angle BEA=2\angle ABC,\quad\angle CFB=2\angle ACB.\]
	Let $D'$ be the intersection of lines $DB$ and $EF$, let $E'$ be the intersection of $EC$ and $DF$, and let $F'$ be the intersection of $FA$ and $DE$. Find, with proof, the value of the sum
	\[\frac{DB}{DD'}+\frac{EC}{EE'}+\frac{FA}{FF'}.\]
	\forum[aops]{119201}
\end{probEG}

\begin{proof}
	Si $P$ es un punto interior de $ABC$ tal que
	\[\angle CPA=\pi-\angle A,\quad\angle APB=\pi-\angle B,\quad\angle BPC=\pi-\angle C\]
	entonces $P$ es la intersección de circunferencias $\odot(D,DA),\odot(E,EB),\odot(F,FC)$ de donde
	\[\triangle PDE\equiv\triangle ADE,\quad\triangle PEF\equiv\triangle BEF,\quad\triangle PFD\equiv\triangle CFD.\]
	Luego,
	\[\cycsum\frac{DB}{DD'}=\cycsum\frac{\abs{DEBF}}{\abs{DEF}}=3+\frac{\abs{ADE}+\abs{BEF}+\abs{CFD}}{\abs{PDE}+\abs{PEF}+\abs{PFD}}=4.\]
\end{proof}

\begin{probEG}[IMO Shortlist 2001/G6]
	Let $ABC$ be a triangle and $P$ an exterior point in the plane of the triangle. Suppose $AP,BP,CP$ meet the sides $BC,CA,AB$ (or extensions thereof) in $D,E,F$, respectively. Suppose further that the areas of triangles $PBD,PCE,PAF$ are all equal. Prove that each of these areas is equal to the area of triangle $ABC$ itself.
	\forum[aops]{119203}
\end{probEG}

\begin{proof}
	Sin pérdida de generalidad, supongamos que $P$ pertenece al lado opuesto de $A$ con respecto a $BC$. Si $C$ está entre $B$ y $D$, entonces $\triangle PBD$ contiene al $\triangle PCE$. Si $B$ está entre $C$ y $D$, entonces $\triangle PAF$ contiene al $\triangle PBD$. Por ende, $D$ está en el lado $BC$. Si $A$ está entre $E$ y $C$, entonces $\triangle PCE$ contiene al $\triangle PBD$. Si $B$ está entre $F$ y $A$, entonces $\triangle PAF$ contiene al $\triangle PBD$. Es decir, los segmentos dirigidos $\frac{PA}{PD},\frac{PB}{PE},\frac{PC}{PF}$ tienen signos $+,-,+$, respectivamente. Note que
	\[\frac{\abs{PBD}}{\abs{ABC}}=\frac{BD}{BC}\cdot\frac{PD}{AD}\quad\text{y}\quad\frac{BD}{BC}\cdot\frac{FC}{FP}\cdot\frac{AP}{AD}=1\]
	considerando áreas y segmentos con signos. Es decir,
	\[\left(\frac{PC}{PF}-1\right)\cdot\frac{PA}{PD}=\frac{\abs{ABC}}{\abs{PBD}}=k\]
	donde $k\in\RR^-$ es una constante. Con un poco de cálculo obtenemos
	\[(x(x-1)-k)(k+1)=0\]
	donde $x=\frac{PA}{PD}\ge 1$. Por lo tanto, $k=-1$.
\end{proof}

\begin{probEG}[IMO Shortlist 2001/G7]
	Let $O$ be an interior point of acute triangle $ABC$. Let $A_1$ lie on $BC$ with $OA_1$ perpendicular to $BC$. Define $B_1$ on $CA$ and $C_1$ on $AB$ similarly. Prove that $O$ is the circumcenter of $ABC$ if and only if the perimeter of $A_1B_1C_1$ is not less than any one of the perimeters of $AB_1C_1$, $BC_1A_1$, and $CA_1B_1$.
	\forum[aops]{119204}
\end{probEG}

\begin{proof}
	Digamos que $O'\ne O$ es un punto interior de $\triangle ABC$, siendo $O$ el circuncentro de $ABC$. Si $O'$ pertenece en el interior del $\triangle BOC$, tenemos que
	\[\angle BC_1A_1=\angle BO'A_1=\frac{\pi}{2}-\angle A_1BO'\ge\frac{\pi}{2}-\angle CBO=\angle A\]
	y análogamente $\angle CB_1A_1\ge\angle A$, de donde $A_1B_1+A_1C_1\le AB_1+AC_1$. Es decir, $p(A_1B_1C_1)<p(AB_1C_1)$.
\end{proof}

\begin{probEG}[IMO Shortlist 2001/G8]
	Let $ABC$ be a triangle with $\angle BAC=60\dg$. Let $AP$ bisect $\angle BAC$ and let $BQ$ bisect $\angle ABC$, with $P$ on $BC$ and $Q$ on $AC$. If $AB+BP=AQ+QB$, what are the angles of the triangle?
	\forum[aops]{119207}
\end{probEG}

\begin{proof}
	Sea $R$ un punto en la recta $AB$ tal que $BR=BP$ y $B$ está entre $R$ y $A$. Si $S$ es un punto del rayo $AC$ tal que $AR=AS$, entonces
	\[QS=AS-AQ=AR-AQ=AB+BR-AQ=AB+BP-AQ=QB\]
	y
	\[\angle QSP=\angle ASP=\angle ARP=\half\angle ABP=\angle QBP.\]
	Como $BPSQ$ no es un paralelogramo, $QP$ es bisectriz de $\angle BQC$. Es decir,
	\[\frac{BQ}{QC}=\frac{BP}{PC}=\frac{BA}{AC}=\frac{c}{b}\]
	de donde $BQ=\frac{ca}{c+a}$. Además, como $\angle A=60\dg$, tenemos que
	\[\left(\frac{ca}{c+a}\right)^2=BQ^2=AB^2+AQ^2-AB\cdot AQ=c^2+\left(\frac{bc}{c+a}\right)^2-\frac{bc^2}{c+a}\]
	y
	\[a^2=BC^2=AB^2+AC^2-AB\cdot AC=b^2+c^2-bc.\]
	Simplificando nos queda que $a=b-2c$, y de esto no es difícil ver que los ángulos $\angle A,\angle B,\angle C$ son $60\dg,105\dg,15\dg$.
\end{proof}

\note[Punto\\Humpty \cite{ref:humpty}]{Jueves\\2022-05-19}

\begin{probEG}
	In $\triangle ABC$ with orthocenter $H$, the circle with diameter $\ol{AH}$ and $\odot(BHC)$ intersect again on the $A$-median at a point $X_A$.
\end{probEG}

\begin{proof}
	Si $A'$ es la reflexión de $A$ con respecto a $BC$, entonces $HA'$ es el diámetro de $\odot(BHC)$ de donde $X_A$ pertenece a $AA'$, que claramente es la $A$-mediana.
\end{proof}

\begin{probEG}[ELMO Shortlist 2013/G3]
	In $\triangle ABC$, a point $D$ lies on line $BC$. The circumcircle of $ABD$ meets $AC$ at $F$ (other than $A$), and the circumcircle of $ADC$ meets $AB$ at $E$ (other than $A$). Prove that as $D$ varies, the circumcircle of $AEF$ always passes through a fixed point other than $A$, and that this point lies on the median from $A$ to $BC$.
	\forum[aops]{3151962}
\end{probEG}

\begin{proof}
	Si $P=BF\cap EC$ y $A'$ la reflexión de $A$ con respecto a $BC$, entonces $P$ pertenece a $\odot(AEF)$ y $\odot(BHC)$. Como $AA'$ pasa por $X_A$, $\angle AX_AP=\angle A'CP=\angle AEP$ de donde $X_A\in\odot(AEF)$.
\end{proof}

\begin{probEG}[ELMO 2014/5]
	Let $ABC$ be a triangle with circumcenter $O$ and orthocenter $H$. Let $\omega_1$ and $\omega_2$ denote the circumcircles of triangles $BOC$ and $BHC$, respectively. Suppose the circle with diameter $\ol{AO}$ intersects $\omega_1$ again at $M$, and line $AM$ intersects $\omega_1$ again at $X$. Similarly, suppose the circle with diameter $\ol{AH}$ intersects $\omega_2$ again at $N$, and line $AN$ intersects $\omega_2$ again at $Y$. Prove that lines $MN$ and $XY$ are parallel.
	\forum[aops]{3534946}
\end{probEG}

\begin{proof}
	Pista: con angulitos tenemos que $M$ es el punto medio de la cuerda $A$-simediana, $N$ es el $A$-punto Humpty, $X$ es la intersección de las tangentes a $\odot(ABC)$ por $B$ y $C$, $Y$ es la reflexión de $A$ con respecto a $BC$.
\end{proof}

\begin{probEG}[USA TST 2005/6]
	Let $ABC$ be an acute scalene triangle with $O$ as its circumcenter. Point $P$ lies inside triangle $ABC$ with $\angle PAB=\angle PBC$ and $\angle PAC=\angle PCB$. Point $Q$ lies on line $BC$ with $QA=QP$. Prove that $\angle AQP=2\angle OQB$.
	\forum[aops]{734440}
\end{probEG}

\begin{probEG}[Brazil National Olympiad 2015/6]
	Let $\triangle ABC$ be a scalene triangle and $X$, $Y$ and $Z$ be points on the lines $BC$, $AC$ and $AB$, respectively, such that $\dang AXB=\dang BYC=\dang CZA$. The circumcircles of $BXZ$ and $CXY$ intersect at $P$. Prove that $P$ is on the circumference which diameter has ends in the orthocenter $H$ and in the barycenter $G$ of $\triangle ABC$.
	\forum[aops]{5469201}
\end{probEG}

\begin{proof}
	Se prueba que $P$ es el centro de roto-homotecia que manda $\triangle ABC\mapsto\triangle A_1B_1C_1$ donde $A_1=BY\cap CZ$ y análogamente para $B_1$ y $C_1$. Con esto se prueba que $\dang BAP=\dang CBP$ y $\dang CAP=\dang BCP$, de donde $P$ es el $A$-punto Humpty, entonces $\angle HPA=\angle HPG=90\dg$.
\end{proof}

\begin{probEG}[Sharygin Geometry Olympiad 2015 Final Round/10.3]
	Let $A_1$, $B_1$ and $C_1$ be the midpoints of sides $BC$, $CA$ and $AB$ of triangle $ABC$, respectively. Points $B_2$ and $C_2$ are the midpoints of segments $BA_1$ and $CA_1$ respectively. Point $B_3$ is symmetric to $C_1$ with respect to $B$, and $C_3$ is symmetric to $B_1$ with respect to $C$.
Prove that one of common points of circles $BB_2B_3$ and $CC_2C_3$ lies on the circumcircle of triangle $ABC$.
	\forum[aops]{10667375}
\end{probEG}

\begin{proof}
	Considere al centro de roto-homotecia que manda $AB\mapsto B_1B_2$ y otra que manda $AC\mapsto C_1C_2$, y sea $P$ ese punto. Luego,
	\[\angle BB_3B_2=\angle(AB,B_1B_2)=\angle BPB_2\]
	de donde $P\in\odot(BB_2B_3)$ y análogamente $P\in\odot(CC_2C_3)$. Además,
	\[\angle PBB_2+\angle PCC_2=\angle PAB_1+\angle PAC_1=\angle A\]
	de donde $P\in\odot(ABC)$.
\end{proof}

\note{Viernes\\2022-05-20}

\begin{problem}[IMO Shortlist 2014/G6]
	Let $ABC$ be a fixed acute-angled triangle. Consider some points $E$ and $F$ lying on the sides $AC$ and $AB$, respectively, and let $M$ be the midpoint of $EF$. Let the perpendicular bisector of $EF$ intersect the line $BC$ at $K$, and let the perpendicular bisector of $MK$ intersect the lines $AC$ and $AB$ at $S$ and $T$, respectively. We call the pair $(E,F)$ \emph{interesting}, if the quadrilateral $KSAT$ is cyclic. Suppose that the pair $(E_1,F_1)$ and $(E_2,F_2)$ are interesting. Prove that
	\[\frac{E_1E_2}{AB}=\frac{F_1F_2}{AC}.\]
\end{problem}

\begin{proof}
	Como $M$ pertenece a la mediana del $\triangle AST$ y su reflexión con respecto a $BC$ pertenece a $\odot(AST)$, entonces $AK$ es la simediana del $\triangle AST$. Como $K$ pertenece a la mediatriz del segmento $EF$, tenemos que $KE$ y $KF$ son tangentes a $\odot(AEF)$. Luego,
	\begin{align*}
		\frac{AE}{AB}
		&=\frac{AE}{EF}\cdot\frac{EF}{KE}\cdot\frac{KE}{CK}\cdot\frac{CK}{BC}\cdot\frac{BC}{AB}\\
		&=\frac{\sin{\alpha}}{\sin{\angle A}}\cdot\frac{\sin{2\angle A}}{\sin{\angle A}}\cdot\frac{\sin{\angle C}}{\sin{\alpha}}\cdot\frac{CK}{BC}\cdot\frac{\sin{\angle A}}{\sin{\angle C}}\\
		&=2\cos{\angle A}\cdot\frac{CK}{BC}
	\end{align*}
	donde $\alpha=\angle AFE=\angle CEK$, y análogamente $\frac{AF}{AC}=2\cos{\angle A}\cdot\frac{BK}{BC}$. Por ende,
	\[\frac{AE}{AB}+\frac{AF}{AC}=2\cos{\angle A}\]
	es una constante. Con esto es suficiente.
\end{proof}

\begin{probEG}[USA TSTST 2015/2]
	Let $ABC$ be a scalene triangle. Let $K_a$, $L_a$ and $M_a$ be the respective intersections with $BC$ of the internal angle bisector, external angle bisector, and the median from $A$. The circumcircle of $AK_aL_a$ intersects $AM_a$ a second time at point $X_a$ different from $A$. Define $X_b$ and $X_c$ analogously. Prove that the circumcenter of $X_aX_bX_c$ lies on the Euler line of $ABC$.
	\forum[aops]{5017915}
\end{probEG}

\begin{proof}
	Como $X_a$ es un punto de la $A$-mediana que pertenece a la $A$-circunferencia de Apolonio, entonces $X_a$ es el $A$-punto Humpty del $\triangle ABC$. Por ende, $\angle HX_aG=90\dg$ y el circuncentro del $\triangle X_aX_bX_c$ es el punto medio del segmento $HG$.
\end{proof}

\begin{probEG}[WOOT 2013 Practice Olympiad 3/5]
	A semicircle has center $O$ and diameter $AB$. Let $M$ be a point on $AB$ extended past $B$. A line through $M$ intersects the semicircle at $C$ and $D$, so that $D$ is closer to $M$ than $C$. The circumcircles of triangles $AOC$ and $DOB$ intersect at $O$ and $K$. Show that $\angle MKO=90\dg$.
\end{probEG}

\begin{probEG}[IMO 2010/4, modified]
	In $\triangle ABC$ with orthocenter $H$, suppose $P$ is the projection of $H$ onto the $C$-median; let the second intersections of $AP,BP,CP$ with $\odot(ABC)$ be $K,L,M$ respectively. Show that $MK=ML$.
\end{probEG}

\begin{probEG}[EGMO 2016/4]
	Two circles $\omega_1$ and $\omega_2$, of equal radius intersect at different points $X_1$ and $X_2$. Consider a circle $\omega$ externally tangent to $\omega_1$ at $T_1$ and internally tangent to $\omega_2$ at point $T_2$. Prove that lines $X_1T_1$ and $X_2T_2$ intersect at a point lying on $\omega$.
\end{probEG}

\begin{probEG}[USA TST 2008/7]
	Let $ABC$ be a triangle with $G$ as its centroid. Let $P$ be a variable point on segment $BC$. Points $Q$ and $R$ lie on sides $AC$ and $AB$ respectively, such that $PQ\parallel AB$ and $PR\parallel AC$. Prove that, as $P$ varies along segment $BC$, the circumcircle of triangle $AQR$ passes through a fixed point $X$ such that $\angle BAG=\angle CAX$.
	\forum[aops]{1247506}
\end{probEG}

\begin{probEG}[Mathematical Reflections/O371]
	Let $ABC$ be a triangle with $AB<AC$. Let $D,E$ be the feet of altitudes from $B,C$ to sides $AC,AB$ respectively. Let $M,N,P$ be the midpoints of the segments $BC,MD,ME$ respectively. Let $NP$ intersect $BC$ again at a point $S$ and let the line through $A$ parallel to $BC$ intersect $DE$ again at point $T$. Prove that $ST$ is tangent to the circumcircle of triangle $ADE$.
\end{probEG}

\begin{problem}[ELMO Shortlist 2012/G7]
	Let $\triangle ABC$ be an acute triangle with circumcenter $O$ such that $AB<AC$, let $Q$ be the intersection of the external bisector of $\angle A$ with $BC$, and let $P$ be a point in the interior of $\triangle ABC$ such that $\triangle BPA$ is similar to $\triangle APC$. Show that $\angle QPA+\angle OQB=90\dg$.
	\forum[aops]{2728473}
\end{problem}

\begin{problem}[Iranian Geometry Olympiad 2014/S4]
	The tangent to the circumcircle of an acute triangle $ABC$ (with $AB<AC$) at $A$ meets $BC$ at $P$. Let $X$ be a point on line $OP$ such that $\angle AXP=90\dg$. Points $E$ and $F$ lie on sides $AB$ and $AC$, respectively, and are on the same side of line $OP$ such that $\dang EXP=\dang ACX$ and $\dang FXO=\dang ABX $. Let $EF$ meet the circumcircle of triangle $ABC$ at points $K,L$. Prove that the line $OP$ is tangent to the circumcircle of triangle $KLX$.
	\forum[aops]{3758542}
\end{problem}

\note[Teoría de Números]{}

\begin{probEG}[IMO Shortlist 2001/N4]
	Let $p\ge 5$ be a prime number. Prove that there exists an integer $a$ with $1\le a\le p-2$ such that neither $a^{p-1}-1$ nor $(a+1)^{p-1}-1$ is divisible by $p^2$.
\end{probEG}

\begin{probMG}[IMO Shortlist 2001/N5]
	Let $a>b>c>d$ be positive integers and suppose that
	\[ac+bd=(b+d+a-c)(b+d-a+c).\]
	Prove that $ab+cd$ is not prime.
\end{probMG}

\begin{probHR}[IMO Shortlist 2001/N6]
	Is it possible to find $100$ positive integers not exceeding $25000$, such that all pairwise sums of them are different?
\end{probHR}

%\section{Semana 11 (05/23 -- 05/29)}

\note[Álgebra]{Lunes\\2022-05-23}

\begin{probEG}
	Sean $a,b,c\in\RR^+$. Demuestre que
	\[\frac{a}{\sqrt{a^2+8bc}}+\frac{b}{\sqrt{b^2+8ca}}+\frac{c}{\sqrt{c^2+8ab}}\ge 1.\]
\end{probEG}

\begin{proof}
	Como la desigualdad dada es homogenea, supongamos que $a+b+c=1$. Sabemos que la función $f(x)=x^{-1/2}$ es convexa, entonces por Jensen
	\begin{align*}
		\cycsum\frac{a}{\sqrt{a^2+8bc}}
		&=\cycsum a\cdot f(a^2+8bc)\\
		&\ge f\left(\cycsum a(a^2+8bc)\right)\\
		&=f(a^3+b^3+c^3+24abc)\\
		&\ge f(a^3+b^3+c^3+3(a+b)(b+c)(c+a))\\
		&=f(1)\\
		&=1.
	\end{align*}
\end{proof}

\begin{probEG}
	Sea $n\ge 3$ un número entero y sean $t_1,t_2,\dots,t_n$ números reales positivos tales que
	\[n^2+1>(t_1+t_2+\dots+t_n)\left(\frac{1}{t_1}+\frac{1}{t_2}+\dots+\frac{1}{t_n}\right).\]
	Demuestre que para todo $i,j,k$, con $1\le i<j<k\le n$, los números $t_i,t_j,t_k$ son las longitudes de los lados de un triángulo.
\end{probEG}

\begin{probEG}
	La secuencia $(a_n)_{n\ge 0}$ está definida por
	\[a_n=\abs{a_{n+1}-a_{n+2}},\quad\forall\,n\ge 0,\]
	donde $a_0,a_1\in\RR^+$, con $a_0\ne a_1$. ¿Puede que la secuencia $(a_n)$ sea acotada?
\end{probEG}

\begin{proof}
	La respuesta es no. Considere tres casos: $a_0>a_1$, $a_0=a_1$ y $a_0<a_1$.
\end{proof}

\begin{probMR}
	Determine todos los polinomios $P(x)$ con coeficientes reales tales que
	\[P(a-b)+P(b-c)+P(c-a)=2P(a+b+c)\]
	para todo $a,b,c\in\RR$, con $ab+bc+ca=0$.
\end{probMR}

\note[Punto\\Dumpty \cite{ref:dumpty}]{Martes\\2022-05-24}

\begin{probEG}[USAMO 2008/2]
	Let $ABC$ be an acute, scalene triangle, and let $M$, $N$, and $P$ be the midpoints of $\ol{BC}$, $\ol{CA}$, and $\ol{AB}$, respectively. Let the perpendicular bisectors of $\ol{AB}$ and $\ol{AC}$ intersect ray $AM$ in points $D$ and $E$ respectively, and let lines $BD$ and $CE$ intersect in point $F$, inside of triangle $ABC$. Prove that points $A$, $N$, $F$, and $P$ all lie on one circle.
	\forum[aops]{1116181}
\end{probEG}

\note{Miércoles\\2022-05-25}

\begin{probMG}[Indian TST Practice Test 2019/2]
	Let $ABC$ be a triangle with $\angle A=\angle C=30\dg$. Points $D,E,F$ are chosen on the sides $AB,BC,CA$ respectively so that $\angle BFD=\angle BFE=60\dg$. Let $p$ and $p_1$ be the perimeters of the triangles $ABC$ and $DEF$, respectively. Prove that $p\le 2p_1$.
	\forum[aops]{12753024}
\end{probMG}

\note[IMO TST Día 1]{Jueves\\2022-05-26}

\begin{probEG}[Peru IMO TST 2022/1]
	Encuentre todos los enteros positivos $n$ tales que para cualquier entero positivo $k$ con $1\le k\le\sqrt{n}$ se cumple que el número $\floor{n}{k}-k$ es impar.
\end{probEG}

\begin{proof}
	Respuesta: $n=2,6,14,30$.
\end{proof}

\begin{probEG}[Peru IMO TST 2022/2]
	Determine todas las funciones $f:\RR\to\RR$ tales que
	\[f(xy+f(x))+f(y)=xf(y)+f(x+y)\]
	para todo par de números reales $x,y$.
\end{probEG}

\begin{proof}
	Respuesta: $f(x)\equiv 2-x,x,0$.
\end{proof}

\begin{probEG}[Peru IMO TST 2022/3]
	Considere un $n$-ágono (un polígono de $n$ lados) con $n\ge 3$. Se asignan números enteros no negativos distintos dos a dos a cada lado y diagonal del $n$-ágono de modo que los números asignados a los lados de cualquier triángulo con vértices en los vértices del $n$-ágono forman una progresión aritmética. Determine el máximo valor de $n$ para el cual esto es posible.
\end{probEG}

\begin{proof}
	Respuesta: $n=4$.
\end{proof}

\begin{probMB}[Peru IMO TST 2022/4]
	Sea $\Omega$ el circuncírculo del triángulo $ABC$, con $\angle BAC>90\dg$ y $AB>AC$. Las rectas tangentes a $\Omega$ en los puntos $B$ y $C$ se intersectan en $D$. La recta tangente a $\Omega$ en el punto $A$ interseca a la recta $BC$ en $E$. La recta paralela a $AE$ que pasa por $D$ interseca a la recta $BC$ en $F$. La circunferencia $\Gamma$ con diámetro $EF$ interseca a la recta $AB$ en los puntos $P$ y $Q$, y a la recta $AC$ en los puntos $X$ e $Y$. Pruebe que uno de los ángulos $\angle AEB$, $\angle PEQ$, $\angle XEY$ es igual a la suma de los otros dos.
\end{probMB}

\begin{proof}
	Sea $Y'=A'B\cap AC$ donde $A'$ es la reflexión de $A$ con respecto a $BC$. Sea $Q'$ la reflexión de $Y'$ con respecto a $BC$. Luego,
	\[\angle EBY'=\angle ABC=\angle EAC=\angle EAY'\]
	de donde $EABY'$ es cíclico. Si $R=AB\cap DF$, entonces $\angle RFB=\angle AEB=\angle CY'B$ de donde
	\[\triangle RFB\cong\triangle AEB\cong\triangle CY'B.\]
	Luego, $B$ es el centro de roto-homotecia que manda $(A,B,C,R)\mapsto(E,B,Y',F)$. Es decir, $\angle EY'F=\angle ACR$. Si $S$ es la intersección de la mediatriz del segmento $BC$ y la recta $AB$, entonces
	\[\angle DRS=180\dg-\angle BRF=180\dg-\angle BCY'=\angle ACB=\angle SBD=\angle DCS\]
	de donde $DRCS$ es cíclico y
	\begin{align*}
		\angle EY'F
		&=\angle ACR\\
		&=\angle CAB-\angle CRB\\
		&=(180\dg-\angle BCD)-\angle CRS\\
		&=180\dg-(90\dg-\angle CDS)-\angle CDS\\
		&=90\dg.
	\end{align*}
	Por lo tanto, $Y'$ y $Q'$ pertenecen a la circunferencia de diámetro $EF$. Sin pérdida de generalidad, $Y'=Y$ y $Q'=Q$. Sabemos que el ángulo que forman las rectas $PX$ y $QY$ es la suma o la diferencia de las medidas de los arcos $PQ$ y $XY$, es decir, la suma o la diferencia de $\angle PEQ$ y $\angle XEY$. Note que $\angle APX=\angle QYA=90\dg-\angle ACB$ y $\angle AQY=90\dg-\angle CBA$, de donde
	\[\angle AQY-\angle APX=\angle ACB-\angle CBA=\angle ACB-\angle CAE=\angle AEB\]
	es la suma o la diferencia de $\angle PEQ$ y $\angle XEY$.
\end{proof}

\note{Sábado\\2022-05-28}

\begin{probMG}
	Demuestre que para cualesquiera enteros positivos $m$ y $n$ existen infinitas parejas de enteros positivos $(a,b)$, coprimos, tales que
	\[a+b\mid a\cdot m^a+b\cdot n^b.\]
\end{probMG}

\begin{proof}
	Pista: considerar $a+b=p$ donde $p$ es un número primo apropiado.
\end{proof}

\note[Simulacro Nivel 2]{}

\begin{probEG}
	Sean $AK$ y $BL$ las alturas de un triángulo $ABC$ trazadas desde los vértices $A$ y $B$, respectivamente. El punto $P$ tomado del segmento $AK$ es tal que $LK=LP$. La paralela a $BC$ que pasa por $P$ intersecta en el punto $Q$ a la paralela a $PL$ que pasa por $B$. Demuestre que $\angle AQB=\angle ACB$.
\end{probEG}

\begin{proof}
	Sea $M$ el punto medio de $QB$. Como $QP\parallel BC\perp AK$, entonces $ML$ es la mediatriz de $PK$. Como $MB\parallel PL$, tenemos
	\[\angle BML=\angle PLM=\angle MLK=\angle A=\angle BAL\]
	de donde $BMAL$ es cíclico y $\angle AMB=\angle ALB=90\dg$. Por ende,
	\[\angle AQB=\angle ABM=\angle ALM=\angle ACB.\]
\end{proof}

\begin{probEB}
	Sea $n>1$ un número entero. Se marcan $k$ casillas en un tablero de $n\times n$. Determine el mayor valor posible de $k$ para el cual es posible permutar filas y columnas de tal manera que las $k$ casillas marcadas estén en la diagonal principal o por encima de ella.
\end{probEB}

\begin{proof}
	Respuesta: $n+1$ por inducción. El siguiente es un contraejemplo de $n+2$.
	\[\begin{tblr}{row{3}={28pt},column{3}={28pt}}
		\clubsuit&         &      &         &\clubsuit\\
		         &\clubsuit&      &         &         \\
		         &         &\ddots&         &         \\
		         &         &      &\clubsuit&         \\
		\clubsuit&         &      &         &\clubsuit
	\end{tblr}\]
\end{proof}

\begin{probMR}
	Sean $a_1$, $a_2$, $a_3$, $b_1$, $b_2$ y $b_3$ enteros positivos distintos dos a dos tales que
	\[(n+1)a_1^n+na_2^n+(n-1)a_3^n\mid(n+1)b_1^n+nb_2^n+(n-1)b_3^n\]
	para todo entero positivo $n$. Demuestre que existe un número entero positivo $k$ tal que $b_i=k\cdot a_i$ para $i=1,2,3$.
\end{probMR}

\note[IMO TST Día 2]{Domingo\\2022-05-29}

\begin{probEG}[Peru IMO TST 2022/5]
	Sea $N$ un entero positivo. Determine todos los enteros positivos $n$ que satisfacen la siguiente condición: para cualquier lista $d_1,d_2,\dots,d_k$ de divisotres de $n$ (no necesariamente distintos) tal que
	\[\frac{1}{d_1}+\frac{1}{d_2}+\dots+\frac{1}{d_k}>N,\]
	algunas de las fracciones $\frac{1}{d_1},\frac{1}{d_2},\dots,\frac{1}{d_k}$ suman exactamente $N$.
\end{probEG}

\begin{proof}
	Respuesta: $n=p^e$ para todo número primo $p$ y $e\in\ZZ^+_0$.
\end{proof}

\begin{probHB}[Peru IMO TST 2022/6]
	Sea $n\ge 2$ un entero y sean $a_1,a_2,\dots,a_n$ números reales positivos cuya suma es $1$. Pruebe que
	\[\sum_{k=1}^n\frac{a_k}{1-a_k}(a_1+a_2+\dots+a_{k-1})^2<\frac13.\]
\end{probHB}

\begin{proof}
	Sea $S_k=a_1+a_2+\dots+a_k$ para todo índice $0\le k\le n$. Luego,
	\[0=S_0<S_1<\cdots<S_{n-1}<S_n=1.\]
	Note que
	\[\frac{a_k}{1-a_k}(a_1+a_2+\dots+a_{k-1})^2=\frac{S_{k-1}^2(S_k-S_{k-1})}{1-S_k+S_{k-1}}.\]
	Ahora, probaremos que si $1\ge a>b\ge 0$ entonces
	\[\frac{b^2(a-b)}{1-a+b}\le\frac{a^3-b^3}{3}\]
	donde la igualdad se cumple si y solo si $(a,b)=(1,0)$. Note que
	\begin{align*}
		3b^2-3b^2(1-a+b)
		&=(a-b)\cdot 3b\cdot b\\
		&\le(a-b)(a+2b)(1-a+b)\\
		&=(a^2+ab+b^2)(1-a+b)-3b^2(1-a+b)
	\end{align*}
	de donde
	\[3b^2\le(a^2+ab+b^2)(1-a+b)\]
	y de esto obtenemos lo deseado. Como $n\ge 2$, es claro que $(S_k,S_{k-1})\ne(1,0)$, así que
	\[\sum_{k=1}^n\frac{a_k}{1-a_k}(a_1+a_2+\dots+a_{k-1})^2<\sum_{k=1}^n\frac{S_k^3-S_{k-1}^3}{3}=\frac{S_n^3-S_0^3}{3}=\frac13.\]
\end{proof}

\begin{probMB}[Peru IMO TST 2022/7]
	Sea $m>1$ un entero. Considere un tablero de tamaño $3m\times 3m$. Una rana se ubica en la casilla de la esquina inferior izquierda $S$ del tablero y quiere llegar mediante saltos a la casilla de la esquina superior derecha $F$ del tablero. La rana solo puede saltar a la casilla vecina derecha o a la casilla vecina de arriba de la casilla en la que se encuentra.

	Algunas casillas están pegajosas y la rana queda atrapada si llega a cualquiera de ellas. El conjunto $X$ de casillas pegajosas se denomina \emph{barricada} si la rana no puede llegar a $F$ partiendo de $S$. Una barricada es \emph{minimal} si no contiene una barricada más pequeña.
	\begin{enumerate}[a)]
		\ii Pruebe que existe una barricada minimal conteniendo al menos $3m^2-3m$ casillas.
		\ii Pruebe que cualquier barricada minimal contiene a lo más $3m^2$ casillas.
	\end{enumerate}
	\emph{Nota:} En la figura se muestra un ejemplo de una barricada minimal $X$ para $m=2$. Las casillas en $X$ han sido marcadas con $\clubsuit$.
	\[\begin{tblr}{}
		         &         &         &         &         &F\\
		\clubsuit&\clubsuit&         &         &         & \\
		         &         &\clubsuit&         &         & \\
		         &         &         &\clubsuit&         & \\
		         &         &         &         &\clubsuit& \\
		S        &         &\clubsuit&         &         &
	\end{tblr}\]
\end{probMB}

\begin{proof}
	Digamos que $X$ es el conjunto de todas las casillas marcadas en el siguiente tablero de $3m\times 3m$.
	\[\begin{tblr}{row{5}={28pt},column{7}={28pt}}
		         &         &         &         &         &         &      &         &         &F        \\
		\clubsuit&         &         &\clubsuit&         &         &      &\clubsuit&         &         \\
		         &\clubsuit&         &         &\clubsuit&         &\cdots&         &\clubsuit&         \\
		         &\clubsuit&         &         &\clubsuit&         &      &         &\clubsuit&         \\
		         &\vdots   &         &         &\vdots   &         &\ddots&         &\vdots   &         \\
		         &\clubsuit&         &         &\clubsuit&         &      &         &\clubsuit&         \\
		         &\clubsuit&         &         &\clubsuit&         &\cdots&         &\clubsuit&         \\
		         &         &\clubsuit&         &         &\clubsuit&      &         &         &\clubsuit\\
		S        &         &         &         &         &         &      &         &         &
	\end{tblr}\]
	Es fácil ver que $X$ es una barricada minimal con $\abs{X}=3m^2-2m>3m^2-3m$. Ahora, para probar que una barricada minimal $X$ contiene a lo más $3m^2$ casillas, asignaremos dos casillas vecinas a cada casilla pegajosa, y de esta manera $3\abs{X}\le 3m\times 3m$ de donde $\abs{X}\le 3m^2$.
\end{proof}

%\section{Semana 12 (05/30 -- 06/05)}

\note[Ecuaciones Funcionales]{Lunes\\2022-05-30}

\begin{probMG}
	Determine todas las funciones $f:\RR\to\RR$ tales que
	\[f(x+yf(x))+f(xf(y)-y)=f(x)-f(y)+2xy\]
	para todo $x,y\in\RR$.
\end{probMG}

\begin{proof}
	Con $(x,y)=(0,0)$ obtenemos $f(0)=0$. Con $(x,y)=(0,x)$ obtenemos $f(-x)=-f(x)$. Con $(x,y)=(-x,y)$ obtenemos
	\[f(x+yf(x))+f(xf(y)+y)=f(x)+f(y)+2xy\]
	de donde
	\[f(y+xf(y))+f(y-xf(y))=2f(y)\]
	y así $f$ es aditiva. Luego, de la ecuación original
	\[f(yf(x))+f(xf(y))=2xy\]
	y
	\[f(xf(x))=x^2.\]
	Si $f(a)=f(b)$, entonces $f(x+a)=f(x+b)$ de donde
	\[2\cdot(a-b)\cdot 2(a-b)=f(3(a-b)f(a-b))=2\cdot 0\cdot 3(a-b)\]
	y $a=b$. Es decir, $f$ es inyectiva. Con $(x,y)=(xf(x),1/x)$ obtenemos
	\[f(xf(x)f(1/x))=f(x)\]
	de donde
	\[f(1/x)=1/f(x)\]
	y con $(x,y)=(x,1/x)$ obtenemos
	\[f(a)+1/f(a)=2\]
	donde $a=f(x)/x$. Luego, $f(a)=1$ de donde $f(x)=xc$ siendo $c\in\RR$ una constante. Finalmente, $f(x)=x$ y $f(x)=-x$ son las únicas que cumplen.
\end{proof}

\begin{probEG}
	Determine todas las funciones $f:\RR\to\RR$ tales que
	\[f(x)^2+2yf(x)+f(y)=f(y+f(x))\]
	para todo $x,y\in\RR$.
\end{probEG}

\begin{proof}
	Es claro que $f(x)=0$ es una solución. Ahora, si existe un $a\in\RR$ tal que $f(a)\ne 0$, con $(x,y)=\left(a,\frac{x-f(a)^2}{2f(a)}\right)$ obtenemos
	\[x=f(a)^2+(x-f(a)^2)=f(t)-f(s)\]
	para algunos $t,s\in\RR$. Con $(x,y)=(s,-f(s))$ obtenemos
	\[f(s)^2-2f(s)^2+f(-f(s))=f(0)\]
	de donde $f(-f(s))=f(s)^2+f(0)$. Luego, con $(x,y)=(t,-f(s))$ obtenemos
	\[(f(t)-f(s))^2+f(0)=f(t)^2-2f(t)f(s)+(f(s)^2+f(0))=f(f(t)-f(s))\]
	de donde $f(x)=x^2+c$ para algún $c\in\RR$ fijo.
\end{proof}

\begin{probMB}
	Determine todas las funciones $f:\RR\to\RR$ tales que
	\[f(f(x)+y)=2x+f(f(y)-x)\]
	para todo $x,y\in\RR$.
\end{probMB}

\begin{proof}
	Si $x=\half(f(0)-t)$ y $y=-f(x)$ para algún $t\in\RR$, entonces
	\[f(0)=f(0)-t+f(f(y)-x)\]
	de donde $f$ es sobreyectiva. Luego, existe un $a\in\RR$ tal que $f(a)=0$. Si $x=a$, tenemos que
	\[f(y)=2a+f(f(y)-a)\]
	de donde $f(x)=x+c$ siendo $c\in\RR$ una constante.
\end{proof}

\begin{probEG}
	Determine todas las funciones $f:\RR\to\RR$ tales que
	\[f(x^2+f(y))=y+f(x)^2\]
	para todo $x,y\in\RR$.
\end{probEG}

\begin{proof}
	Si $x=0$ tenemos que $f(f(x))=x+f(0)^2$. Es decir, $f$ es biyectiva. Luego, existe un $a\in\RR$ tal que $f(a)=0$. Con $(x,y)=(a,f(x))$ obtenemos
	\[f(a^2+f(f(x)))=f(x)+f(a)^2=f(x)\]
	de donde
	\[x+f(0)^2=f(f(x))=x-a^2\]
	y así $f(0)=0$, es decir, $f(f(x))=x$. Luego, si $y=0$ tenemos que $f(x^2)=f(x)^2$ de donde
	\[f(-x)^2=f(x^2)=f(x)^2\]
	y $f(-x)=-f(x)$. Con $y=f(y^2)$ obtenemos
	\[f(x^2+y^2)=f(x^2)+f(y^2)\]
	y $f(x^2)\ge 0$ para todo $x\in\RR$, de donde $f$ satisface las condiciones de la ecuación funcional de Cauchy en el intérvalo $[0,+\infty)$. Es decir, $f(x)=cx$ para algún $c\in\RR$ fijo. Es fácil ver que $c=1$, por lo que $f(x)=x$ para todo $x\in\RR$.
\end{proof}

\note[Combinatoria]{Martes\\2022-05-31}

\begin{probEG}
	Sea $n>2$ un número entero. Hay $n$ lámparas alrededor de una circunferencia, las cuales están etiquetadas con los números del $1$ al $n$ en sentido horario. Cada lámpara puede estar encendida o apagada. Un \emph{movimiento} consiste en cambiar simultáneamente el estado de dos lámparas adyacentes. Si al inicio todas las lámparas están apagadas, ¿cuántas configuraciones distintas de estados de lámparas es posible lograr usando algunos movimientos?
\end{probEG}

\begin{probEG}
	¿De cuántas formas podemos escribir los números del $1$ al $3n$ en las casillas de un tablero de $3$ filas y $n$ columnas, sin repetir, de tal manera que cualesquiera dos números consecutivos estén en casillas vecinas (que comparten un lado) y que los números $1$ y $3n$ estén en casillas vecinas?
\end{probEG}

\begin{probEG}
	¿De cuántas formas podemos escribir los números del $1$ al $3n$ en las casillas de un tablero de $3$ filas y $n$ columnas, sin repetir, de tal manera que el número $1$ esté en la primera columna, el número $3n$ esté en la última columna, y cualesquiera dos números consecutivos estén en casillas vecinas (que comparten un lado)?
\end{probEG}

\begin{probEG}
	Un tablero finito ha sido cubierto por fichas de $1\times 2$ cumpliendo las siguientes reglas:
	\begin{itemize}
		\ii los bordes de las fichas están sobre las líneas del tablero;
		\ii ninguna ficha puede salir del tablero;
		\ii cada casilla está cubierta por exactamente dos fichas.
	\end{itemize}
	Demuestre que es posible quitar algunas fichas del tablero de tal manera que cada casilla esté cubierta por exactamente una ficha.
\end{probEG}

\begin{probEG}
	Sea $n$ un entero positivo. Hay $2^n$ soldados en una fila. Los soldados se pueden reacomodar en una nueva fila respetando la siguiente regla: los soldados que se encuentran en posiciones impares se mueven al frente de la fila, manteniendo sus posiciones entre sí; mientras que los soldados que están en posiciones pares se mueven al frente de la fila, al final de la nueva fila, manteniendo sus posiciones entre sí. Demuestre que después de $n$ reordenamientos los soldados estarán en el mismo orden que al principio.
\end{probEG}

\begin{probMG}[Estonia TST 2021/4]
	\begin{enumerate}[(a)]
		\ii There are $2n$ rays marked in a plane, with $n$ being a natural number. Given that no two marked rays have the same direction and no two marked rays have a common initial point, prove that there exists a line that passes through none of the initial points of the marked rays and intersects with exactly $n$ marked rays.
		\ii Would the claim still hold if the assumption that no two marked rays have a common initial point was dropped?
	\end{enumerate}
	\forum[aops]{23997712}
\end{probMG}

\begin{proof}
	Respuesta: sí.
\end{proof}

\begin{probEG}[Estonia National Olympiad 2020/10.1\protect\footnote{Un video relacionado de \href{https://www.youtube.com/c/MatesMike}{Mates Mike}: \url{https://www.youtube.com/watch?v=5PAGXnPTE94}.}]
	A room of the shape of a rectangular parallelepiped has vertical walls covered by mirrors. A laser beam of diameter $0$ enters the room from one corner and moves horizontally along the bisector of that corner. After reflecting from some wall, the beam continues moving horizontally according to the laws of reflection (i.e. the bisector of the angle between the imaginary continuation of the trajectory of the beam before reflection and the real continuation trajectory is along the wall). When the beam reaches a corner, it will return along the way it arrived.
	\begin{enumerate}[(a)]
		\ii Prove that if the ratio of the side lengths of the floor of the room is rational then the beam eventually returns to the point of entrance.
		\ii Prove that if the ratio of the side lengths of the floor of the room is irrational then the beam never returns to the point of entrance.
	\end{enumerate}
\end{probEG}

\begin{probEG}[Estonia National Olympiad 2020/12.4]
	On a horizontal line, one colors $2k$ points red and, in the right of them, $2l$ points blue. On every move, one chooses two points of different color, such that there is exactly one colored point between them, and interchanges the colors of the chosen points. How many different configurations can one obtain using these moves?
\end{probEG}

\begin{proof}
	Respuesta: $\binom{k+l}{k}^2$.
\end{proof}

\begin{problem}[Estonia National Olympiad 2020/12.5]
	Let $n$ be a positive integer, $n\ge 3$. In a regular $n$-gon, one draws a maximal set of diagonals, no two of which intersect in the interior of the $n$-gon. Every diagonal is labelled with the number of sides of the $n$-gon between the endpoints of the diagonal along the shortest path. Find the maximum value of the sum of the labels.
\end{problem}

\note[Teoría de Números]{Miércoles\\2022-06-01}

\begin{probMG}[CSMO 2020/10.4]
	Sea $a_1,a_2,\dots,a_{17}$ una permutación de los números $1,2,\dots,17$ tal que
	\[(a_1-a_2)(a_2-a_3)\cdots(a_{17}-a_1)=n^{17}\]
	para algún número entero $n$. Determine el máximo valor de $n$.
	\forum[aops]{16999538}
	\begin{hint}
		Let $a_1,a_2,\dots,a_{17}$ be a permutation of $1,2,\dots,17$ such that $(a_1-a_2)(a_2-a_3)\cdots(a_{17}-a_1)=n^{17}$. Find the maximum possible value of $n$.
	\end{hint}
\end{probMG}

\begin{proof}
	La respuesta es $6$. Por MA-MG, es fácil ver que $n\le 8$. Si $n=8$, hay al menos $15$ índices $i$ tal que $\abs{a_i-a_{i+1}}=8$, absurdo. Si $n=7$, $a_i-a_{i+1}$ es impar para todo $i$, absurdo. Así que $n\le 6$ y un ejemplo es
	\[17,8,16,7,15,6,14,5,13,4,2,11,10,1,3,12,9.\]
\end{proof}

\begin{probEG}[CSMO 2020/11.1]
	Let $a_1,a_2,\dots,a_{17}$ be a permutation of $1,2,\dots,17$ such that $(a_1-a_2)(a_2-a_3)\cdots(a_{17}-a_1)=2^n$. Find the maximum possible value of positive integer $n$.
	\forum[aops]{16999514}
\end{probEG}

\begin{proof}
	Respuesta: $38$. Como ejemplo considere a lo siguiente:
	\[1,17,9,13,5,3,11,15,7,8,16,12,4,6,14,10,2.\]
\end{proof}

\begin{probMR}[CSMO 2020/11.7]
	Sea $a_1,a_2,a_3,\dots$ la secuencia de todos los enteros positivos libres de cuadrados, en forma creciente. Demuestre que $a_{n+1}-a_n=2020$ para infinitos enteros positivos $n$.
	\forum[aops]{16952166}
	\begin{hint}
		Arrange all square-free positive integers in ascending order $a_1,a_2,a_3,\dots,a_n,\dots$. Prove that there are infinitely many positive integers $n$, such that $a_{n+1}-a_n=2020$.
	\end{hint}
\end{probMR}

\begin{proof}
	Por el teorema chino del resto, existe un $x\in\ZZ^+$ tal que $p_i\mid x+i$ para todo $1\le i\le 2019$, donde $p_1,p_2,\dots,p_{2019}$ son algunos primos distintos. Luego, sea $M=p_1p_2\dots p_{2019}$ y $p$ un primo. Si fijamos un $N\in\ZZ^+$ suficientemente grande, hay a lo sumo $\ceiling{\frac{N}{p^2}}$ números $1\le k\le N$ tales que $p^2\mid kM^2+x$. Análogamente, hay a lo sumo $\ceiling{\frac{N}{p^2}}$ números $1\le k\le N$ tales que $p^2\mid kM^2+x+2020$. Note que $p^2\le kM^2+x+2020\le M^2(N+2)$ de donde $p\le M\sqrt{N+2}$. Ahora, consideremos un resultado bien conocido:\footnote{Es conocido como $P(2)$, donde $P(s)$ es la función zeta prima (ver \url{https://en.wikipedia.org/wiki/Prime_zeta_function}).}
	\[\sum_{p\text{ es primo}}\frac{1}{p^2}\approx 0.45224742<0.48\]
	de donde
	\begin{align*}
		N-2\sum_{\substack{p\text{ es primo}\\p\le M\sqrt{N+2}}}\ceiling{\frac{N}{p^2}}
		&\ge N-2N\sum_{\substack{p\text{ es primo}}}\frac{1}{p^2}-2M\sqrt{N+2}\\
		&>0.04N-2M\sqrt{N+2}
	\end{align*}
	tiende al infinito cuando $N\to+\infty$, así que con esto terminamos.
\end{proof}

%\section{Semana 13 (06/06 -- 06/12)}

\note[Álgebra]{Martes\\2022-06-07}

\begin{probMB}[ISL 2009/A5]
  Sea $f:\RR\to\RR$ una función. Demuestre que existen $x,y\in\RR$ tales que
  \[f(x-f(y))>yf(x)+x.\]
  \forum[aops]{1932913}
  \begin{hint}
    Let $f$ be any function that maps the set of real numbers into the set of
    real numbers. Prove that there exist real numbers $x$ and $y$ such that
    \[f(x-f(y))>yf(x)+x.\]
  \end{hint}
\end{probMB}

\begin{proof}
  Supongamos por el absurdo que $f(x-f(y))\le yf(x)+x$ para todo $x,y\in\RR$. Si
  $y=0$, tenemos $f(x)\le x+f(0)$ de donde $f(x)\le 0$ para todo $x\le -f(0)$.
  Si $(x,y)=(f(x),x)$ y $x\ge 0$, entonces
  \begin{equation}\label{eq:func_ineq}
    f(0)\le xf(f(x))+f(x)
  \end{equation}
  Ahora, si $f(x)\le -f(0)$ tenemos $f(f(x))\le 0$, de donde $f(0)\le f(x)$. Es
  decir, $f(x)\ge-\abs{f(0)}$ para todo $x\ge 0$. Ahora, digamos que $x\in\RR$
  es un número cualquiera. Luego, existe un $y\to-\infty$ tal que $f(y)<x$. Pero
  si $f(x)>0$, entonces
  \[-\abs{f(0)}\le f(x-f(y))\le yf(x)+x\]
  lo cual es un absurdo. Por ende, $f(x)\le 0$ para todo $x\in\RR$. De la
  ecuación \eqref{eq:func_ineq} tenemos que $f(x)\ge f(0)$ para todo $x\ge 0$.
  Ahora, si existe algún $x\ge 0$ tal que $f(x)<0$, sea $y\in\RR^+$
  suficientemente grande. Luego,
  \[f(0)\le f(x-f(y))\le yf(x)+x\]
  lo cual es un absurdo. Por ende, $f(x)=0$ para todo $x\ge 0$. Ahora, si existe
  algún $x<0$ tal que $f(x)<0$, sea $y\in\RR^+$ suficientemente grande. Luego,
  \[-f(x)(y-1)\le x\]
  lo cual es un absurdo. Por lo tanto, $f\equiv 0$ pero claramente esto es un
  absurdo.
\end{proof}

\begin{probMB}[ISL 2009/A6]
  Suponga que la secuencia $s_1,s_2,s_3,\dots$ es una secuencia estrictamente
  creciente de enteros positivos tal que las subsecuencias
  \[
    s_{s_1},s_{s_2},s_{s_3},\dots\quad\text{y}\quad
    s_{s_1+1},s_{s_2+1},s_{s_3+1},\dots
  \]
  son ambas progresiones aritméticas. Demuestre que $s_1,s_2,s_3,\dots$ es
  también una progresión aritmética.
  \forum[aops]{1561573}
  \begin{hint}
    Suppose that $s_1,s_2,s_3,\dots$ is a strictly increasing sequence of
    positive integers such that the sub-sequences
    \[
      s_{s_1},s_{s_2},s_{s_3},\dots\quad\text{and}\quad
      s_{s_1+1},s_{s_2+1},s_{s_3+1},\dots
    \]
    are both arithmetic progressions. Prove that the sequence
    $s_1,s_2,s_3,\dots$ is itself an arithmetic progression.
  \end{hint}
\end{probMB}

\begin{proof}
  Sea $s_{s_i}=a_1+(i-1)d_1$ y $s_{s_i+1}=a_2+(i-1)d_2$ para todo $i\in\ZZ^+$.
  Luego,
  \[s_{s_i}=a_1+(i-1)d_1<s_{s_i+1}=a_2+(i-1)d_2\le s_{s_{i+1}}=a_1+id_1\]
  de donde $d_1\le d_2$, $a_1<a_2$ y $d_2\le d_1$ de donde $d_1=d_2=d$.
  Entonces,
  \[0<s_{i+1}-s_i\le s_{s_{i+1}}-s_{s_i}=d\]
  y en efecto sean $N$ y $M$ el valor mínimo y máximo de $s_{i+1}-s_i$. Si
  $M=s_{k+1}-s_k$, entonces
  \[N\cdot M=N\cdot(s_{k+1}-s_k)\le s_{s_{k+1}}-s_{s_k}=d\]
  y análogamente $N\cdot M\ge d$ de donde $d=N\cdot M$ y como se cumple la
  igualdad, $N=s_{s_k+1}-s_{s_k}=a_2-a_1$ y análogamente $M=a_2-a_1$ de donde
  $s_{i+1}-s_i$ es constante.
\end{proof}

\begin{probMG}[ISL 2009/A7]
  Determine todas las funciones $f:\RR\to\RR$ tales que
  \[f(xf(x+y))=f(yf(x))+x^2\]
  para todo $x,y\in\RR$.
  \forum[aops]{1932915}
  \begin{hint}
    Find all functions $f$ from the set of real numbers into the set of real
    numbers which satisfy for all $x,y$ the identity
    \[f(xf(x+y))=f(yf(x))+x^2.\]
  \end{hint}
\end{probMG}

\begin{proof}
  Si $x=0$, tenemos $f(0)=f(yf(0))$ de donde si $f(0)\ne 0$ entonces $f$ es
  constante, lo cual es un absurdo. Por ende, $f(0)=0$. Si $y=0$, tenemos
  \[f(xf(x))=f(0)+x^2=x^2.\]
  Si $f(x)=0$, $0=f(0)=x^2$ de donde $x=0$. Si $y=-x$, tenemos $f(-xf(x))=-x^2$
  de donde $f$ es suryectiva. Ahora, si $f(a)=f(b)$ entonces sea
  $(x,y)=(a,b-a)$, luego
  \[a^2=f(af(b))=f((b-a)f(a))+a^2\]
  de donde $f((b-a)f(a))=0$ y $a=b$. Es decir, $f$ es inyectiva. Luego,
  \[f(xf(x))=x^2=f(-xf(-x))\]
  de donde $f(-x)=-f(x)$. Si $(x,y)=(x,2x)$ y $(x,y)=(2x,-x)$, entonces
  \[f(2xf(x))=f(-xf(2x))+4x^2=4x^2-(2x^2)=f(xf(2x))\]
  de donde $f(2x)=2f(x)$. Sea $k\in\RR$ tal que $f(k)=1$. Luego,
  \[1=f(k)=f(kf(k))=k^2\]
  de donde $k=\pm 1$, es decir, $f(1)=\pm 1$. Si $(x,y)=(x,1)$ y
  $(x,y)=(x+1,-x)$ entonces $f(xf(x+1))=f(f(x))+x^2$ y
  $f(1)f(x+1)=-f(xf(x+1))+(x+1)^2$ de donde $f(f(x))=2x+1-f(1)f(x+1)$. Si
  $(x,y)=(1,x-1)$ entonces $f(f(x))=f(1)f(x-1)+1$ de donde
  $2x=f(1)(f(x+1)+f(x-1))$. Si $x=2x+1$, entonces
  \[2(2x+1)=f(1)(f(2x+2)+f(2x))=2f(1)(f(x+1)+f(x))\]
  de donde $f(1)f(x+1)=2x+1-f(1)f(x)$. Es decir,
  \[f(f(x))=2x+1-f(1)f(x+1)=f(1)f(x)\]
  de donde $f(x)=\pm x$ para todo $x\in\RR$.
\end{proof}

\begin{probEB}[ISL 2010/A2]
  Sean $a,b,c,d\in\RR$ tales que $a+b+c+d=6$ y $a^2+b^2+c^2+d^2=12$. Demuestre
  que
  \[36\le 4(a^3+b^3+c^3+d^3)-(a^4+b^4+c^4+d^4)\le 48.\]
  \forum[aops]{2362276}
  \begin{hint}
    Let the real numbers $a,b,c,d$ satisfy the relations $a+b+c+d=6$ and
    $a^2+b^2+c^2+d^2=12$. Prove that
    \[36\le 4(a^3+b^3+c^3+d^3)-(a^4+b^4+c^4+d^4)\le 48.\]
  \end{hint}
\end{probEB}

\begin{proof}
  Note que
  \[\cycsum(a-1)^2=\cycsum a^2-2\cycsum a+4=4\]
  de donde
  \[4\le\cycsum(a-1)^4\le 16.\]
  Como
  \[(x-1)^4=x^4-4x^3+6x^2-4x+1\]
  entonces
  \[
    4\cycsum a^3-\cycsum a^4
    =6\cycsum a^2-4\cycsum a+4-\cycsum(a-1)^4
    =52-\cycsum(a-1)^4
  \]
  de donde
  \[36\le 4\cycsum a^3-\cycsum a^4\le 48.\]
\end{proof}

\begin{probMG}[ISL 2010/A3]
  Sean $x_1,x_2,\dots,x_{100}\in\RR_0^+$ tales que $x_i+x_{i+1}+x_{i+2}\le 1$
  para todo $i=1,2,\dots,100$ (donde $x_{101}=x_1$, $x_{102}=x_2$). Determine el
  máximo valor de $S=\sum_{i=1}^{100}x_ix_{i+2}$.
  \forum[aops]{2362280}
  \begin{hint}
    Let $x_1,\dots,x_{100}$ be nonnegative real numbers such that
    $x_i+x_{i+1}+x_{i+2}\le 1$ for all $i=1,\dots,100$ (we put
    $x_{101}=x_1,x_{102}=x_2$). Find the maximal possible value of the sum
    $S=\sum^{100}_{i=1}x_ix_{i+2}$.
  \end{hint}
\end{probMG}

\begin{proof}
  Note que
  \[x_i^2+x_ix_{i+1}+x_ix_{i+2}\le x_i\]
  de donde
  \[
    S
    \le\half\sum(x_i+x_{i+1})(1-x_i-x_{i+1})
    \le\frac12\cdot\frac14\cdot 100
    =\frac{25}{2}
  \]
  y un ejemplo es cuando
  \[
    x_i=\begin{cases}
      \half & \text{si }2\mid i, \\
      0 & \text{si }2\nmid i.
    \end{cases}
  \]
\end{proof}

\begin{probEG}[ISL 2010/A4]
  Una secuencia $x_1,x_2,\dots$ está definida por $x_1=1$ y
  $x_{2k}=-x_k,\,x_{2k-1}=(-1)^{k+1}x_k$ para todo $k\ge 1$. Demuestre que
  $x_1+x_2+\dots+x_n\ge 0$ para todo $n\ge 1$.
  \forum[aops]{2362283}
  \begin{hint}
    A sequence $x_1,x_2,\dots$ is defined by $x_1=1$ and
    $x_{2k}=-x_k,\,x_{2k-1}=(-1)^{k+1}x_k$ for all $k\ge 1$. Prove that
    $x_1+x_2+\dots+x_n\ge 0$ for all $n\ge 1$.
  \end{hint}
\end{probEG}

\begin{proof}
  Pista: inducción de $x\mapsto 4x,4x+1,4x+2,4x+3$.
\end{proof}

\begin{probEG}[ISL 2010/A5]
  Determine todas las funciones $f:\QQ^+\to\QQ^+$ tales que
  \[f(f(x)^2y)=x^3f(xy)\]
  para todo $x,y\in\QQ^+$.
  \forum[aops]{2362286}
  \begin{hint}
    Determine all functions $f:\QQ^+\to\QQ^+$ which satisfy the following
    equation for all $x,y\in\QQ^+$:
    \[f(f(x)^2y)=x^3f(xy).\]
  \end{hint}
\end{probEG}

\begin{proof}
  Si $y=1$ entonces $f(f(x)^2)=x^3f(x)$ por lo que $f$ es inyectiva. Luego, si
  $y=f(y)^2$ entonces
  \[f(f(x)^2f(y)^2)=x^3f(xf(y)^2)=x^3y^3f(xy)=f(f(xy)^2)\]
  de donde $f$ es multiplicativa. Si $g(x)=\frac{f(f(x))}{x}$, tenemos
  $f(x)=xg(x)^2$ de donde $g(g(x))^4=g(x)^5$ y esto significa que si $g(x)\ne 1$
  y $N$ es el mínimo común múltiplo de todos los exponentes primos de $g(x)$,
  entonces $4N\mid 5N$ lo cual es un absurdo. Es decir, $g(x)=1$ de donde
  $f(x)=x$ para todo $x\in\QQ^+$.
\end{proof}

\begin{probHR}[ISL 2010/A6]
  Sean $f,g:\ZZ^+\to\ZZ^+$ tales que $f(g(n))=f(n)+1$ y $g(f(n))=g(n)+1$ para
  todo $n\in\ZZ^+$. Demuestre que $f(n)=g(n)$ para todo $n\in\ZZ^+$.
  \forum[aops]{2362289}
  \begin{hint}
    Suppose that $f$ and $g$ are two functions defined on the set of positive
    integers and taking positive integer values. Suppose also that the equations
    $f(g(n))=f(n)+1$ and $g(f(n))=g(n)+1$ hold for all positive integers. Prove
    that $f(n)=g(n)$ for all positive integer $n$.
  \end{hint}
\end{probHR}

\note{Miércoles\\2022-06-08}

\begin{probHR}[ISL 2010/A7\protect\footnote{IMO 2010/6}]
  Sean $a_1,a_2,\dots,a_r$ números reales positivos. Para $n>r$, inductivamente
  definimos
  \[a_n=\max_{1\le k\le n-1}(a_k+a_{n-k}).\]
  Demuestre que existen enteros positivos $\ell\le r$ y $N$ tales que
  $a_n=a_{n-\ell}+a_\ell$ para todo $n\ge N$.
  \forum[aops]{1936918}
  \begin{hint}
    \begin{otherlanguage*}{english}
      Let $a_1,\dots,a_r$ be positive real numbers. For $n>r$, we inductively
      define
      \[a_n=\max_{1\le k\le n-1}(a_k+a_{n-k}).\]
      Prove there exist positive integers $\ell\le r$ and $N$ such that
      $a_n=a_{n-\ell}+a_\ell$ for all $n\ge N$.
    \end{otherlanguage*}
  \end{hint}
\end{probHR}

\begin{probMR}[ISL 2010/A8]
  Sean $a,b,c,d,e,f\in\RR^+$ tales que $a<b<c<d<e<f$. Consideremos $a+c+e=S$ y
  $b+d+f=T$. Demuestre que
  \[2ST>\sqrt{3(S+T)\left(S(bd+bf+df)+T(ac+ae+ce)\right)}.\]
  \forum[aops]{2362291}
  \begin{hint}
    Given six positive numbers $a,b,c,d,e,f$ such that $a<b<c<d<e<f$. Let
    $a+c+e=S$ and $b+d+f=T$. Prove that
    \[2ST>\sqrt{3(S+T)\left(S(bd+bf+df)+T(ac+ae+ce)\right)}.\]
  \end{hint}
\end{probMR}

\begin{probEG}[ISL 2011/A2]
  Determine todas las secuencias $(x_1,$ $x_2,\dots,x_{2011})$ de enteros
  positivos tales que para todo entero positivo $n$ existe un entero positivo
  $a$ que satisface:
  \[x_1^n+2x_2^n+\dots+2011x_{2011}^n=a^{n+1}+1.\]
  \forum[aops]{2737640}
  \begin{hint}
    Determine all sequences $(x_1,x_2,\dots,x_{2011})$ of positive integers such
    that for every positive integer $n$ there is an integer $a$ with
    \[x_1^n+2x_2^n+\dots+2011x_{2011}^n=a^{n+1}+1.\]
  \end{hint}
\end{probEG}

\begin{probMG}[ISL 2011/A3]
  Determine todas las parejas de funciones $(f,g)$, con $f,g:\RR\to\RR$, tales
  que
  \[g(f(x+y))=f(x)+(2x+y)g(y)\]
  para todo $x,y\in\RR$.
  \forum[aops]{2737643}
  \begin{hint}
    Determine all pairs $(f,g)$ of functions from the set of real numbers to
    itself that satisfy
    \[g(f(x+y))=f(x)+(2x+y)g(y)\]
    for all real numbers $x$ and $y$.
  \end{hint}
\end{probMG}

\begin{problem}[ISL 2011/A4]
  Determine todas las parejas de funciones $(f,g)$, con $f,g:\ZZ^+\to\ZZ^+$,
  tales que
  \[f^{g(n)+1}(n)+g^{f(n)}(n)=f(n+1)-g(n+1)+1\]
  para todo $n\in\ZZ^+$.\\[4pt]
  \emph{Nota.} $f^k(n)=\underbrace{f(f(\dots f}_{k\text{ veces}}(n)\dots))$.
  \forum[aops]{2737644}
  \begin{hint}
    Determine all pairs $(f,g)$ of functions from the set of positive integers
    to itself that satisfy
    \[f^{g(n)+1}(n)+g^{f(n)}(n)=f(n+1)-g(n+1)+1\]
    for every positive integer $n$. Here, $f^k(n)$ means
    $\underbrace{f(f(\dots f}_k(n)\dots))$.
  \end{hint}
\end{problem}

\begin{proof}
  Sea $a_1\in\ZZ^+$ tal que $f(a_1)$ es mínimo. Note que si $a_1>1$, con
  $n=a_1-1$ tenemos $f(f^{g(n)}(n))<f(a_1)$ lo cual es un absurdo, por lo que
  $a_1=1$. Ahora, si $a_2>1$ es tal que $f(a_2)$ es el segundo mínimo, entonces
  con $n=a_2-1$ tenemos $f(f^{g(n)}(n))<f(a_2)$ de donde
  \[f(f^{g(n)-1}(n))=f^{g(n)}(n)=1.\]
  Es decir, $f^{g(n)-1}(n)=1$ y así sucesivamente hasta que $n=1$, por lo que
  $a_2=2$. Podemos probar de manera similar que $a_3=3,a_4=4,\dots$ y que
  $f(a_2)=2,f(a_3)=3,\dots$ de donde $f(n)=n$ para todo $n\in\ZZ^+$. Luego,
  $g^{f(n)}(n)+g(n+1)=2$ de donde $g(n)=1$ para todo $n\in\ZZ^+$.
\end{proof}

\note[Teoría de Números]{Jueves\\2022-06-09}

\begin{probEG}[ISL 2010/N2]
  Determine el mayor entero $n$ para el cual existe un conjunto
  $\{s_1,s_2,\dots,s_n\}$ que consiste de $n$ enteros positivos que satisfacen
  \[
    \left(1-\frac{1}{s_1}\right)\left(1-\frac{1}{s_2}\right)\cdots\left(1-\frac{1}{s_n}\right)
    =\frac{51}{2010}.
  \]
  \forum[aops]{2361998}
  \begin{hint}
    Find the least positive integer $n$ for which there exists a set
    $\{s_1,s_2,\dots,s_n\}$ consisting of $n$ distinct positive integers such
    that
    \[
      \left(1-\frac{1}{s_1}\right)\left(1-\frac{1}{s_2}\right)\cdots\left(1-\frac{1}{s_n}\right)
      =\frac{51}{2010}.
    \]
  \end{hint}
\end{probEG}

\begin{proof}
  Note que
  \[\frac{51}{2010}\ge\frac12\cdot\frac23\cdots\frac{n}{n+1}=\frac{1}{n+1}\]
  de donde $n\ge 39$, y un ejemplo es
  $(s_1,\dots,s_{39})=(2,3,\dots,33,35,36,\dots,40,67)$.
\end{proof}

\begin{probEB}[ISL 2010/N3]
  Determine el menor entero positivo $n$ para el cual existen polinomios
  $f_1,f_2,\dots,f_n$ con coeficientes racionales tales que
  \[x^2+7=f_1(x)^2+f_2(x)^2+\dots+f_n(x)^2.\]
  \forum[aops]{2362006}
  \begin{hint}
    Find the smallest number $n$ such that there exist polynomials
    $f_1,f_2,\dots,f_n$ with rational coefficients satisfying
    \[x^2+7=f_1(x)^2+f_2(x)^2+\dots+f_n(x)^2.\]
  \end{hint}
\end{probEB}

\begin{proof}
  Es claro que $\deg{f_i}\le 1$ así que sea $f_i(x)=a_ix+b_i$ donde
  $a_i,b_i\in\QQ$ para todo $1\le i\le 5$. Luego, $\sum a_i^2=1$,
  $\sum a_ib_i=0$ y $\sum b_i^2=7$. Vamos a multiplicar los $a_i$'s y $b_i$'s
  por $D\in\ZZ^+$ para que sean todos enteros (digamos que $D$ es el mínimo).
  Luego,
  \[\sum(a_i+b_i)^2=\sum(a_i-b_i)^2=8D^2.\]
  Si $n\le 4$, como $8\mid\sum(a_i\pm b_i)^2$ entonces $2\mid a_i-b_i$, de donde
  \[D^2=\sum a_i^2\equiv\sum b_i^2=7D^2\pmod 4.\]
  Es decir, $4\mid 6D^2$ de donde $2\mid D$. Luego,
  \[8\mid 2D^2=\sum\left(\frac{a_i\pm b_i}{2}\right)^2\]
  de donde $4\mid a_i\pm b_i$ y
  \[4\mid(a_i+b_i)+(a_i-b_i)=2a_i\]
  de donde $a_i$'s y $b_i$'s son pares. Es decir, podemos dividir los $a_i$'s,
  los $b_i$'s y el $D$ por $2$, lo cual es un absurdo a la minimalidad de $D$.
  Por ende, $n\ge 5$ y un ejemplo es $(f_1,\dots,f_5)=(x,2,1,1,1)$.
\end{proof}

\begin{probEG}[ISL 2010/G3\protect\footnote{Peru IMO TST 2011/2}]
  Sea $A_1A_2\cdots A_n$ un polígono convexo. El punto $P$ es escogido en el
  interior del polígono de tal manera que sus proyecciones $P_1,P_2,\dots,P_n$
  sobre las rectas $A_1A_2,A_2A_3,\dots,A_nA_1$, respectivamente, pertenecen a
  los lados del polígono (y no a sus prolongaciones). Pruebe que si
  $X_1,\dots,X_n$ son puntos arbitrarios que pertenecen a los lados
  $A_1A_2,\dots,A_nA_1$, respectivamente, se cumple la desigualdad:
  \[
    \max\left\{\frac{X_1X_2}{P_1P_2},\dots,\frac{X_nX_1}{P_nP_1}\right\}
    \ge 1.
  \]
  \forum[aops]{2361975}
  \begin{hint}
    \begin{otherlanguage*}{english}
      Let $A_1A_2\cdots A_n$ be a convex polygon. Point $P$ inside this polygon
      is chosen so that its projections $P_1,\dots,P_n$ onto lines
      $A_1A_2,\dots,A_nA_1$ respectively lie on the sides of the polygon. Prove
      that for arbitrary points $X_1,\dots,X_n$ on sides $A_1A_2,\dots,A_nA_1$
      respectively,
      \[
        \max\left\{\frac{X_1X_2}{P_1P_2},\dots,\frac{X_nX_1}{P_nP_1}\right\}
        \ge 1.
      \]
    \end{otherlanguage*}
  \end{hint}
\end{probEG}

\begin{proof}
  Supongamos por el absurdo que $X_iX_{i+1}<P_iP_{i+1}$ para todo índice
  $1\le i\le n$. Note que si $X_i$ está en el segmento $A_iP_i$, entonces
  $X_{i+1}$ está en el segmento $A_{i+1}P_{i+1}$ y así sucesivamente. Por ende,
  supongamos sin pérdida de generalidad que $X_i$ está en el segmento $A_iP_i$
  para todo $1\le i\le n$. Es claro que existe un $i$ tal que
  $\angle PX_iP_i\le\angle PX_{i+1}P_{i+1}$ (digamos $i=1$). Ahora, sea $Y$ un
  punto en el segmento $A_1P_1$ tal que $\angle PYP_1=\angle PX_2P_2$. Luego,
  $\triangle PYP_1\sim\triangle PX_2P_2$ de donde
  $\triangle PYX_2\sim\triangle PP_1P_2$. Por ende,
  \[X_1X_2\ge YX_2=\frac{PY}{PP_1}\cdot P_1P_2\ge P_1P_2\]
  lo cual es un absurdo.
\end{proof}

\begin{probMG}[ISL 2010/N4\protect\footnote{Peru IMO TST 2011/3}]
  Sean $a,b$ números enteros, y sea $P(x)=ax^3+bx$. Dado un entero positivo $n$,
  decimos que el par ordenado $(a,b)$ es \emph{n-bueno} si $n\mid P(m)-P(k)$
  implica que $n\mid m-k$ para todos los enteros $m,k$. Decimos también que el
  par $(a,b)$ es \emph{muy bueno} si $(a,b)$ es $n$-bueno para infinitos enteros
  positivos $n$.
  \begin{enumerate}[(a)]
    \ii Encuentre un par $(a,b)$ que sea $51$-bueno, pero que no sea muy bueno.
    \ii Demuestre que todos los pares $2010$-buenos también son muy buenos.
  \end{enumerate}
  \forum[aops]{2362008}
  \begin{hint}
    Let $a,b$ be integers, and let $P(x)=ax^3+bx$. For any positive integer $n$
    we say that the pair $(a,b)$ is \emph{n-good} if $n\mid P(m)-P(k)$ implies
    $n\mid m-k$ for all integers $m,k$. We say that $(a,b)$ is \emph{very good}
    if $(a,b)$ is $n$-good for infinitely many positive integers $n$.
    \begin{enumerate}[(a)]
      \ii Find a pair $(a,b)$ which is $51$-good, but not very good.
      \ii Show that all $2010$-good pairs are very good.
    \end{enumerate}
  \end{hint}
\end{probMG}

\begin{proof}
  Para la parte (a), considere el polinomio $P(x)=x^3-51^2x$ con
  $(a,b)=(1,-51^2)$. Luego, si $51\mid P(m)-P(k)$ entonces
  \[m\equiv m^3\equiv k^3\equiv k\pmod 3\]
  y
  \[m\equiv m^{33}\equiv k^{33}\equiv k\pmod 17\]
  lo cual implica que $51\mid m-k$. Por ende, $(a,b)$ es $51$-bueno. Si $n>51$ y
  $(m,k)=(51,0)$, note que $n\mid 0=P(m)-P(k)$ pero $n\nmid 51=m-k$, así que
  $(a,b)$ no es muy bueno.

  Para la parte (b), note que $P(m)-P(k)=(m-k)(a(m^2+mk+k^2)+b)$. Supongamos que
  existe un par de enteros $(m,k)$ tales que $67\mid a(m^2+mk+k^2)+b$. Sea
  $(m_1,k_1)$ un par de enteros tales que $m_1\equiv k_1\equiv 0\pmod 30$,
  $m_1\equiv m\pmod{67}$ y $k_1\equiv k\pmod{67}$, considerando que
  $2010=30\times 67$. Luego, $2010\mid P(m_1)-P(k_1)$ de donde
  $2010\mid m_1-k_1$, es decir, $m\equiv k\pmod{67}$. Ahora, sea $(m_2,k_2)$ un
  par de enteros tales que $m_2\equiv k_2\equiv 0\pmod 30$,
  $m_2\equiv m\pmod{67}$ y $k_2\equiv -2m\pmod{67}$. Luego,
  $2010\mid P(m_2)-P(k_2)$ de donde $2010\mid m_2-k_2$, es decir, $67\mid m$ y
  $67\mid b$. Ahora, sea $(m_3,k_3)$ un par de enteros tales que
  $m_3\equiv k_3\equiv 0\pmod 30$, $m_3\equiv 1\pmod{67}$ y
  $k_3\equiv 2^{22}\pmod{67}$. Luego, $2010\mid P(m_3)-P(k_3)$ de donde
  $2010\mid m_3-k_3$, es decir, $2^{22}\equiv 1\pmod{67}$ lo cual es un absurdo.
  Por ende, $67\nmid a(m^2+mk+k^2)+b$ para todo par de enteros $(m,k)$, lo cual
  implica que $(a,b)$ es $n$-bueno para todo $n=67^e$ siendo $e\in\ZZ^+$, es
  decir, $(a,b)$ es muy bueno.
\end{proof}

\begin{probEG}[ISL 2010/N5\protect\footnote{IMO 2010/3}]
  Determine todas las funciones $f:\NN\to\NN$ tales que $(f(m)+n)(m+f(n))$ es un
  cuadrado perfecto para todos los enteros positivos $m$ y $n$.
  \forum[aops]{1935854}
  \begin{hint}
    Find all functions $f:\NN\to\NN$ such that the number $(f(m)+n)(m+f(n))$ is
    a square for all $m,n\in\NN$.
  \end{hint}
\end{probEG}

\begin{proof}
  Se sabe que para todo primo $p$ se cumple lo siguiente: para todo $t\in\ZZ$
  existen $k_1,k_2\in\ZZ^+$ suficientemente grandes tales que $t=k_1-k_2$, donde
  $2\mid\nu_p(k_1),\nu_p(k_2)$. Ahora, sea $p$ un primo y supongamos que
  $p\mid f(m)-f(n)$ para algunos $m,n\in\ZZ^+$. Si $f(m)-f(n)=pt=pk_1-pk_2$, sea
  $N=pk_1-f(m)=pk_2-f(n)$ un entero positivo. Como
  $(f(m)+N)(m+f(N))=pk_1(m+f(N))$ es un cuadrado perfecto y $\nu_p(pk_1)$ es
  impar, entonces $p\mid m+f(N)$ y análogamente $p\mid n+f(N)$ de donde
  $p\mid m-n$. Si $f(a)=f(b)$ tenemos que $p\mid f(a)-f(b)$ para todo primo $p$,
  es decir, $p\mid a-b$ para todo primo $p$ de donde $a=b$. Si existe un primo
  $p$ que divide a $f(a+1)-f(a)$, entonces $p\mid (a+1)-a=1$ lo cual es un
  absurdo, así que $f(a+1)-f(a)=\pm 1$. Esto implica que $f(x)=x+c$ donde
  $c\in\ZZ^+_0$ es fijo.
\end{proof}

\begin{probEG}[ISL 2002/N3]
  Sean $p_1,p_2,\dots,p_n$ números primos distintos y mayores que $3$. Demuestre
  que el número $2^{p_1p_2\cdots p_n}+1$ tiene al menos $4^n$ divisores
  positivos.
  \forum[aops]{118690}
  \begin{hint}
    Let $p_1,p_2,\dots,p_n$ be distinct primes greater than $3$. Show that
    $2^{p_1p_2\cdots p_n}+1$ has at least $4^n$ divisors.
  \end{hint}
\end{probEG}

\begin{proof}[Solución 1]
  Procederemos por inducción sobre $n$. Si $n=1$, los números
  $1<3<\frac{N}{3}<N$ son los divisores del número $N=2^{p_1}+1$. Ahora, sea
  $n>1$ y supongamos que $a+1$ tiene al menos $4^{n-1}$ divisores, donde
  $a=2^{p_1p_2\cdots p_{n-1}}$ y $p_1<p_2<\cdots<p_n$. Note que
  \[
    \mcd(2^{p_1p_2\cdots p_{n-1}}+1,2^{p_n}+1)
    =2^{\mcd(p_1p_2\cdots p_{n-1},p_n)}+1
    =3
  \]
  así que $\frac{2^{p_n}+1}{3}$ divide a $N=\frac{a^{p_n}+1}{a+1}$. En efecto,
  sea $N=\frac{2^{p_n}+1}{3}\cdot M$. Note que
  \[\nu_3\left(\frac{2^{p_n}+1}{3}\right)=\nu_3(3)=1\]
  de donde si $p$ es un factor primo de $\frac{2^{p_n}+1}{3}$, entonces $p>3$ y
  $\ord_p(2)=2p_n\mid p-1$. Luego, $p>p_n$ de donde $p$ es coprimo con
  $p_1p_2\cdots p_n$. Por ende,
  \[
    \nu_p\left(\frac{2^{p_n}+1}{3}\right)
    =\nu_p(2^{p_n}+1)
    =\nu_p(2^{p_1p_2\cdots p_n}+1)
  \]
  de donde $p\nmid M$. Note que
  \[N=a^{p_n-1}-a^{p_n-2}+\dots+1\equiv p_n\pmod{a+1}\]
  de donde $\mcd(N,a+1)\mid p_n$. Como $\frac{2^{p_n}+1}{3}>p_n$ y $M>p_n$ son
  coprimos, existen factores primos $q_1\ne q_2$ de $\frac{2^{p_n}+1}{3}$ y $M$,
  coprimos con $a+1$. Por ende, $2^{p_1p_2\cdots p_n}+1$ tiene al menos
  $4^{n-1}\cdot 2\cdot 2=4^n$ divisores.
\end{proof}

\begin{proof}[Solución 2]
  Por Zsigmondy, para cada divisor $d$ de $p_1p_2\cdots p_n$ existe un factor
  primo primitivo de $2^d+1$ (excepto cuando $d=3$, lo cual no puede ocurrir),
  así que tenemos al menos $2^n$ factores primos de $2^{p_1p_2\cdots p_n}+1$ y
  por ende al menos $2^{2^n}\ge 4^n$ divisores.
\end{proof}

\note{Viernes\\2022-06-10}

\begin{probEG}[ISL 2002/N4]
  Determine si existe un entero positivo $m$ para el cual la ecuación
  \[\frac1a+\frac1b+\frac1c+\frac{1}{abc}=\frac{m}{a+b+c}\]
  tiene infinitas soluciones en los enteros positivos $a,b,c$.
  \forum[aops]{118691}
  \begin{hint}
    Is there a positive integer $m$ such that the equation
    \[\frac1a+\frac1b+\frac1c+\frac{1}{abc}=\frac{m}{a+b+c}\]
    has infinitely many solutions in positive integers $a,b,c$?
  \end{hint}
\end{probEG}

\begin{proof}
  Probaremos que $m=12$ cumple, y digamos que la terna $(a,b,c)$ de enteros
  positivos con $a<b<c$ satisface la ecuación dada y que además $a\mid bc+1$,
  $b\mid c+a$ y $c\mid ab+1$. Ahora, sea
  $(a',b',c')=\left(b,c,\frac{bc+1}{a}\right)$ una terna de enteros positivos.
  Note que la terna $(a',b',c')$ también satisface las condiciones anteriores y
  la ecuación dada, y que $a'+b'+c'>a+b+c$. Como la terna $(1,2,3)$ cumple, es
  claro que existen infinitas soluciones para la ecuación dada.
\end{proof}

\begin{probEG}[ISL 2002/N6]
  Determine todas las parejas de enteros positivos $m,n\ge 3$ para los cuales
  existen infinitos enteros positivos $a$ que satisface
  \[\frac{a^m+a-1}{a^n+a^2-1}\in\ZZ.\]
  \forum[aops]{118695}
  \begin{hint}
    Find all pairs of positive integers $m,n\ge 3$ for which there exist
    infinitely many positive integers $a$ such that
    \[\frac{a^m+a-1}{a^n+a^2-1}\]
    is itself an integer.
  \end{hint}
\end{probEG}

\begin{proof}
  Sean $Q,R\in\ZZ[x]$ con $\deg R<n$ tales que
  \[x^m+x-1=Q(x)\cdot(x^n+x^2-1)+R(x).\]
  Luego, $a^n+a^2-1\mid R(a)$ para infinitos $a\in\ZZ^+$ de donde $R(x)=0$, es
  decir,
  \[x^m+x-1=Q(x)\cdot(x^n+x^2-1).\]
  Ahora, sea $\eps\in(0,1)$ tal que $\eps^n+\eps^2=1$. Luego,
  \[a^n+a>a^n+a^2=a^m+a=1>a^{2n}+a\]
  de donde $m<2n$. Si $a=2$, entonces $2^n+3\mid 2^m+1$ de donde
  $2^n+3\mid 3\cdot 2^{m-n}-1$. Como $0<m-n<n$, se cumple la igualdad, entonces
  $(m,n)=(5,3)$. Note que
  \[x^5+x-1=(x^3+x^2-1)(x^2-x+1)\]
  así que la única pareja que cumple es $(5,3)$.
\end{proof}

\begin{probMG}[ISL 2003/N5]
  Un entero $n$ es llamado \emph{bueno} si $\abs n$ no es un cuadrado perfecto.
  Determine todos los números enteros $m$ con la siguiente propiedad: $m$ se
  puede representar de infinitas formas, como la suma de tres números buenos
  distintos cuyo producto es el cuadrado de un número entero impar.
  \forum[aops]{18558}
  \begin{hint}
    An integer $n$ is said to be \emph{good} if $\abs n$ is not the square of an
    integer. Determine all integers $m$ with the following property: $m$ can be
    represented, in infinitely many ways, as a sum of three distinct good
    integers whose product is the square of an odd integer.
  \end{hint}
\end{probMG}

\begin{proof}
  Es claro que esos tres números son de la forma $a^2yz,b^2zx,c^2xy$ donde
  $a,b,c\in\ZZ^+$ y $x,y,z\in\ZZ$ son impares, es decir,
  \[n=a^2yz+b^2zx+c^2xy\equiv\frac{(x+y+z)^2-x^2-y^2-z^2}{2}\equiv 3\pmod 4.\]
  Ahora, sea $p\equiv 5\pmod 8$ un número primo suficientemente grande y sea
  \[(x,y,z)=\left(-p,p+2n,\frac{1+pn(p+2n)}{2}\right)\]
  una terna de enteros impares coprimos. Como $x\equiv y\equiv 3\pmod 8$,
  entonces $n$ es la suma de los números buenos distintos $yz,zx,n^2xy$ cuyo
  producto es un cuadrado perfecto de un número impar. Finalmente, todo entero
  $n\equiv 3\pmod 4$ cumple.
\end{proof}

\begin{probEG}[ISL 2003/N6\protect\footnote{IMO 2003/6}]
  Sea $p$ un número primo. Demuestre que existe un número primo $q$ tal que para
  todo número entero $n$, el número $n^p-p$ no es múltiplo de $q$.
  \forum[aops]{266}
  \begin{hint}
    Let $p$ be a prime number. Prove that there exists a prime number $q$ such
    that for every integer $n$, the number $n^p-p$ is not divisible by $q$.
  \end{hint}
\end{probEG}

\begin{proof}
  Sea $q\not\equiv 1\pmod{p^2}$ un factor primo de
  \[\frac{p^p-1}{p-1}=p^{p-1}+p^{p-2}+\dots+1.\]
  Luego, $\ord_qp\mid\mcd(p,q-1)$ de donde $\ord_qp=p$ y $q=pk+1$ tal que
  $p\nmid k$. Ahora, si existe un $n\in\ZZ$ tal que $q\mid n^p-p$ entonces
  claramente $q\nmid n$ y
  \[p^k\equiv n^{pk}=n^{q-1}\equiv 1\pmod q\]
  de donde $p\mid k$ lo cual es un absurdo.
\end{proof}

\begin{probMG}[ISL 2005/N7]
  Sea $P(x)=a_nx^n+a_{n-1}x^{n-1}+\dots+a_0$, donde $a_0,\dots,a_n\in\ZZ$,
  $a_n>0$ y $n\ge 2$. Demuestre que existe un $m\in\ZZ^+$ para el cual $P(m!)$
  es un número compuesto.
  \forum[aops]{789443}
  \begin{hint}
    Let $P(x)=a_nx^n+a_{n-1}x^{n-1}+\dots+a_0$, where $a_0,\dots,a_n$ are
    integers, $a_n>0$, $n\ge 2$. Prove that there exists a positive integer $m$
    such that $P(m!)$ is a composite number.
  \end{hint}
\end{probMG}

\begin{proof}
  Si $\abs{a_0}\ne 1$, existe un primo $q$ tal que $q\mid a_0$. Es decir,
  $q\mid P(N!)>N!$ para algún $N\in\ZZ^+$ suficientemente grande, así que
  $P(N!)$ es compuesto. Si $a_0=\pm 1$, sea $m=p_i!+1$ para algún $i\in\ZZ^+$
  suficientemente grande, donde $p_i$ es el $i$-ésimo primo. Ahora, sea
  $P'(x)=a_0x^n+a_1x^{n-1}+\dots+a_n$, luego
  \[P\left(\frac{1}{m!}\right)=\frac{P'(m!)}{(m!)^n}=k\cdot\frac{a_n}{(m!)^n}\]
  donde
  \[
    k
    =a_0\cdot\frac{(m!)^n}{a_n}+\dots+a_{n-1}\cdot\frac{m!}{a_n}+1
    \equiv 1\;\;\left(\bmod\ \frac{m!}{a_n}\right).
  \]
  Es decir, existe un factor primo $q\ge p_i!+p_{i+1}$ de $P'(m!)$ tal que
  $N!=(q-p_i!-2)!\ge q$, luego
  \begin{align*}
    N!
    &= (q-p_i!-2)! \\
    &= \frac{(q-1)!}{(q-p_i!-1)\cdots(q-2)(q-1)} \\
    &\equiv \frac{-1}{(-1)^m\cdot m!} \\
    &= \frac{1}{m!}\pmod q
  \end{align*}
  de donde
  \[P(N!)\equiv P\left(\frac{1}{m!}\right)\equiv 0\pmod q.\]
  Luego, $q\mid P(N!)>N!$ así que $P(N!)$ es compuesto.
\end{proof}

\begin{probEG}[ISL 2009/N4]
  Determine todos los $n\in\ZZ^+$ para los cuales existe una secuencia
  $a_1,a_2,\dots,a_n$ que satisface
  \[a_{k+1}=\frac{a_k^2+1}{a_{k-1}+1}-1\]
  para todo $2\le k\le n-1$.
  \forum[aops]{1932944}
  \begin{hint}
    Find all positive integers $n$ such that there exists a sequence of positive
    integers $a_1,a_2,\dots,a_n$ satisfying
    \[a_{k+1}=\frac{a_k^2+1}{a_{k-1}+1}-1\]
    for every $k$ with $2\le k\le n-1$.
  \end{hint}
\end{probEG}

\begin{proof}
  Para $n\le 4$, considere la secuencia $4,33,217,1384$. Ahora, suponga que
  $n\ge 5$. Note que si $a_{k-1}$ es impar entonces $a_k$ es impar y $a_{k+1}$
  es par para todo $2\le k\le n-1$, así que si $a_2$ es impar, entonces $a_3$ es
  impar y $a_4$ es par, pero $a_4$ es impar ya que $a_3$ es impar, lo cual es un
  absurdo. Es decir, si $(a,b)=(a_1,a_2)$ tenemos que $2\mid b$ y
  \[
    (a_3,a_4)
    =\left(\frac{b^2-a}{a+1},\frac{(b^2-a)^2-b(a+1)^2}{(a+1)^2(b+1)}\right)
  \]
  de donde $a+1\mid b^2+1$ y $b+1\mid a^2+1$. Si $d=\mcd(a+1,b+1)>1$ entonces
  $d=2$, pero $2\nmid b+1$ lo cual es un absurdo. Luego,
  \[k=\frac{a^2+b^2}{(a+1)(b+1)}\in\ZZ^+.\]
  Ahora, fijemos $k$ y supongamos que $(a,b)$ es la pareja que cumple la
  ecuación de arriba y que además la suma $a+b$ es la mínima. Note que si
  $(a,b)$ cumple entonces $a+1\mid b^2+1$ de donde $a\le b^2$. Si $a=b^2$
  entonces de $b+1\mid a^2+1$ tenemos que $b+1\mid 2$, de donde $a=b=1$, luego
  $k=\half\not\in\ZZ^+$ lo cual es un absurdo. Es decir, $a<b^2$ y la pareja
  $\left(\frac{b^2-a}{a+1},b\right)$ también cumple, así que
  \[\frac{b^2}{a}>\frac{b^2-a}{a+1}\ge a\]
  de donde $b>a$ y análogamente $a>b$ lo cual es un absurdo.
\end{proof}

\begin{probEB}[ISL 2008/N1]
  Sea $n$ un entero positivo y sea $p$ un número primo. Demuestre que si los
  números enteros $a$, $b$ y $c$ (no necesariamente positivos) satisfacen las
  ecuaciones
  \[a^n+pb=b^n+pc=c^n+pa\]
  entonces $a=b=c$.
  \forum[aops]{1555927}
  \begin{hint}
    Let $n$ be a positive integer and let $p$ be a prime number. Prove that if
    $a,b,c$ are integers (not necessarily positive) satisfying the equations
    \[a^n+pb=b^n+pc=c^n+pa\]
    then $a=b=c$.
  \end{hint}
\end{probEB}

\begin{proof}
  Note que si $a=b$ entonces $a=b=c$. Ahora, supongamos que $a,b,c$ son
  distintos, luego
  \[a-b\mid b^n-a^n=p(b-c)\]
  y análogamente $b-c\mid p(c-a)$ y $c-a\mid p(a-b)$. Si $q\ne p$ es un primo,
  Entonces $\nu_q(a-b)\le\nu_q(b-c)$ y análogamente con $\nu_q(b-c)$ y
  $\nu_q(c-a)$, así que $\nu_q(a-b)=\nu_q(b-c)=\nu_q(c-a)$. Es decir,
  \[
    (\abs{a-b},\abs{b-c},\abs{c-a})
    =(M\cdot p^\alpha,M\cdot p^\beta,M\cdot p^\gamma)
  \]
  para algún entero positivo impar $M$ y $\alpha,\beta,\gamma\in\ZZ^+_0$ donde
  $\alpha=\max(\alpha,\beta,\gamma)$ sin pérdida de generalidad. Luego,
  $\pm p^\alpha\pm p^\beta\pm p^\gamma=0$ de donde $p=2$. Entonces, $a,b,c$
  tienen la misma paridad, de donde
  \[
    \frac{a^n-b^n}{a-b}
    \equiv\frac{b^n-c^n}{b-c}
    \equiv\frac{c^n-a^n}{c-a}
    \equiv na^{n-1}\pmod 2.
  \]
  Si $na^{n-1}$ es impar, tenemos
  \[\nu_2(a-b)<\nu_2(2(a-b))=\nu_2(c^n-a^n)=\nu_2(c-a)\]
  lo cual es un absurdo, así que $na^{n-1}$ es par, de donde
  $a-b\mid\frac{b^n-a^n}{2}=b-c$ y análogamente $b-c\mid c-a$ y $c-a\mid a-b$.
  Luego, $\abs{a-b}=\abs{b-c}=\abs{c-a}$ y $a=b=c$, lo cual es un absurdo.
\end{proof}

\begin{probEG}[ISL 2008/N2]
  Sean $a_1,a_2,\dots,a_n$ números enteros positivos distintos, con $n\ge 3$.
  Demuestre que existen índices distintos $i$ y $j$ de tal manera que la suma
  $a_i+a_j$ no divide a ninguno de los números $3a_1,3a_2,\dots,3a_n$.
  \forum[aops]{1555929}
  \begin{hint}
    Let $a_1,a_2,\dots,a_n$ be distinct positive integers, $n\ge 3$. Prove that
    there exist distinct indices $i$ and $j$ such that $a_i+a_j$ does not divide
    any of the numbers $3a_1,3a_2,\dots,3a_n$.
  \end{hint}
\end{probEG}

\begin{proof}
  Supongamos por el absurdo que para todo $i\ne j$ existe $k$ tal que
  $a_i+a_j\mid 3a_k$. Sin pérdida de generalidad, supongamos que
  $a_1<a_2<\dots<a_n$ y que $\mcd(a_1,a_2,\dots,a_n)=1$. Si $1\le i\le n-1$,
  existe un $k$ tal que $a_i+a_n=3a_k$ o $a_i+a_n=\frac32a_k$, es decir,
  $a_i\equiv -a_n\pmod 3$ para todo $1\le i\le n-1$ de donde $3\ne a_n$.
  Entonces, si $1\le i\le n-2$ existe un $k$ tal que $a_i+a_{n-1}\mid 3a_k$, de
  donde $a_i+a_{n-1}\mid a_k$. Luego, $a_{n-1}<a_k$ de donde $k=n$. Si
  $2a_i+2a_{n-1}\le a_n$, entonces $2a_{n-1}<a_n$. Sea $k$ tal que
  $a_{n-1}+a_n=3a_k$ o $a_{n-1}+a_n=\frac32a_k$. Luego,
  $3a_{n-1}<a_{n-1}+a_n\le 3a_k$ de donde $k=n$. Por ende, $a_{n-1}=2a_n$ o
  $2a_{n-1}=a_n$, lo cual es un absurdo. Es decir, $a_i=a_n-a_{n-1}$ para todo
  $1\le i\le n-2$, de donde $n=3$. Luego, $3a_1<a_2+a_3\mid 3a_k<2a_2+2a_3$ para
  algún $k>1$, de donde $a_2+a_3=3a_2$ o $a_2+a_3=3a_3$. Por ende, $a_3=2a_2$ o
  $a_2=2a_3$ lo cual es un absurdo.
\end{proof}

\begin{probEG}[ISL 2009/N2]
  Un entero positivo $N$ es llamado \emph{balanceado} si $N=1$ o si $N$ puede
  representarse como el producto de una cantidad par de números primos, no
  necesariamente distintos. Dados los enteros positivos $a$ y $b$, considere el
  polinomio $P(x)=(x+a)(x+b)$.
  \begin{enumerate}[(a)]
    \ii Demuestre que existen dos enteros positivos distintos $a$ y $b$ para los
    cuales cada uno de los números $P(1),P(2),\dots,P(50)$ es balanceado.
    \ii Demuestre que si $P(n)$ es balanceado para todo entero positivo $n$,
    entonces $a=b$.
  \end{enumerate}
  \forum[aops]{1932941}
  \begin{hint}
    A positive integer $N$ is called \emph{balanced} if $N=1$ or if $N$ can be
    written as a product of an even number of not necessarily distinct primes.
    Given positive integers $a$ and $b$, consider the polynomial $P$ defined by
    $P(x)=(x+a)(x+b)$.
    \begin{enumerate}[(a)]
      \ii Prove that there exist distinct positive integers $a$ and $b$ such
      that all the number $P(1),P(2),\dots,P(50)$ are balanced.
      \ii Prove that if $P(n)$ is balanced for all positive integers $n$, then
      $a=b$.
    \end{enumerate}
  \end{hint}
\end{probEG}

\begin{proof}
  Si $f(n)=\sum_{p\text{ es primo}}\nu_p(n)$, entonces $N$ es balanceado si y
  solo si $f(N)$ es par. Es decir, $P(n)$ es balanceado si y solo si
  $f(x+a)\equiv f(x+b)\pmod 2$. Ahora, por el principio de las casillas existen
  enteros positivos $a\ne b$ tales que
  \[(f(a+1),f(a+2),\dots,f(a+50))\equiv(f(b+1),f(b+2),\dots,f(b+50))\pmod 2\]
  de donde $P(1),P(2),\dots,P(50)$ es balanceado. Ahora, si $P(n)$ es balanceado
  para todo $n\in\ZZ^+$, entonces $f(n+a)\equiv f(n+b)\pmod 2$ para todo
  $n\in\ZZ^+$. Si $a>b$, entonces por Dirichlet existe un primo $p=(a-b)k+1>b$
  para algún $k\in\ZZ^+$, de donde
  \[1=f(p)\equiv f(p^2)\equiv 0\pmod 2\]
  lo cual es un absurdo, así que $a=b$.
\end{proof}

\begin{probMG}[ISL 2005/N2\protect\footnote{IMO 2005/2}]
  Sea $a_1,a_2,\dots$ una secuencia de números enteros con infinitos términos
  positivos e infinitos términos negativos. Suponga que para cada entero
  positivo $n$, los números $a_1,a_2,\dots,a_n$ tienen $n$ restos distintos al
  ser divididos entre $n$. Demuestre que cada número entero aparece exactamente
  una vez en la secuencia.
  \forum[aops]{281572}
  \begin{hint}
    Let $a_1,a_2,\dots$ be a sequence of integers with infinitely many positive
    terms and infinitely many negative terms. Suppose that for every positive
    integer $n$, the numbers $a_1,a_2,\dots,a_n$ leave $n$ different remainders
    on division by $n$. Prove that every integer occurs exactly once in the
    sequence.
  \end{hint}
\end{probMG}

\begin{proof}
  Sea $N\in\ZZ^+$ suficientemente grande. Note que si $N\mid a_i-a_j$ para
  algunos $i\ne j$, por la condición tenemos $\max(i,j)>N$, es decir, ningún
  entero se repite en la secuencia y además $\max(i,j)>\abs{a_i-a_j}$ para todo
  $i\ne j$. Si existe un $a\in\ZZ$ que no aparece en la secuencia, por la
  condición existe un $1\le i\le N$ tal que $a_i\equiv a\pmod N$, de donde
  $a_i\ge a+N$ o $a_i\le a-N$. Además, $N\ge\max(i,j)>\abs{a_i-a_j}$ para todo
  $1\le j\ne i\le N$, de donde $N>a_i-a_j>-N$. Si $a_i\ge a+N>a$, entonces
  $a_j>a_i-N\ge a$ de donde $a_k>a$ para todo $1\le k\le N$. Análogamente, si
  $a_i\le a-N<a$, entonces $a_k<a$ para todo $1\le k\le N$, de donde $a_k>a$
  para todo $k\in\ZZ^+$ o $a_k<a$ para todo $k\in\ZZ^+$, lo cual contradice la
  infinidad de términos negativos o positivos. Es decir, cada entero aparece
  exactamente una vez en la secuencia.
\end{proof}

\begin{probEG}[ISL 2008/N6\protect\footnote{IMO 2008/3}]
  Demuestre que existen infinitos $n\in\ZZ^+$ para los cuales $n^2+1$ tiene un
  factor primo mayor que $2n+\sqrt{2n}$.
  \forum[aops]{1190546}
  \begin{hint}
    Prove that there are infinitely many positive integers $n$ such that $n^2+1$
    has a prime divisor greater than $2n+\sqrt{2n}$.
  \end{hint}
\end{probEG}

\begin{proof}
  Sea $p\equiv 1\pmod 4$ un primo suficientemente grande. Luego, existe un
  entero $x=\frac{p-1}{2}-m$ donde $1<m<\frac{p-1}{2}$ tal que $p\mid x^2+1$, de
  donde $p\mid(2m+1)^2+4$. Luego, $p<(2m+1)^2+2m+1$ de donde
  $2x=p-2m-1<(2m+1)^2$ y $\sqrt{2x}<2m+1=p-2x$ de donde $p>2x+\sqrt{2x}$. Como
  $x\ge\sqrt{p-1}$, $n^2+1$ tiene un factor primo mayor que $2n+\sqrt{2n}$ para
  infinitos $n\in\ZZ^+$.
\end{proof}

%\section{Semana 14 (06/13 -- 06/19)}

\note{Lunes\\2022-06-13}

%\section{Semana 15 (06/20 -- 06/26)}

%\note[]{Lunes\\2022-06-20}
%\note[]{Martes\\2022-06-21}
%\note[]{Miércoles\\2022-06-22}
%\note[]{Jueves\\2022-06-23}
%\note[]{Viernes\\2022-06-24}


\section{Enunciados en Inglés}

Aquí pueden ver los enunciados originales (o traducidos al inglés) de los problemas.

\makehints
\clearpage
\printbibliography

\end{document}
